% Generated by Sphinx.
\def\sphinxdocclass{report}
\documentclass[a4paper,12pt,polish]{sphinxmanual}
\usepackage[utf8]{inputenc}
\DeclareUnicodeCharacter{00A0}{\nobreakspace}
\usepackage[T1]{fontenc}

\usepackage{times}
\usepackage[Sonny]{fncychap}
\usepackage{longtable}
\usepackage{sphinx}
\usepackage{multirow}
\usepackage{amsmath,amssymb}
\usepackage{babel}
\usepackage{tcolorbox}

\makeatletter
\g@addto@macro\@verbatim\footnotesize
\makeatother

\definecolor{niceblue}{HTML}{E7F0FE}
\definecolor{nicedarkerblue}{HTML}{93B7EC}

\makeatletter\newenvironment{icsebox}{\begin{tcolorbox}[colframe=nicedarkerblue,colback=niceblue,leftrule=3mm]}{\end{tcolorbox}}
\renewenvironment{notice}[2]{\begin{icsebox}\def\py@noticetype{#1}\par\strong{#2}}{\end{icsebox}}\makeatother


\title{Dynamika stochastyczna}
\date{23/10/2013}
\release{I}
\author{Jerzy Łuczka, Łukasz Machura}
\newcommand{\sphinxlogo}{}
\renewcommand{\releasename}{Wydanie}
\makeindex

\makeatletter
\def\PYG@reset{\let\PYG@it=\relax \let\PYG@bf=\relax%
    \let\PYG@ul=\relax \let\PYG@tc=\relax%
    \let\PYG@bc=\relax \let\PYG@ff=\relax}
\def\PYG@tok#1{\csname PYG@tok@#1\endcsname}
\def\PYG@toks#1+{\ifx\relax#1\empty\else%
    \PYG@tok{#1}\expandafter\PYG@toks\fi}
\def\PYG@do#1{\PYG@bc{\PYG@tc{\PYG@ul{%
    \PYG@it{\PYG@bf{\PYG@ff{#1}}}}}}}
\def\PYG#1#2{\PYG@reset\PYG@toks#1+\relax+\PYG@do{#2}}

\def\PYG@tok@gd{\def\PYG@tc##1{\textcolor[rgb]{0.63,0.00,0.00}{##1}}}
\def\PYG@tok@gu{\let\PYG@bf=\textbf\def\PYG@tc##1{\textcolor[rgb]{0.50,0.00,0.50}{##1}}}
\def\PYG@tok@gt{\def\PYG@tc##1{\textcolor[rgb]{0.00,0.25,0.82}{##1}}}
\def\PYG@tok@gs{\let\PYG@bf=\textbf}
\def\PYG@tok@gr{\def\PYG@tc##1{\textcolor[rgb]{1.00,0.00,0.00}{##1}}}
\def\PYG@tok@cm{\let\PYG@it=\textit\def\PYG@tc##1{\textcolor[rgb]{0.25,0.50,0.56}{##1}}}
\def\PYG@tok@vg{\def\PYG@tc##1{\textcolor[rgb]{0.73,0.38,0.84}{##1}}}
\def\PYG@tok@m{\def\PYG@tc##1{\textcolor[rgb]{0.13,0.50,0.31}{##1}}}
\def\PYG@tok@mh{\def\PYG@tc##1{\textcolor[rgb]{0.13,0.50,0.31}{##1}}}
\def\PYG@tok@cs{\def\PYG@tc##1{\textcolor[rgb]{0.25,0.50,0.56}{##1}}\def\PYG@bc##1{\colorbox[rgb]{1.00,0.94,0.94}{##1}}}
\def\PYG@tok@ge{\let\PYG@it=\textit}
\def\PYG@tok@vc{\def\PYG@tc##1{\textcolor[rgb]{0.73,0.38,0.84}{##1}}}
\def\PYG@tok@il{\def\PYG@tc##1{\textcolor[rgb]{0.13,0.50,0.31}{##1}}}
\def\PYG@tok@go{\def\PYG@tc##1{\textcolor[rgb]{0.19,0.19,0.19}{##1}}}
\def\PYG@tok@cp{\def\PYG@tc##1{\textcolor[rgb]{0.00,0.44,0.13}{##1}}}
\def\PYG@tok@gi{\def\PYG@tc##1{\textcolor[rgb]{0.00,0.63,0.00}{##1}}}
\def\PYG@tok@gh{\let\PYG@bf=\textbf\def\PYG@tc##1{\textcolor[rgb]{0.00,0.00,0.50}{##1}}}
\def\PYG@tok@ni{\let\PYG@bf=\textbf\def\PYG@tc##1{\textcolor[rgb]{0.84,0.33,0.22}{##1}}}
\def\PYG@tok@nl{\let\PYG@bf=\textbf\def\PYG@tc##1{\textcolor[rgb]{0.00,0.13,0.44}{##1}}}
\def\PYG@tok@nn{\let\PYG@bf=\textbf\def\PYG@tc##1{\textcolor[rgb]{0.05,0.52,0.71}{##1}}}
\def\PYG@tok@no{\def\PYG@tc##1{\textcolor[rgb]{0.38,0.68,0.84}{##1}}}
\def\PYG@tok@na{\def\PYG@tc##1{\textcolor[rgb]{0.25,0.44,0.63}{##1}}}
\def\PYG@tok@nb{\def\PYG@tc##1{\textcolor[rgb]{0.00,0.44,0.13}{##1}}}
\def\PYG@tok@nc{\let\PYG@bf=\textbf\def\PYG@tc##1{\textcolor[rgb]{0.05,0.52,0.71}{##1}}}
\def\PYG@tok@nd{\let\PYG@bf=\textbf\def\PYG@tc##1{\textcolor[rgb]{0.33,0.33,0.33}{##1}}}
\def\PYG@tok@ne{\def\PYG@tc##1{\textcolor[rgb]{0.00,0.44,0.13}{##1}}}
\def\PYG@tok@nf{\def\PYG@tc##1{\textcolor[rgb]{0.02,0.16,0.49}{##1}}}
\def\PYG@tok@si{\let\PYG@it=\textit\def\PYG@tc##1{\textcolor[rgb]{0.44,0.63,0.82}{##1}}}
\def\PYG@tok@s2{\def\PYG@tc##1{\textcolor[rgb]{0.25,0.44,0.63}{##1}}}
\def\PYG@tok@vi{\def\PYG@tc##1{\textcolor[rgb]{0.73,0.38,0.84}{##1}}}
\def\PYG@tok@nt{\let\PYG@bf=\textbf\def\PYG@tc##1{\textcolor[rgb]{0.02,0.16,0.45}{##1}}}
\def\PYG@tok@nv{\def\PYG@tc##1{\textcolor[rgb]{0.73,0.38,0.84}{##1}}}
\def\PYG@tok@s1{\def\PYG@tc##1{\textcolor[rgb]{0.25,0.44,0.63}{##1}}}
\def\PYG@tok@gp{\let\PYG@bf=\textbf\def\PYG@tc##1{\textcolor[rgb]{0.78,0.36,0.04}{##1}}}
\def\PYG@tok@sh{\def\PYG@tc##1{\textcolor[rgb]{0.25,0.44,0.63}{##1}}}
\def\PYG@tok@ow{\let\PYG@bf=\textbf\def\PYG@tc##1{\textcolor[rgb]{0.00,0.44,0.13}{##1}}}
\def\PYG@tok@sx{\def\PYG@tc##1{\textcolor[rgb]{0.78,0.36,0.04}{##1}}}
\def\PYG@tok@bp{\def\PYG@tc##1{\textcolor[rgb]{0.00,0.44,0.13}{##1}}}
\def\PYG@tok@c1{\let\PYG@it=\textit\def\PYG@tc##1{\textcolor[rgb]{0.25,0.50,0.56}{##1}}}
\def\PYG@tok@kc{\let\PYG@bf=\textbf\def\PYG@tc##1{\textcolor[rgb]{0.00,0.44,0.13}{##1}}}
\def\PYG@tok@c{\let\PYG@it=\textit\def\PYG@tc##1{\textcolor[rgb]{0.25,0.50,0.56}{##1}}}
\def\PYG@tok@mf{\def\PYG@tc##1{\textcolor[rgb]{0.13,0.50,0.31}{##1}}}
\def\PYG@tok@err{\def\PYG@bc##1{\fcolorbox[rgb]{1.00,0.00,0.00}{1,1,1}{##1}}}
\def\PYG@tok@kd{\let\PYG@bf=\textbf\def\PYG@tc##1{\textcolor[rgb]{0.00,0.44,0.13}{##1}}}
\def\PYG@tok@ss{\def\PYG@tc##1{\textcolor[rgb]{0.32,0.47,0.09}{##1}}}
\def\PYG@tok@sr{\def\PYG@tc##1{\textcolor[rgb]{0.14,0.33,0.53}{##1}}}
\def\PYG@tok@mo{\def\PYG@tc##1{\textcolor[rgb]{0.13,0.50,0.31}{##1}}}
\def\PYG@tok@mi{\def\PYG@tc##1{\textcolor[rgb]{0.13,0.50,0.31}{##1}}}
\def\PYG@tok@kn{\let\PYG@bf=\textbf\def\PYG@tc##1{\textcolor[rgb]{0.00,0.44,0.13}{##1}}}
\def\PYG@tok@o{\def\PYG@tc##1{\textcolor[rgb]{0.40,0.40,0.40}{##1}}}
\def\PYG@tok@kr{\let\PYG@bf=\textbf\def\PYG@tc##1{\textcolor[rgb]{0.00,0.44,0.13}{##1}}}
\def\PYG@tok@s{\def\PYG@tc##1{\textcolor[rgb]{0.25,0.44,0.63}{##1}}}
\def\PYG@tok@kp{\def\PYG@tc##1{\textcolor[rgb]{0.00,0.44,0.13}{##1}}}
\def\PYG@tok@w{\def\PYG@tc##1{\textcolor[rgb]{0.73,0.73,0.73}{##1}}}
\def\PYG@tok@kt{\def\PYG@tc##1{\textcolor[rgb]{0.56,0.13,0.00}{##1}}}
\def\PYG@tok@sc{\def\PYG@tc##1{\textcolor[rgb]{0.25,0.44,0.63}{##1}}}
\def\PYG@tok@sb{\def\PYG@tc##1{\textcolor[rgb]{0.25,0.44,0.63}{##1}}}
\def\PYG@tok@k{\let\PYG@bf=\textbf\def\PYG@tc##1{\textcolor[rgb]{0.00,0.44,0.13}{##1}}}
\def\PYG@tok@se{\let\PYG@bf=\textbf\def\PYG@tc##1{\textcolor[rgb]{0.25,0.44,0.63}{##1}}}
\def\PYG@tok@sd{\let\PYG@it=\textit\def\PYG@tc##1{\textcolor[rgb]{0.25,0.44,0.63}{##1}}}

\def\PYGZbs{\char`\\}
\def\PYGZus{\char`\_}
\def\PYGZob{\char`\{}
\def\PYGZcb{\char`\}}
\def\PYGZca{\char`\^}
\def\PYGZsh{\char`\#}
\def\PYGZpc{\char`\%}
\def\PYGZdl{\char`\$}
\def\PYGZti{\char`\~}
% for compatibility with earlier versions
\def\PYGZat{@}
\def\PYGZlb{[}
\def\PYGZrb{]}
\makeatother

\begin{document}
\shorthandoff{"}
\maketitle
\tableofcontents
\phantomsection\label{index::doc}

\begin{quote}\begin{description}
\item[{Autorzy}] \leavevmode
Jerzy Łuczka,
Łukasz Machura

\item[{Wersja}] \leavevmode
0.2 10/2013

\item[{Pobierz}] \leavevmode
\code{podręcznik (v0.2, PDF)}

\end{description}\end{quote}


\chapter{Dynamika deterministyczna}
\label{index:dynamika-stochastyczna}\label{index:dynamika-deterministyczna}

\section{Opis i modelowanie zjawisk oraz procesów przy pomocy równań różniczkowych}
\label{ch1/chI011:opis-i-modelowanie-zjawisk-oraz-procesow-przy-pomocy-rownan-rozniczkowych}\label{ch1/chI011::doc}
Jednym z podstawowych praw fizyki, jakie poznajemy w szkole średniej jest II zasada dynamiki Newtona. Opisuje ona klasyczne układy mechaniczne. Układy te są idealizacją realnych układów występujących w otaczającym nas świecie. W najprostszej wersji II zasada dynamiki Newtona w odniesieniu do jednej cząstki poruszającej się tylko wzdłuż jednej osi współrzędnych, np. wzdłuż osi OX, może być sformułowana w następującej postaci:
\begin{quote}

Ruch cząstki jest zdeterminowany przez siły jakie działają na cząstkę
\end{quote}

Z punktu widzenia matematycznego, ruch cząstki opisany jest przez równanie Newtona:
\phantomsection\label{ch1/chI011:equation-eqn1}\begin{gather}
\begin{split}m a = F\end{split}\label{ch1/chI011-eqn1}
\end{gather}
W równaniu tym występują trzy wielkości:

$m$  to masa cząstki
$a$ jest przyśpieszeniem cząstki
$F$ jest siłą działającą na czastkę.
Ponieważ ruch zachodzi tylko wzdłuż osi OX (tak zakładamy dla prostoty), siła  $F$  działa tylko w kierunku OX oraz przyśpieszenie $a$ jest wzdłuż osi OX.

Wiemy z kursu fizyki, że przyśpieszenie cząstki jest pochodną ( względem czasu) pierwszego rzędu prędkości $v$ cząstki. Z kolei prędkość cząstki $v$ jest pochodną pierwszego rzędu położenia czastki $x$.
\phantomsection\label{ch1/chI011:equation-eqn2}\begin{gather}
\begin{split}a= \frac{d}{dt} v= \frac{d}{dt} \frac{d}{dt} x = \frac{d^2x}{dt^2}\end{split}\label{ch1/chI011-eqn2}
\end{gather}
W ogólnej postaci siła
\phantomsection\label{ch1/chI011:equation-eqn3}\begin{gather}
\begin{split}F = F(x, v, t) = F(x, dx/dt, t)\end{split}\label{ch1/chI011-eqn3}
\end{gather}
może zależeć od położenia $x$ cząstki, jej prędkości $v=dx/dt$ oraz czasu $t$.

Jeżeli przyśpieszenie $a$ oraz siłę $F$ w takiej postaci podstawimy do równania Newtona, to jego postać jest następująca:
\phantomsection\label{ch1/chI011:equation-eqn4}\begin{gather}
\begin{split} m  \frac{d^2x}{dt^2} = F\left(x, \frac{dx}{dt}, t\right) \qquad\end{split}\label{ch1/chI011-eqn4}
\end{gather}
W ten sposób otrzymujemy równanie różniczkowe, które opisuje jednowymiarowy ruch cząstki wzdłuż osi OX.  Co możemy powiedzieć o tym równaniu:

Jest to równanie różniczkowe drugiego rzędu, ponieważ pojawia się  pochodna drugiego rzędu $d^2x/dt^2$.
Jest to równanie różniczkowe zwyczajne, ponieważ  nie występują pochodne cząstkowe a jedynie pochodne ze względu na jedną zmienną - w tym przypadku pochodne względem czasu $t$.
Samo równanie Newtona nie wystarczy, aby opisać ruch cząstki. Musimy zadać warunki początkowe dla tego równania. Ponieważ jest to równanie drugiego rzędu, musimy zadać dwa warunki początkowe: początkowe położenie $x(t_0) = x_0$ oraz początkową  prędkość $v(t_0) = v_0$. Warunki początkowe można zadać w dowolnej chwili czasu $t_0$, ale zazwyczaj tą chwilą początkową jest umowna chwila  $t_0 = 0$.
Równanie \eqref{ch1/chI011-eqn4} możemy zatem przedstawić w równoważnej postaci:
\phantomsection\label{ch1/chI011:equation-eqn5}\begin{gather}
\begin{split}\frac{dx}{dt} = v \qquad\end{split}\label{ch1/chI011-eqn5}
\end{gather}\phantomsection\label{ch1/chI011:equation-eqn6}\begin{gather}
\begin{split}\frac{dv}{dt} = \frac{1}{m} F\left(x, v, t\right) \qquad\end{split}\label{ch1/chI011-eqn6}
\end{gather}
gdzie wprowadziliśmy nową zmienną $v$ która ma interpretację prędkości cząstki. W ten sposób otrzymaliśmy układ 2 równań różniczkowych pierwszego rzędu. Jak później zobaczymy, taka manipulacja jest użyteczna przy wprowadzeniu pojęcia przestrzeni fazowej dla równań różniczkowych.  Jeżeli siła $F$ nie zależy w sposób jawny od czasu, to układ równań
\phantomsection\label{ch1/chI011:equation-eqn7}\begin{gather}
\begin{split}\frac{dx}{dt} = v \qquad\end{split}\label{ch1/chI011-eqn7}
\end{gather}\phantomsection\label{ch1/chI011:equation-eqn8}\begin{gather}
\begin{split} m \frac{dv}{dt} =  F(x, v) \qquad\end{split}\label{ch1/chI011-eqn8}
\end{gather}
nazywamy autonomicznym. Innymi słowy, jest to autonomiczny układ 2 równań różniczkowych zwyczajnych 1-rzędu. Mówimy wówczas, że jego przestrzeń fazowa jest 2-wymiarowa.

Jeżeli cząstka porusza się na płaszczyźnie $(X, Y)$, to równanie Newtona ma postać:
\phantomsection\label{ch1/chI011:equation-eqn9}\begin{gather}
\begin{split} m  \frac{d^2x}{dt^2} = F\left(x, y, \frac{dx}{dt}, \frac{dy}{dt}, t\right) \qquad\end{split}\label{ch1/chI011-eqn9}
\end{gather}\phantomsection\label{ch1/chI011:equation-eqn10}\begin{gather}
\begin{split} m  \frac{d^2y}{dt^2} = G\left(x, y, \frac{dx}{dt}, \frac{dy}{dt}, t\right) \qquad\end{split}\label{ch1/chI011-eqn10}
\end{gather}
gdzie $F$  i  $G$  są składowymi siły działającymi  na cząstkę w kierunku $x$ oraz $y$. W ogólnym przypadku siły te zależą od położenia cząstki $(x, y)$, jej składowych prędkości $(dx/dt, dy/dt)$ oraz czasu $t$.

Jeżeli składowe siły $F$  i $G$  nie zależą w sposób jawny od czasu, to postępując podobnie jak poprzednio otrzymamy układ:
\phantomsection\label{ch1/chI011:equation-eqn11}\begin{gather}
\begin{split}\frac{dx}{dt} = v \qquad\end{split}\label{ch1/chI011-eqn11}
\end{gather}\phantomsection\label{ch1/chI011:equation-eqn12}\begin{gather}
\begin{split}\frac{dy}{dt} = u \qquad\end{split}\label{ch1/chI011-eqn12}
\end{gather}\phantomsection\label{ch1/chI011:equation-eqn13}\begin{gather}
\begin{split} m  \frac{dv}{dt} = F(x, y, v, u) \qquad\end{split}\label{ch1/chI011-eqn13}
\end{gather}\phantomsection\label{ch1/chI011:equation-eqn14}\begin{gather}
\begin{split} m  \frac{du}{dt} = G(x, y, v, u ) \qquad\end{split}\label{ch1/chI011-eqn14}
\end{gather}
Jest to autonomiczny układ 4 równań różniczkowych zwyczajnych 1-rzędu. Mówimy wówczas, że jego przestrzeń fazowa jest 4-wymiarowa.

Dla cząstki poruszającej się w przestrzeni $(X, Y, Z)$, mamy 3 równania Newtona 2-rzędu. Jeżeli  3 składowe siły   nie zależą w sposób jawny od czasu, to postępując podobnie jak poprzednio otrzymamy  układ 6 równań różniczkowych 1-rzędu i przestrzeń fazowa jest 6-wymiarowa.W ogólności, dla $N$ cząstek poruszających się w przestrzeni, przestrzeń fazowa ma wymiar $6N$. Analiza takich równań przekracza możliwości współczesnej matematyki w tym sensie, że mało wiemy o ogólnych własnościach konkretnych układów, które modelujemy.  To powoduje, że musimy stosować numeryczne metody i komputer jest nieodzownym narzędziem analizy.

Powyżej podaliśmy jeden przykład modelowania. Bazuje on na formaliźmie Newtona i równaniach  ruchu Newtona, Może być stosowany do opisu dynamiki cząstek klasycznych. Czasami wygodnie jest stosować inny formalizm jak na przykład formalizm Lagrange'a lub formalizm Hamiltona. W wielu przypadkach wszystkie trzy formalizmy są równoważne. Dla tzw. układów z więzami, wygodnie jest stosować formalizm Lagrange'a lub formalizm Hamiltona.

Definiując układ równań różniczkowych jako autonomiczny, zakładaliśmy że siła nie zależy w sposób jawny od czasu. Może wydawać się, że jest to jakieś ograniczenie. Nie jest to prawdą. Układy nieautonomiczne można sprowadzić do układów autonomicznych wprowadzając dodatkową zmienną niezależną, dodatkowe ``położenie''. Pokażemy to na prostym przykładzie. Rozpatrzmy cząstkę poruszającą się wzdłuż osi X. Na cząstkę działa siła tarcie proporcjonalna do prędkości cząstki, $F = -\gamma v$, działa siła potencjalna $F(x) = -V'(x)$ pochodząca od energii potencjalnej $V(x)$ (nazywanej skrótowo potencjałem). Siła ta jest ujemnym gradientem potencjału (czyli pochodną $V'(x)$). Dodatkowo na cząstkę działa periodyczna w czasie siła $F(t)  = A cos(\omega t)$. Równanie Newtona ma postać
\phantomsection\label{ch1/chI011:equation-eqn15}\begin{gather}
\begin{split}m\ddot x = -\gamma \dot x - V'(x) + A cos(\omega t) \qquad\end{split}\label{ch1/chI011-eqn15}
\end{gather}
gdzie kropki oznaczają pochodne względem czasu, a apostrof oznacza pochodną względem $x$. I tak
\phantomsection\label{ch1/chI011:equation-eqn16}\begin{gather}
\begin{split}\dot x = \frac{dx}{dt}, \qquad \ddot x = \frac{d^2x}{dt^2}, \qquad V'(x) = \frac{dV(x)}{dx}\end{split}\label{ch1/chI011-eqn16}
\end{gather}
Równanie to możemy przedstawić w postaci układu 3 równań różniczkowych:
\phantomsection\label{ch1/chI011:equation-eqn17}\begin{gather}
\begin{split}\dot x = v \qquad\end{split}\label{ch1/chI011-eqn17}
\end{gather}\phantomsection\label{ch1/chI011:equation-eqn18}\begin{gather}
\begin{split}m \dot v = -\gamma v -V'(x) + A cos (z)\qquad\end{split}\label{ch1/chI011-eqn18}
\end{gather}\phantomsection\label{ch1/chI011:equation-eqn19}\begin{gather}
\begin{split}\dot z = \omega\qquad\end{split}\label{ch1/chI011-eqn19}
\end{gather}
Równoważność  pokazujemy  w następujący sposób:

w równaniu  \eqref{ch1/chI011-eqn18}  należy zastąpić $v$   z równania \eqref{ch1/chI011-eqn17} wyrażeniem   $v=\dot x$ pamiętając jednocześnie że $\dot v = \ddot x$
równanie \eqref{ch1/chI011-eqn19} można scałkować i otrzymamy $z=\omega t$; wstawiamy to wyrażenie do równania \eqref{ch1/chI011-eqn18}.
W ten sposób otrzymujemy znowu równanie \eqref{ch1/chI011-eqn15}. Tak więc jedno równanie różniczkowe nieutonomiczne 2-rzędu jest równoważne układowi 3 równań różniczkowych 1-rzędu. Odpowiadająca temu układowi przestrzeń fazowa jest 3-wymiarowa.  Z przykładu tego płynie ważna wskazówka, jak otrzymywać autonomiczny układ równań różniczkowych 1-rzędu. Liczba tych równań definiuje przestrzeń fazową układu. Wymiar tej przestrzeni jest jedną z najważniejszych charakterystyk. Proszę to zapamiętać!

Fizyka stosuje też aparat równań różniczkowych cząstkowych. Studenci kierunku fizyka i pokrewnych kierunków znają przykłady takich równań. Równanie Schrodingera, równanie falowe, równanie dyfuzji, równania Maxwela to są równania różniczkowe cząstkowe. Ich analiza jest znacznie trudniejsza. Istnieją specjalne i specyficzne  metody matematyczne pozwalające otrzymać informację o własnościach układów opisywanych takimi równaniami.

W wielu dziedzinach nauki (chemia, biologia, socjologia, nauki ekonomiczne) stosuje się fenomenologiczny sposób modelowania.  Aby uzmysłowić, jak go stosować podamy jeden przykład.


\subsection{Modelowanie procesu wzrostu}
\label{ch1/chI011:modelowanie-procesu-wzrostu}
Procesy wzrostu pojawiają się na wielu obszarach. Nie trzeba być bystrym obserwatorem, aby zauważyć co wokół nas może wzrastać. My rozważamy jedną  z możliwych klas procesów wzrosu: wzrost populacji zajęcy czy  bakterii, wzrost depozytów pieniężnych na lokatach bankowych, wzrost stężenia substancji w reakcjach chemicznych  czy  wzrost liczby komórek nowotworowych. Często procesom wzrostu towarzyszą procesy malenia (zaniku, śmierci, ...). My je będziemy pomijać. Rozpatrzmy konkretny przykład: wzrost pieniędzy na lokacie bankowej. Załóżmy, żę w chwili czasu $t$ jest na lokacie $x(t)$ (np. złotych polskich). Pytamy, ile pieniędzy przyrośnie  po pewnym czasie $h$, czyli ile pieniędzy będzie w chwili $t+h$. Zaczynamy modelować ten proces. Oznaczmy, że w chwili $t+h$ jest $x(t+h)$ pieniędzy. Na tę kwotę składają się pieniądze $x(t)$ oraz przyrost $\delta$ z odsetek, czyli
\phantomsection\label{ch1/chI011:equation-eqn20}\begin{gather}
\begin{split}x(t+h)  =  x(t) + \delta\end{split}\label{ch1/chI011-eqn20}
\end{gather}
Przyrost $\delta ` zależy od :math:`x(t)$,  od wielkości oprocentowania $k$ oraz od tego jak długo $(h)$ trzymamy pieniądze na lokacie, czyli
\phantomsection\label{ch1/chI011:equation-eqn21}\begin{gather}
\begin{split} \delta \propto  x(t), \qquad \delta \propto  k, \qquad \delta \propto  h\end{split}\label{ch1/chI011-eqn21}
\end{gather}
Możemy to skomasować pisząc:
\phantomsection\label{ch1/chI011:equation-eqn22}\begin{gather}
\begin{split}\delta = k  x(t)  h\end{split}\label{ch1/chI011-eqn22}
\end{gather}
Dlatego też
\phantomsection\label{ch1/chI011:equation-eqn23}\begin{gather}
\begin{split}x(t+h) = x(t) +  k  x(t)  h\end{split}\label{ch1/chI011-eqn23}
\end{gather}
Relacje tę możemy przepisać w postaci
\phantomsection\label{ch1/chI011:equation-eqn24}\begin{gather}
\begin{split}\frac{x(t+h) - x(t)}{h} = k  x(t)\end{split}\label{ch1/chI011-eqn24}
\end{gather}
W granicy małych przyrostów czasu $h \to 0$, lewa strona jest definicją pochodnej
\phantomsection\label{ch1/chI011:equation-eqn25}\begin{gather}
\begin{split}\frac{dx(t)}{dt} =  k x(t), \quad \quad x(0) = x_0\end{split}\label{ch1/chI011-eqn25}
\end{gather}
gdzie $x_0$ jest wartością początkową naszej lokaty. W ten oto sposób otrzymaliśmy równanie opisujące dynamikę wzrostu pieniędzy na naszej lokacie bankowej. Jest to równanie różniczkowe zwyczajne, 1-go rzędu, autonomiczne. Jego przestrzeń fazowa jest 1-wymiarowa.

Poniżej pokazujemy  rozwiązania tego równania dla 3 różnych wartości $k$.


\begin{verbatim}
var('N1,N2,N3')
T = srange(0,3,0.01)
## rozwiązania dla różnych wartości k=0, 0.1, 0.2
sol=desolve_odeint( vector([0, 0.1*N2,  0.2*N3]), [5,5,5],T,[N1,N2,N3])
## wykresy rozwiązań dla różnych wartości k=-1, 0, 0.5
line(zip(T,sol[:,0]), figsize=(5, 3), legend_label="k=0") +\
 line(zip(T,sol[:,1]), color='red', legend_label="k=0.1")+\
 line(zip(T,sol[:,2]), color='green', legend_label="k=0.2")
\end{verbatim}


Inne procesy wzrostu także można modelować tym równaniem. Równanie to jest też punktem wyjściowym do modyfikacji, uogólnień, rozszerzeń, itp.  Proste rozszerzenie polega na uzależnieniu współczynnika tempa wzrostu k od dodatkowych czynników. Na przykład przy modelowaniu wzrostu populacji zajęcy, możemy uzależnić tempo wzrostu k od liczby zajęcy  w populacji: duża ilość zajęcy powoduje dużą konsumpcję pożywienia, a to z kolei skutkuje zmaleniem ilości pożywienia i utrudnieniami w zdobywaniu pożywienia. W efekcie zmniejsza się tempo wzrostu k. Innymi słowy, $k$ powinno być malejącą funkcją $x(t)$ liczebników w populacji.  Istnieje nieskończenie wiele takich funkcji.  Na przykład
\phantomsection\label{ch1/chI011:equation-eqn26}\begin{gather}
\begin{split}k  \to  k(x) = exp(-b x), \quad \quad b>0\end{split}\label{ch1/chI011-eqn26}
\end{gather}
jest malejącą funkcją $x$. Teraz równanie różniczkowe ma postać
\phantomsection\label{ch1/chI011:equation-eqn27}\begin{gather}
\begin{split}\frac{dx}{dt} = x e^{-bx}, \quad x = x(t), \quad x(0) = x_0\end{split}\label{ch1/chI011-eqn27}
\end{gather}
Jakie są skutki takiej zmiany? Pokazujemy to na poniższym rysunku. Zauważamy, że tempo wzrostu populacji zmniejsza się w porównaniu z poprzednim przypadkiem.

Model można rozszerzyć uwzględniając procesy śmierci: te naturalne i te wskutek istnienia drapieżników, które zjadają  osobników populacji. Prosty model  ofiara-drapieżca  jest 2-wymiarowy: opisuje zmiany w populacji ofiar i zmiany w populacji drapieżników. Jest to autonomiczny układ 2 równań różniczkowych zwyczajnych.


\begin{verbatim}
var('x,y,z')
U = srange(0,300,0.01)
sol=desolve_odeint( vector([x*exp(-0.1*x),  y*exp(-0.2*y),  z*exp(-0.3*z)]), [5,5,5],U,[x,y,z])
## pokazujemy rozwiązania dla różnych wartości k=-1, 0, 0.5
line(zip(U,sol[:,0]), figsize=(5, 3), legend_label="k=0")+\
 line(zip(U,sol[:,1]), color='red', legend_label="k=0.1")+\
 line(zip(U,sol[:,2]), color='green', legend_label="k=0.2")
\end{verbatim}



\section{Modelowanie z czasem dyskretnym}
\label{ch1/chI011:modelowanie-z-czasem-dyskretnym}
Powyżej otrzymaliśmy takie oto wyrażenie na przyrost:
\phantomsection\label{ch1/chI011:equation-eqn28}\begin{gather}
\begin{split}x(t+h) = x(t) + k h x(t)\end{split}\label{ch1/chI011-eqn28}
\end{gather}
Jeżeli zmiany następowałyby nie w sposób ciągły  lecz dyskretny (np.  co 1 dzien, co jedną godzinę) wówczas  krok czasowy $h$ jest dyskretny. Można wprowadzić oznaczenia
\phantomsection\label{ch1/chI011:equation-eqn29}\begin{gather}
\begin{split}x_n = x(t), \quad \quad x_{n+1} = x(t+h)\end{split}\label{ch1/chI011-eqn29}
\end{gather}
i wówczas równanie dla przyrostu ma postać
\phantomsection\label{ch1/chI011:equation-eqn30}\begin{gather}
\begin{split}x_{n+1} = x_n + \alpha x_n, \quad \quad \alpha = k h\end{split}\label{ch1/chI011-eqn30}
\end{gather}
W ten sposób otrzymujemy równanie z czasem dyskretnym. Ogólna postać tego typu równania to
\phantomsection\label{ch1/chI011:equation-eqn31}\begin{gather}
\begin{split}x_{n+1} = f(x_n)\end{split}\label{ch1/chI011-eqn31}
\end{gather}
które mówi nam, jaką wartość przyjmuje dana wielkość w następnym kroku $n+1$ jeżeli znana jest wartość tej wielkości w kroku $n$. Równanie to nazywa się też równaniem rekurencyjnym. W zależności od postaci funkcji $f(x)$ otrzymujemy różne modele dynamiki układów.

Układ 2 równań z czasem dyskretnym ma postać
\phantomsection\label{ch1/chI011:equation-eqn32}\begin{gather}
\begin{split}x_{n+1} = f(x_n, y_n)\end{split}\label{ch1/chI011-eqn32}
\end{gather}\phantomsection\label{ch1/chI011:equation-eqn33}\begin{gather}
\begin{split}y_{n+1} = g(x_n, y_n)\end{split}\label{ch1/chI011-eqn33}
\end{gather}
Analiza jakościowa takiego układu jest bardzo trudna. Czasami  nieumiejętne stosowanie numerycznej analizy może skutkować tym, że umkną nam istotne cechy takiego układu, zwłaszcza gdy w układzie  występują dodatkowo  parametry których zmiana może powodować coś, co nazywa się bifurkacjami.  Ale o tym w dalszej części książki.


\section{Istnienie i jednoznaczność rozwiązań}
\label{ch1/chI012:istnienie-i-jednoznacznosc-rozwiazan}\label{ch1/chI012::doc}
Do opisu  realnych zjawisk przy pomocy równań różniczkowych zwyczajnych z warunkami początkowymi zadanymi w chwili  czasu $t=0$, potrzebne nam są rozwiązania dla czasów $t>0$ (ewolucja czasowa).  Można też rozpatrywać przypadek $t<0$ ale to zaliczyłbym do ćwiczeń matematycznych.  Ważnym zagadnieniem jest istnienie rozwiązań równań różniczkowych. Możemy zapytać, czy zawsze rozwiązania równań różniczkowych istnieją i jeżeli istnieją, to czy to są jedyne rozwiązania z warunkiem początkowym. Oczywiście dla różnych warunków początkowych układ może różnie ewoluować, ale gdy startuje  zawsze z tego samego  stanu (warunku) początkowego to czy ewolucja jest taka sama? Na tym polega problem jednoznaczności rozwiązań. Jeżeli dla danego warunku początkowego istnieją  np. 3 rozwiązania, to jak ewoluuje układ: istnieją 3 możliwości i którą możliwość wybiera układ? Gdyby tak było dla realnych układów to nie moglibyśmy przewidywać ewolucji układu, nie moglibyśmy sterować układami, brak byłoby determinizmu.  W rozwoju nauk ścisłych to właśnie determinizm stał się kołem napędowym rozwoju cywilzacyjnego ludzkości. To determinizm pozwala budować urządzenia, które działają tak jak my sobie tego życzymy: telewizor odbiera wybrany przeze mnie program, używam telefonu do komunikacji  z moją rodziną, wystrzelona rakieta ma taką trajektorię jaką zaplanowałem, itd.  Zbadamy 3 przykłady, które przybliżą nam powyższą problematykę. Źródło tych przykładów jest w  książce: J. Hale, H. Kocak, ``Dynamics and Bifurcations''. Ksiązka jest znakomita.


\subsection{Przykład 1}
\label{ch1/chI012:przyklad-1}
Równanie
\phantomsection\label{ch1/chI012:equation-eqn1}\begin{gather}
\begin{split}\frac{dx}{dt}=-2x, \qquad x(0) = x_0\end{split}\label{ch1/chI012-eqn1}
\end{gather}
jest równaniem różniczkowym liniowym. Jest to jedno z najprostszych równań różniczkowych.  Można łatwo sprawdzić, że funkcja
\phantomsection\label{ch1/chI012:equation-eqn2}\begin{gather}
\begin{split}x(t) = x_0   e^{-2t}\end{split}\label{ch1/chI012-eqn2}
\end{gather}
jest rozwiązaniem i spełnia warunek początkowy $x(0) = x_0$. Funkcja ta jest dobrze określona dla wszystkich skończonych  wartości czasu $t \in (-\infty, \infty)$.  Nie ma tu większych ograniczeń.  Jest to jedyne rozwiązanie.  Poniższy rysunek daje wyobrażenie o rozwiązaniach $x(t)$ dla 3 różnych warunków początkowych. Przy okazji zauważmy, że wszystkie trzy rozwiązania dążą do tego samego stanu $x=0$  dla długich czasów $t\to \infty$.


\begin{verbatim}
var('t')
g(t,a) = a*exp(-2*t)
p1 = plot(g(t,a=1), (t,0,2), legend_label=r"$x(0)=1$", color='blue')
p2 = plot(g(t,a=0), (t,0,2), legend_label=r"$x(0)=0$", color='red')
p3 = plot(g(t,a=-1), (t,0,2), legend_label=r"$x(0)=-1$", color='green')
show(p1+p2+p3, figsize=[6,3], axes_labels=[r'$t$',r'$x(t)$'], axes=False, frame=True)
\end{verbatim}



\subsection{Przykład 2}
\label{ch1/chI012:przyklad-2}
Równanie
\phantomsection\label{ch1/chI012:equation-eqn3}\begin{gather}
\begin{split}\frac{dx}{dt}= 3 x^2, \qquad x(0) = x_0\end{split}\label{ch1/chI012-eqn3}
\end{gather}
jest równaniem różniczkowym nieliniowym.   Prawa strona tego równania jest określona dla wszystkich wartości $x$. Podobnie jak poprzednie równanie, można  je rozwiązać metodą separacji zmiennych. Otrzymamy funkcję
\phantomsection\label{ch1/chI012:equation-eqn4}\begin{gather}
\begin{split}x(t) = \frac{x_0}{1-3 x_0 t}\end{split}\label{ch1/chI012-eqn4}
\end{gather}
która jest rozwiązaniem i spełnia warunek początkowy. Funkcja ta nie jest określona dla wszystkich skończonych  wartości czasu $t \in (-\infty, \infty)$.  Istnieją  ograniczenia dla wartości czasu $t$. Ale jest to jedyne rozwiązanie.


\begin{verbatim}
var('t')
g  = plot(-4.0/(1 +12*t), (t,0,0.5), detect_poles='show', legend_label=r'$x(0)=-4$', color='blue')
g += plot(lambda t: 0.0, (t,0,0.5), legend_label=r'$x(0)=0$', color='red')
g += plot(1.0/(1-3*t), (t,0,1/3), detect_poles='show', legend_label=r'$x(0)=1$', color='green')
g.show(axes_labels=[r'$t$',r'$x$'], ymin=-4, ymax=8, figsize=[6,3], axes=False, frame=True)
\end{verbatim}


Wszystkie rozwiązania z ujemnym warunkiem początkowym $x(0) < 0$ sa dobrze zdefiniowane dla wszystkich czasów $t>0$ (krzywa niebieska). Podobnie jest z rozwiązaniem $x(t) = 0$ dla warunku początkowego $x(0)=0$ (krzywa czerwona). Natomiast rozwiązanie z  dodatnim warunkiem początkowym $x(0) > 0$ rozbiega się w skończonym czasie $t< 1/3x_0$ . Gdyby to równanie miało opisywać ruch cząstki, to oznacza że w skończonym czasie cząstka przebywa nieskończoną odległość. To jest niefizyczne. Równanie  to mogłoby   opisywać proces wybuchu  substancji: $x$ mogłoby być objętością pęczniejącej substancji która  wybucha po skończonym czasie.


\subsection{Przykład 3}
\label{ch1/chI012:przyklad-3}
Równanie
\phantomsection\label{ch1/chI012:equation-eqn5}\begin{gather}
\begin{split}\frac{dx}{dt}=  2 \sqrt x, \qquad x(0) = x_0 \ge 0\end{split}\label{ch1/chI012-eqn5}
\end{gather}
jest równaniem różniczkowym nieliniowym.  Prawa strona tego równania jest określona dla nieujemnych wartości $x \ge 0$.  Podobnie jak  2 poprzednie równania, można  je rozwiązać metodą separacji zmiennych. Otrzymamy rozwiązanie
\phantomsection\label{ch1/chI012:equation-eqn6}\begin{gather}
\begin{split}x(t) = (t +  \sqrt x_0)^2\end{split}\label{ch1/chI012-eqn6}
\end{gather}
Funkcja ta jest określona dla wszystkich wartości czasu $t >0$.   Jest to jedyne  rozwiązanie  z wyjątkiem jednego warunku początkowego: $x(0) = 0$. Dla tego warunku początkowego istnieje jeszcze jedno rozwiązanie, a mianowicie $x(t) = 0$. Tak więc dla $x(0) = 0$ mamy  2 różne rozwiązania
\phantomsection\label{ch1/chI012:equation-eqn7}\begin{gather}
\begin{split}x(t) = t^2, \qquad x(t) = 0\end{split}\label{ch1/chI012-eqn7}
\end{gather}
Jak przebiega ewolucja, gdy układ startuje ze stanu początkowego $x(0) = 0$ ? W tym przypadku rozwiązania są niejednoznaczne.


\begin{verbatim}
var('t')
p1=plot(t**2,(t,0,1), legend_label=r"$x(0)=1$", color='blue')
p2=plot(0,(t,0,1), legend_label=r"$x(0)=0$", color='red')
show(p1+p2, figsize=[6,3], axes=False, frame=True)
\end{verbatim}


Co jest takiego charakterystycznego w ostatnim przykładzie, że pojawia się niejednoznaczność rozwiązania równania różniczkowego?  Na to pytanie daje odpowiedź  twierdzenie o jednoznaczności rozwiązania równania różniczkowego. Potrzebna nam będzie własność funkcji:

Mówimy, że funkcja $f(x)$ spełnia  warunek Lipschitza na zbiorze otwartym $U$ jeżeli istnieje taka stała $L > 0$,  że
\phantomsection\label{ch1/chI012:equation-eqn8}\begin{gather}
\begin{split}|f(x_2) -f(x_1)| \le L|x_2 - x_1|\end{split}\label{ch1/chI012-eqn8}
\end{gather}
dla wszystkich $x_1, x_2 \in U$.

Warunek Lipschitza można zapisać w postaci
\phantomsection\label{ch1/chI012:equation-eqn9}\begin{gather}
\begin{split}|f(x+h) -f(x)| \le L h \quad \quad \mbox{lub jako} \quad \quad \frac{f(x+h) - f(x)}{h}| \le L\end{split}\label{ch1/chI012-eqn9}
\end{gather}
Z tego wynika że jeżeli  $f(x)$ ma ograniczoną pochodną, to spełnia warunek Lipschitza. Są  oczywiście nieróżniczkowalne funkcje, które spełniają warunek Lipschitza.
\begin{description}
\item[{Twierdzenie Picarda}] \leavevmode
Jeżeli funkcja $f(x)$ jest ciągła w $U$ oraz spełnia warunek Lipschtza w  $U$ wówczas równanie różniczkowe

\end{description}
\phantomsection\label{ch1/chI012:equation-eqn10}\begin{gather}
\begin{split}\frac{dx}{dt} = f(x), \qquad x(0) = x_0\end{split}\label{ch1/chI012-eqn10}
\end{gather}
ma dokładnie jedno rozwiązanie w $U$.

Istnieje kilka  modyfikacji tego twierdzenia, ale na nasze potrzeby ta najprostsza wersja jest wystarczająca.

Teraz możemy odpowiedzieć, dlaczego w 3 przykładzie rozwiązanie jest niejednoznaczne: funkcja $f(x) = 2\sqrt x$ nie spełnia warunku Lipschitza ponieważ pochodna
\phantomsection\label{ch1/chI012:equation-eqn11}\begin{gather}
\begin{split}\frac{df(x)}{dx} = \frac{1}{\sqrt x}\end{split}\label{ch1/chI012-eqn11}
\end{gather}
w punkcie $x=0$ jest rozbieżna. W punktach $x>0$  pochodna ma wartość skończoną i jest spełnione twierdzenie Picarda. Dlatego też  rozwiązania są jednoznaczne.


\subsection{Dodatek}
\label{ch1/chI012:dodatek}
Sage z powodzeniem jest w stanie rozwiązywać pewne równania różniczkowe zwyczajne. Zobaczmy jak poradzi sobie z powyższymi przykładami.


\subsubsection{Przykład 1}
\label{ch1/chI012:id1}\phantomsection\label{ch1/chI012:equation-eqn12}\begin{gather}
\begin{split}\frac{dx}{dt}=-2x, \qquad x(0) = x_0\end{split}\label{ch1/chI012-eqn12}
\end{gather}
z rozwiązaniem
\phantomsection\label{ch1/chI012:equation-eqn13}\begin{gather}
\begin{split}x(t) = x_0   e^{-2t}.\end{split}\label{ch1/chI012-eqn13}
\end{gather}
Na początek zadamy sobie zmienne. Druga linijka mówi o tym, że zmienna $x$ będzie funkcją parametru $t$ (czasu). Zamiast
używac nazwy \code{g} użyjemy świerzo obliczonego rozwiązania \code{rozw}.


\begin{verbatim}
var('t x_0')
x = function('x', t)
rrz = diff(x,t) == -2*x
rozw = desolve(rrz, x)
rozw = rozw.subs(c=x_0)
print "rozwiązanie równania"
show(rozw)
p1 = plot(rozw(x_0=1), (t,0,2), legend_label=r"$x(0)=1$", color='blue')
p2 = plot(rozw(x_0=0), (t,0,2), legend_label=r"$x(0)=0$", color='red')
p3 = plot(rozw(x_0=-1), (t,0,2), legend_label=r"$x(0)=-1$", color='green')
show(p1+p2+p3, figsize=[6,3], axes_labels=[r'$t$',r'$x(t)$'], axes=False, frame=True)
\end{verbatim}



\subsubsection{Przykład 2}
\label{ch1/chI012:id2}\phantomsection\label{ch1/chI012:equation-eqn14}\begin{gather}
\begin{split}\frac{dx}{dt}= 3 x^2, \qquad x(0) = x_0\end{split}\label{ch1/chI012-eqn14}
\end{gather}
z rozwiązaniem
\phantomsection\label{ch1/chI012:equation-eqn15}\begin{gather}
\begin{split}x(t) = \frac{x_0}{1-3 x_0 t}.\end{split}\label{ch1/chI012-eqn15}
\end{gather}

\begin{verbatim}
var('t x_0 c')
x = function('x', t)
print "Definiujemy równanie różniczkowe"
rrz = diff(x,t) == 3*x^2
rozw2 = desolve(rrz, x)
print "i je rozwiązujemy..."
show(rozw2)
print "krok 1\n obliczamy x(t) z poprzedniego kroku"
rozw2 = solve(rozw2,x)[0].rhs()
show(rozw2)
print "krok 2\n obliczamy x(0)"
buf = rozw2(t=0) == x_0
show(buf)
print "krok 3\n wyznaczamy stałą c"
buf = solve(buf,c)[0].rhs()
show(buf)
print "krok 4\n wstawiamy c do równania"
rozw2 = rozw2.subs(c=buf).full_simplify()
show(rozw2)
print "I na koniec prezentujemy wyniki"
x0 = -4
w = plot(rozw2(x_0=x0), (t,0,1), detect_poles='show', legend_label=r'$x(0)=%d$'%x0, color='blue')
x0 = 0
w += plot(rozw2(x_0=x0), (t,0,1), legend_label=r'$x(0)=%d$'%x0, color='red')
x0 = 1
w += plot(rozw2(x_0=x0), (t,0,1/3), legend_label=r'$x(0)=%d$'%x0, color='green')
w.show(axes_labels=[r'$t$',r'$x$'], tick_formatter='latex', xmin=0, xmax=0.5, ymin=-4.1, ymax=8, figsize=(6,3), axes=False, frame=True)
\end{verbatim}



\subsubsection{Przykład 2}
\label{ch1/chI012:id3}\phantomsection\label{ch1/chI012:equation-eqn16}\begin{gather}
\begin{split}\frac{dx}{dt}=  2 \sqrt x, \qquad x(0) = x_0 \ge 0\end{split}\label{ch1/chI012-eqn16}
\end{gather}
z rozwiązaniem
\phantomsection\label{ch1/chI012:equation-eqn17}\begin{gather}
\begin{split}x(t) = (t +  \sqrt x_0)^2\end{split}\label{ch1/chI012-eqn17}
\end{gather}

\begin{verbatim}
var('t x_0 c')
forget()
assume(x_0>=0)
assume(t+c>0)
print "równanie"
x = function('x', t)
rrz = diff(x,t) == 2*sqrt(x)
show(rrz)
print "i jego rozwiązanie"
rozw3 = solve(desolve(rrz, x),x)[0]
show(rozw3)
print "stała całkowania"
buf = solve(x_0 == rozw3.rhs()(t=0),c)
show(buf)
print "mamy dwa możliwe rozwiązania, wybieramy to z dodatnim c"
buf = buf[1]
show(buf)
print "i dostajemy ostatecznie"
rozw3 = rozw3.subs(c=buf.rhs())
show(rozw3)
print "I na koniec prezentujemy wyniki"
p1=plot(rozw3.rhs()(x_0=0),(t,0,1), legend_label=r"$x(0)=1$", color='blue')
show(p1, figsize=[6,3], axes=False, frame=True)
\end{verbatim}


No tak, ale gdzie jest rozwiązanie $x(t) = 0$? Na chwilę obecną Sage nie rozróżni obu możliwych rozwiązań. Dlatego umiejętność analitycznego rozwiązania takich problemów wciąż jest niezbędna!


\section{Układy dynamiczne z czasem ciągłym}
\label{ch1/chI021::doc}\label{ch1/chI021:uklady-dynamiczne-z-czasem-ciaglym}
We Wstępie podaliśmy kilka przykładów układów opisanych za pomocą równań różniczkowych zwyczajnych. Pierwsza klasa układów to  układy opisywane przez  mechanikę klasyczną i jej  równania Newtona. Inna klasa układów to równania fenomenologiczne opisujące procesy wzrostu, procesy kinetyki chemicznej, dynamiki populacyjnej w układach biologicznych, itd. Te dwie klasy układów opisywane są układem równań różniczkowych zwyczajnych zapisanych w ogolnej postaci jako układ
\phantomsection\label{ch1/chI021:equation-eqn1}\begin{gather}
\begin{split}\frac{dx_1}{dt} = F_1(x_1, x_2, x_3, ..., x_n)\end{split}\label{ch1/chI021-eqn1}\\\begin{split}\frac{dx_2}{dt} = F_2(x_1, x_2, x_3, ..., x_n)\end{split}\notag\\\begin{split}\vdots\end{split}\notag\\\begin{split}\frac{dx_n}{dt} = F_n(x_1, x_2, x_3, ..., x_n)\end{split}\notag
\end{gather}
Ten układ możemy zapisać w notacji wektorowej w  postaci
\phantomsection\label{ch1/chI021:equation-eqn5}\begin{gather}
\begin{split}\frac{d\vec x}{dt} = \vec F(\vec x), \quad \quad \vec x(0)  = \vec x_0  \qquad \quad\end{split}\label{ch1/chI021-eqn5}
\end{gather}
gdzie wektory
\phantomsection\label{ch1/chI021:equation-eqn6}\begin{gather}
\begin{split}\vec x = [x_1, x_2, x_3, ...., x_n], \quad \quad \vec F = [F_1, F_2, F_3, ..., F_n]\end{split}\label{ch1/chI021-eqn6}
\end{gather}
oraz dany jest zbiór warunków początkowych
\phantomsection\label{ch1/chI021:equation-eqn7}\begin{gather}
\begin{split}\vec x(0) = [x_1(0), x_2(0), x_3(0), ... , x_n(0)] = \vec x_0 = [x_1^{(0)}, x_2^{(0)}, x_3^{(0)}, ... ,  x_n^{(0)}]\end{split}\label{ch1/chI021-eqn7}
\end{gather}
Wskaźnik $n$ mówi, ile równań różniczkowych jest ``ukrytych'' w powyższym zapisie wektorowym. Innymi słowy, rozważamy układ $n$ równań różniczkowych scharaktaryzowanych przez $n$ funkcji skalarnych $F_i(x_1, x_2, x_3, ..., x_n), (i=1,2,3, ..., n)$. Zauważmy, że rozważamy funkcje $F_i$ które nie zależą w sposób jawny od czasu. W takim przypadku mówimy, że jest to układ autonomiczny $n$ równań  różniczkowych zwyczajnych. Ponadto wektor $\vec F$ można traktowac jako pole wektorowe stowarzyszone z układem równań różniczkowych lub pole wektorowe generowane przez układ równań różniczkowych. Do tej kwestii powrócimy jeszcze.


\subsection{Istnienie i jednoznaczność rozwiązań}
\label{ch1/chI021:istnienie-i-jednoznacznosc-rozwiazan}
Do opisu  realnych zjawisk przy pomocy równań różniczkowych zwyczajnych z warunkami początkowymi zadanymi w chwili  czasu $t=0$, potrzebne nam są rozwiązania dla czasów $t>0$ (ewolucja czasowa).  Można też rozpatrywać przypadek $t<0$ ale to zaliczyłbym do ćwiczeń matematycznych.  Ważnym zagadnieniem jest istnienie rozwiązań równań różniczkowych. Możemy zapytać, czy zawsze rozwiązania równań różniczkowych istnieją i jeżeli istnieją, to czy to są jedyne rozwiązania z warunkiem początkowym. Oczywiście dla różnych warunków początkowych układ może różnie ewoluować, ale gdy startuje  zawsze z tego samego  stanu (warunku) początkowego to czy ewolucja jest taka sama? Na tym polega problem jednoznaczności rozwiązań. Jeżeli dla danego warunku początkowego istnieją  np. 3 rozwiązania, to jak ewoluuje układ: istnieją 3 możliwości i którą możliwość wybiera układ? Gdyby tak było dla realnych układów to nie moglibyśmy przewidywać ewolucji układu, nie moglibyśmy sterować układami, brak byłoby determinizmu.  W rozwoju nauk ścisłych to właśnie determinizm stał się kołem napędowym rozwoju cywilzacyjnego ludzkości. To determinizm pozwala budować urządzenia, które działają tak jak my sobie tego życzymy: telewizor odbiera wybrany przeze mnie program, używam telefonu do komunikacji  z moją rodziną, wystrzelona rakieta ma taką trajektorię jaką zaplanowałem, itd.  Zbadamy 3 przykłady, które przybliżą nam powyższą problematykę. Źródło tych przykładów jest w  książce: J. Hale, H. Kocak, ``Dynamics and Bifurcations''. Książka jest znakomita.


\subsubsection{Przykład 1}
\label{ch1/chI021:przyklad-1}
Równanie
\phantomsection\label{ch1/chI021:equation-eqn8}\begin{gather}
\begin{split}\frac{dx}{dt}=-2x, \qquad x(0) = x_0\end{split}\label{ch1/chI021-eqn8}
\end{gather}
jest równaniem różniczkowym liniowym. Jest to jedno z najprostszych równań różniczkowych.  Można łatwo sprawdzić, że funkcja
\phantomsection\label{ch1/chI021:equation-eqn9}\begin{gather}
\begin{split}x(t) = x_0   e^{-2t}\end{split}\label{ch1/chI021-eqn9}
\end{gather}
jest rozwiązaniem i spełnia warunek początkowy $x(0) = x_0$. Funkcja ta jest dobrze określona dla wszystkich skończonych  wartości czasu $t \in (-\infty, \infty)$.  Nie ma tu większych ograniczeń.  Jest to jedyne rozwiązanie.  Poniższy rysunek daje wyobrażenie o rozwiązaniach $x(t)$ dla 3 różnych warunków początkowych. Przy okazji zauważmy, że wszystkie trzy rozwiązania dążą do tego samego stanu $x=0$  dla długich czasów $t\to \infty$.


\begin{verbatim}
var('t')
g(t,a) = a*exp(-2*t)
p1=plot(g(t,1),(t,0,2),figsize=(6, 3), legend_label="x(0)=1", color='blue' )
p2=plot(g(t,0),(t,0,2),figsize=(6, 3), legend_label="x(0)=0", color='red' )
p3=plot(g(t,-1),(t,0,2),figsize=(6, 3), legend_label="x(0)=-1", color='green' )
show(p1+p2+p3, axes_labels=[r'$t$',r'$x(t)$'], frame=True, axes=False)
\end{verbatim}



\subsubsection{Przykład 2}
\label{ch1/chI021:przyklad-2}
Równanie
\phantomsection\label{ch1/chI021:equation-eqn10}\begin{gather}
\begin{split}\frac{dx}{dt}= 3 x^2, \qquad x(0) = x_0\end{split}\label{ch1/chI021-eqn10}
\end{gather}
jest równaniem różniczkowym nieliniowym.   Prawa strona tego równania jest określona dla wszystkich wartości $x$. Podobnie jak poprzednie równanie, można  je rozwiązać metodą separacji zmiennych. Otrzymamy funkcję
\phantomsection\label{ch1/chI021:equation-eqn11}\begin{gather}
\begin{split}x(t) = \frac{x_0}{1-3 x_0 t}\end{split}\label{ch1/chI021-eqn11}
\end{gather}
która jest rozwiązaniem i spełnia warunek początkowy. Funkcja ta nie jest określona dla wszystkich skończonych  wartości czasu $t \in (-\infty, \infty)$.  Istnieją  ograniczenia dla wartości czasu $t$. Ale jest to jedyne rozwiązanie.


\begin{verbatim}
var('t')
#detect_poles - wykrywanie i rysowanie biegunów
g=plot(-4.0/(1 +12*t),t,0,5,detect_poles='show',legend_label=r'$x(0)= -4$', color='blue')
g+=plot(lambda t: 0.0,t,0,5,legend_label=r'$x(0)=0$',color='red')
g+=plot(1.0/(1-3*t),t,0,0.33,detect_poles='show', legend_label=r'$x(0)=1$',color='green')
g.show(axes_labels=[r'$t$',r'$x$'],tick_formatter='latex',xmin=0,xmax=0.5,ymin=-4.1,ymax=8, figsize=(7,4), frame=True, axes=False)
\end{verbatim}


Wszystkie rozwiązania z ujemnym warunkiem początkowym $x(0) < 0$ sa dobrze zdefiniowane dla wszystkich czasów $t>0$ (krzywa niebieska). Podobnie jest z rozwiązaniem $x(t) = 0$ dla warunku początkowego $x(0)=0$ (krzywa czerwona). Natomiast rozwiązanie z  dodatnim warunkiem początkowym $x(0) > 0$ rozbiega się w skończonym czasie $t< 1/3x_0$ . Gdyby to równanie miało opisywać ruch cząstki, to oznacza że w skończonym czasie cząstka przebywa nieskończoną odległość. To jest niefizyczne. Równanie  to mogłoby   opisywać proces wybuchu  substancji: $x$ mogłoby być objętością pęczniejącej substancji która  wybucha po skończonym czasie.


\subsubsection{Przykład 3}
\label{ch1/chI021:przyklad-3}
Równanie
\phantomsection\label{ch1/chI021:equation-eqn12}\begin{gather}
\begin{split}\frac{dx}{dt}=  2 \sqrt x, \qquad x(0) = x_0 \ge 0\end{split}\label{ch1/chI021-eqn12}
\end{gather}
jest równaniem różniczkowym nieliniowym.  Prawa strona tego równania jest określona dla nieujemnych wartości $x \ge 0$.  Podobnie jak  2 poprzednie równania, można  je rozwiązać metodą separacji zmiennych. Otrzymamy rozwiązanie
\phantomsection\label{ch1/chI021:equation-eqn13}\begin{gather}
\begin{split}x(t) = (t +  \sqrt x_0)^2\end{split}\label{ch1/chI021-eqn13}
\end{gather}
Funkcja ta jest określona dla wszystkich wartości czasu $t >0$.   Jest to jedyne  rozwiązanie  z wyjątkiem jednego warunku początkowego: $x(0) = 0$. Dla tego warunku początkowego istnieje jeszcze jedno rozwiązanie, a mianowicie $x(t) = 0$. Tak więc dla $x(0) = 0$ mamy  2 różne rozwiązania
\phantomsection\label{ch1/chI021:equation-eqn14}\begin{gather}
\begin{split}x(t) = t^2, \qquad x(t) = 0\end{split}\label{ch1/chI021-eqn14}
\end{gather}
Jak przebiega ewolucja, gdy układ startuje ze stanu początkowego $x(0) = 0$ ? W tym przypadku rozwiązania są niejednoznaczne.


\begin{verbatim}
var('t')
p1=plot(t*t,(t,0,1),figsize=(6, 3), legend_label="x(0)=1", color='blue' )
p2=plot(0,(t,0,1),figsize=(6, 3), legend_label="x(0)=0", color='red' )
show(p1+p2, frame=True, axes=False)
\end{verbatim}


Co jest takiego charakterystycznego w ostatnim przykładzie, że pojawia się niejednoznaczność rozwiązania równania różniczkowego?  Na to pytanie daje odpowiedź  twierdzenie o jednoznaczności rozwiązania równania różniczkowego. Potrzebna nam będzie własność funkcji:

Mówimy, że funkcja $f(x)$ spełnia  warunek Lipschitza na zbiorze otwartym $U$ jeżeli istnieje taka stała $L > 0$,  że
\phantomsection\label{ch1/chI021:equation-eqn15}\begin{gather}
\begin{split}|f(x) -f(y)| \le L|x- y|\end{split}\label{ch1/chI021-eqn15}
\end{gather}
dla wszystkich $x, y2 \in U$.

Warunek Lipschitza można zapisać w postaci
\phantomsection\label{ch1/chI021:equation-eqn16}\begin{gather}
\begin{split}|f(x+h) -f(x)| \le L h \quad \quad \mbox{lub jako} \quad \quad \frac{f(x+h) - f(x)}{h}| \le L\end{split}\label{ch1/chI021-eqn16}
\end{gather}
Z tego wynika że jeżeli  $f(x)$ ma ograniczoną pochodną, to spełnia warunek Lipschitza. Są  oczywiście nieróżniczkowalne funkcje, które spełniają warunek Lipschitza.

Twierdzenie Picarda: Jeżeli funkcja $f(x)$ jest ciągła w $U$ oraz spełnia warunek Lipschtza w  $U$ wówczas równanie różniczkowe
\phantomsection\label{ch1/chI021:equation-eqn17}\begin{gather}
\begin{split}\frac{dx}{dt} = f(x), \qquad x(0) = x_0\end{split}\label{ch1/chI021-eqn17}
\end{gather}
ma dokładnie jedno rozwiązanie w $U$.

Istnieje kilka  modyfikacji tego twierdzenia, ale na nasze potrzeby ta najprostsza wersja jest wystarczająca.

W przypadku układu równań różniczkowych, warunek Lipschitza ma postać
\phantomsection\label{ch1/chI021:equation-eqn18}\begin{gather}
\begin{split}|F_i(x_1, x_2, x_3, ..., x_n) - F_i(y_1, y_2, y_3, ..., y_n)| \le L  \sum_{k=1}^n|x_k-y_k|\end{split}\label{ch1/chI021-eqn18}
\end{gather}
Nierówność ta musi  być spełniona dla wszystkich funkcji $F_i$ i twierdzenie Picarda brzmi podobnie. Warunek Lipschitza jest spełniony gdy pochodne cząstkowe są ograniczone,
\phantomsection\label{ch1/chI021:equation-eqn19}\begin{gather}
\begin{split}\lvert\frac{\partial F_i}{\partial x_k}\rvert \le K\end{split}\label{ch1/chI021-eqn19}
\end{gather}
dla dodatnich $K$.

Teraz możemy odpowiedzieć, dlaczego w 3 przykładzie rozwiązanie jest niejednoznaczne: funkcja $f(x) = 2\sqrt x$ nie spełnia warunku Lipschitza ponieważ pochodna
\phantomsection\label{ch1/chI021:equation-eqn20}\begin{gather}
\begin{split}\frac{df(x)}{dx} = \frac{1}{\sqrt x}\end{split}\label{ch1/chI021-eqn20}
\end{gather}
w punkcie $x=0$ jest rozbieżna. W punktach $x>0$  pochodna ma wartość skończoną i jest spełnione twierdzenie Picarda. Dlatego też  rozwiązania są jednoznaczne dla $x(0) > 0$.


\subsection{Przestrzeń fazowa}
\label{ch1/chI021:przestrzen-fazowa}
Jeszcze raz przepiszemy równania różniczkowe   \eqref{ch1/chI021-eqn5}  w  notacji:
\phantomsection\label{ch1/chI021:equation-eqn21}\begin{gather}
\begin{split}\frac{dx_1}{dt} = F_1(x_1, x_2, x_3, ..., x_n)\end{split}\label{ch1/chI021-eqn21}\\\begin{split}\frac{dx_2}{dt} = F_2(x_1, x_2, x_3, ..., x_n)\end{split}\notag\\\begin{split}\vdots\end{split}\notag\\\begin{split}\frac{dx_n}{dt} = F_n(x_1, x_2, x_3, ..., x_n)\end{split}\notag
\end{gather}
Powyższy układ równań różniczkowych  definiuje pewien układ dynamiczny (matematyczna definicja układu dynamicznego może być bardzo abstrakcyjna, ale na nasze potrzeby wystarczy to, co napisaliśmy).  Zbiór wszystkich możliwych wartości $\{x_1, x_2, x_3, ..., x_n\}$ tworzy zbiór który nazywamy przestrzenią fazową układu \eqref{ch1/chI021-eqn24}. Wymiar tej przestrzeni wynosi $n$, czyli tyle ile jest równań różniczkowych.

W zależności od kontekstu, będziemy stosowali różne zapisy tych samych równań.

Przykłady:
\begin{enumerate}
\item {} 
Jedno równanie różniczkowe. Zwykle będziemy stosowali  zapis

\end{enumerate}
\begin{quote}
\phantomsection\label{ch1/chI021:equation-eqn25}\begin{gather}
\begin{split}\frac{dx}{dt} = \dot x = f(x)\end{split}\label{ch1/chI021-eqn25}
\end{gather}
Przestrzeń fazowa  jest 1-wymiarowa.
\end{quote}
\begin{enumerate}
\setcounter{enumi}{1}
\item {} 
Dwa równania różniczkowe. Zwykle będziemy stosowali  zapis

\end{enumerate}
\begin{quote}
\phantomsection\label{ch1/chI021:equation-eqn26}\begin{gather}
\begin{split}\frac{dx}{dt} = \dot x = f(x, y)\end{split}\label{ch1/chI021-eqn26}\\\begin{split}\frac{dy}{dt} = \dot y= g(x, y)\end{split}\notag
\end{gather}
Przestrzeń fazowa  jest 2-wymiarowa.
\end{quote}
\begin{enumerate}
\setcounter{enumi}{2}
\item {} 
Trzy  równania różniczkowe. Zwykle będziemy stosowali  zapis

\end{enumerate}
\begin{quote}
\phantomsection\label{ch1/chI021:equation-eqn28}\begin{gather}
\begin{split}\frac{dx}{dt} = \dot x = f(x, y, z)\end{split}\label{ch1/chI021-eqn28}\\\begin{split}\frac{dy}{dt} = \dot y= g(x, y, z)\end{split}\notag\\\begin{split}\frac{dz}{dt} = \dot z= h(x, y, z)\end{split}\notag
\end{gather}
Przestrzeń fazowa  jest 3-wymiarowa.
\end{quote}
\begin{enumerate}
\setcounter{enumi}{3}
\item {} 
Równanie Newtona dla cząstki poruszającej się tylko wzdłuż osi $OX$ na którą działa siła $F(x)$ zależna tylko od położenia  ma postać

\end{enumerate}
\begin{quote}
\phantomsection\label{ch1/chI021:equation-eqn31}\begin{gather}
\begin{split}m \ddot x= F(x)\end{split}\label{ch1/chI021-eqn31}
\end{gather}
gdzie $m$ jest masą czastki. Jest to równanie różniczkowe 2-go rzędu i jest ono  równoważne układowi 2 równań różniczkowych 1-go rzędu:
\phantomsection\label{ch1/chI021:equation-eqn32}\begin{gather}
\begin{split}\dot x = v\end{split}\label{ch1/chI021-eqn32}\\\begin{split}\dot v = \frac{1}{m} F(x)\end{split}\notag
\end{gather}
Przestrzeń fazowa  jest 2-wymiarowa: jest to zbiór możliwych położeń i prędkości cząstki, $\{x, v\}$.
Mimo swej prostoty, ten model jest niesłychanie ważny. Stanowi on punkt wyjścia dla zrozumienia wielu ważnych
aspektów układów dynamicznych.
\end{quote}


\subsubsection{Geometryczne własności przestrzeni fazowej}
\label{ch1/chI021:geometryczne-wlasnosci-przestrzeni-fazowej}

\paragraph{Krzywa fazowe}
\label{ch1/chI021:krzywa-fazowe}
Aby uniknąć na tym etapie abstrakcyjnych definicji, będziemy rozważać dla przykładu 2-wymiarowy układ dynamiczny
\phantomsection\label{ch1/chI021:equation-eqn33}\begin{gather}
\begin{split} \dot x = f(x, y), \quad \quad x(0) = x_0\end{split}\label{ch1/chI021-eqn33}\\\begin{split} \dot y= g(x, y),\quad \quad y(0) = y_0\end{split}\notag
\end{gather}
Przestrzeń fazowa jest 2-wymiarowa. Może to być płaszczyzna lub jej część. Ale może to być bardziej skomplikowany zbiór 2-wymiarowy. Na przykład może to być sfera (podobna do  powierzchnii piłki), może to być torus (podobny do dętki rowerowej). Mogą to być jeszcze bardziej skomplikowane obiekty 2-wymiarowe. Ale dla naszych celów wystarczy rozważać płaszczyznę. Na płaszczyźnie można estetycznie przedstawiać coś w formie rysunków. Wprowadzamy na płaszczyźnie kartezjański układ współrzędnych o osiach OX i OY. Warunek początkowy $\{x_0=x(0), y_0=y(0)\}$ jest punktem o odpowiednich współrzędnych. Rozwiązyjemy powyższy układ równań różniczkowych numerycznie przy pomocy najprostszego schematu:
\phantomsection\label{ch1/chI021:equation-eqn35}\begin{gather}
\begin{split}\frac{x(t+h) - x(t)}{h} = f(x(t), y(t))\end{split}\label{ch1/chI021-eqn35}\\\begin{split}\frac{y(t+h) - y(t)}{h} = g(x(t), y(t))\end{split}\notag
\end{gather}
Przepiszemy to w postaci
\phantomsection\label{ch1/chI021:equation-eqn37}\begin{gather}
\begin{split}x(t+h) = x(t) + f(x(t), y(t)) h\end{split}\label{ch1/chI021-eqn37}\\\begin{split}y(t+h) = y(t) + g(x(t), y(t)) h\end{split}\notag
\end{gather}\begin{enumerate}
\item {} 
Obliczenia numeryczne musimy zacząć od warunku początkowego w chwili $t=0$, czyli w pierwszym kroku obliczamy

\end{enumerate}
\begin{quote}
\phantomsection\label{ch1/chI021:equation-eqn39}\begin{gather}
\begin{split}x_1 =x(h) = x(0) + f(x(0), y(0)) h\end{split}\label{ch1/chI021-eqn39}\\\begin{split}y_1 = y(h) = y(0) + g(x(0), y(0)) h\end{split}\notag
\end{gather}
Na płaszczyżnie otrzymujemy punkt o współrzędnych $\{x_1, y_1\}$. Zaznaczmy go na płaszczyźnie. Teraz mamy już 2 punkty:
\phantomsection\label{ch1/chI021:equation-eqn41}\begin{gather}
\begin{split}\{x_0, y_0\}, \quad \quad \{x_1, y_1\}\end{split}\label{ch1/chI021-eqn41}
\end{gather}\end{quote}
\begin{enumerate}
\setcounter{enumi}{1}
\item {} 
W następnym kroku wybieramy czas $t=h$:

\end{enumerate}
\begin{quote}
\phantomsection\label{ch1/chI021:equation-eqn42}\begin{gather}
\begin{split}x_2 =x(h+h) = x(2h) =  x(h) + f(x(h), y(h)) h\end{split}\label{ch1/chI021-eqn42}\\\begin{split}y_2 = y(h+h) = y(2h) =  y(h) + g(x(h), y(h)) h\end{split}\notag
\end{gather}
Wykorzystamy oznaczenie jek wyżej: $x_1 =  x(h),  y_1 = y(h)$ i przepiszemy te równania w postaci
\phantomsection\label{ch1/chI021:equation-eqn43}\begin{gather}
\begin{split}x_2 =  x_1 + f(x_1, y_1) h\end{split}\label{ch1/chI021-eqn43}\\\begin{split}y_2 =  y_1 + g(x_1, y_1) h\end{split}\notag
\end{gather}
Na płaszczyżnie otrzymujemy punkt o współrzędnych $\{x_2, y_2\}$. Zaznaczmy go na płaszczyźnie. Teraz mamy już 3 punkty:
\phantomsection\label{ch1/chI021:equation-eqn45}\begin{gather}
\begin{split}\{x_0, y_0\}, \quad \quad \{x_1, y_1\},  \quad \quad \{x_2, y_2\}\end{split}\label{ch1/chI021-eqn45}
\end{gather}\end{quote}
\begin{enumerate}
\setcounter{enumi}{2}
\item {} 
Widzimy od razu, że w 3 kroku otrzymujemy równania

\end{enumerate}
\begin{quote}
\phantomsection\label{ch1/chI021:equation-eqn46}\begin{gather}
\begin{split}x_3 =  x_2 + f(x_2, y_2) h\end{split}\label{ch1/chI021-eqn46}\\\begin{split}y_3 =  y_2 + g(x_2, y_2) h\end{split}\notag
\end{gather}
i otrzymujemy punkt o współrzędnych $\{x_3, y_3\}$.
\end{quote}
\begin{enumerate}
\setcounter{enumi}{3}
\item {} 
Zauważamy, że w n-tym kroku otrzymujemy równania

\end{enumerate}
\begin{quote}
\phantomsection\label{ch1/chI021:equation-eqn48}\begin{gather}
\begin{split}x_n =  x_{n-1} + f(x_{n-1}, y_{n-1}) h\end{split}\label{ch1/chI021-eqn48}\\\begin{split}y_n =  y_{n-1} + g(x_{n-1}, y_{n-1}) h\end{split}\notag
\end{gather}\end{quote}
\begin{enumerate}
\setcounter{enumi}{21}
\item {} 
Częściej pisze się, co się otrzymuje w następnym kroku, czyli n+1 :

\end{enumerate}
\begin{quote}
\phantomsection\label{ch1/chI021:equation-eqn50}\begin{gather}
\begin{split}x_{n+1} =  x_n + f(x_n, y_n) h\end{split}\label{ch1/chI021-eqn50}\\\begin{split}y_{n+1} =  y_n + g(x_n, y_n) h\end{split}\notag
\end{gather}\end{quote}

Otrzymujemy równania rekurencyjne, które pozwalają wyznaczyć ewolucję układu, czyli rozwiązanie numeryczne układu równań różniczkowych. Na płaszczyżnie $XY$ otrzymujemy ciąg punktów o współrzędnych
\phantomsection\label{ch1/chI021:equation-eqn52}\begin{gather}
\begin{split}\{x_n, y_n\}\end{split}\label{ch1/chI021-eqn52}
\end{gather}
Jeżeli przyrost czasu $h$ jest odpowiednio mały, to ten ciąg punktów łączymy linią ciągłą i otrzymujemy  krzywą na płaszczyźnie. Ta krzywa nazywa się krzywą fazową układu dynamicznego.   Mając narysowaną taką krzywą fazową, możemy wnioskować o ewolucji układu i cechach charakterystycznych zachowania się układu w czasie $t$.       Poniżej przedstawiamy dwa przykłady: krzywe fazowe dla oscylatora harmonicznego  i oscylatora harmonicznego tłumionego.


\subparagraph{Oscylator harmoniczny}
\label{ch1/chI021:oscylator-harmoniczny}
Przykładem oscylatora harmonicznego jest ciało o masie $m$ przyczepione do sprężyny i poruszające się wzdłuż osi $OX$.  Siła działające na to ciało jest proporcjonalna do wychylenia $x$ od położenia równowagi i przeciwnie skierowana do wychylenia; gdy rozciągamy sprężynę w kierunku większych dodatnich wartości $x$ to siła działa w kierunku ujemnych wartości $x$:
\phantomsection\label{ch1/chI021:equation-eqn53}\begin{gather}
\begin{split} F = -kx\end{split}\label{ch1/chI021-eqn53}
\end{gather}
gdzie $k$ charakteryzuje ``sprężystość'' sprężyny. Równanie Newtona ma postać:
\phantomsection\label{ch1/chI021:equation-eqn54}\begin{gather}
\begin{split}m\ddot x = -kx, \quad \quad \mbox{lub w postaci} \quad \quad \ddot x = -(k/m) x = -\omega^2 x\end{split}\label{ch1/chI021-eqn54}
\end{gather}
gdzie $\omega^2 = k/m$. To równanie drugiego rzędu jest równoważne 2 równaniom pierwszego rzędu:
\phantomsection\label{ch1/chI021:equation-eqn55}\begin{gather}
\begin{split}\dot x = y, \quad \quad x(0) = x_0\end{split}\label{ch1/chI021-eqn55}\\\begin{split}\dot y = -\omega^2 x, \quad \quad y(0) = y_0\end{split}\notag
\end{gather}

\subparagraph{Tłumiony oscylator harmoniczny}
\label{ch1/chI021:tlumiony-oscylator-harmoniczny}
Jeżeli w poprzednim przykładzie założymy bardziej realistyczną sytuację, w której  układ nie jest w próżni, ale znajduje się w środowisku (np. w powietrzu, w wodzie lub innej cieczy), to na ciało działa dodatkowa siła, a mianowicie siła tarcia  (tłumienia) $F_d$. Siła tarcia jest proporcjonalna do prędkości cząstki i przeciwnie skierowana do kierunku ruchu
\phantomsection\label{ch1/chI021:equation-eqn57}\begin{gather}
\begin{split}F_d = -\gamma_0 v = -\gamma_0 \dot x\end{split}\label{ch1/chI021-eqn57}
\end{gather}
gdzie $\gamma_0$ nazywa sie współczynnikiem tarcia (tłumienia).

Siła tarcia jest związana z oddziaływaniem ciała z cząsteczkami otoczenia. Otoczenie stawia opór gdy ciało porusza się w nim i im większa jest prędkość ciała tym większy jest opór otoczenia. Doświadczamy to, gdy biegniemy cali zanurzeni w wodzie.

Uwzględniając siłę tarcia, równanie Newtona przyjmuje postać
\phantomsection\label{ch1/chI021:equation-eqn58}\begin{gather}
\begin{split}m\ddot x = -kx - \gamma_0 \dot x, \quad \quad \mbox{lub w postaci} \quad \quad \ddot x = -\frac{k}{m} x - \frac{\gamma_0}{m} x = -\omega^2 x - \gamma \dot x\end{split}\label{ch1/chI021-eqn58}
\end{gather}
gdzie $\omega^2 = k/m$  oraz $\gamma = \gamma_0/m$. Równanie  powyższe  jest równoważne 2 równaniom pierwszego rzędu:
\phantomsection\label{ch1/chI021:equation-eqn59}\begin{gather}
\begin{split}\dot x = y, \quad \quad x(0) = x_0\end{split}\label{ch1/chI021-eqn59}\\\begin{split}\dot y = -\gamma y -\omega^2 x, \quad \quad y(0) = y_0\end{split}\notag
\end{gather}
Oczywiście gdy $\gamma = 0$, wówczas  otrzymujemy równanie oscylatora harmonicznego bez tarcia (nietłumionego).

Poniżej przedstawiamy krzywe fazowe dla tych 2 przykładów.


\subparagraph{Oscylator harmoniczny bez tarcia}
\label{ch1/chI021:oscylator-harmoniczny-bez-tarcia}

\begin{verbatim}
var('x y')
def schemat_eulera2D(vec, ics, Tlist):
 i = 0
 dx = [ics[0]]
 dy = [ics[1]]
 h = Tlist[i+1] - Tlist[i]
 iks(x,y) = vec[0R]*h
 igrek(x,y) = vec[1R]*h
 for time in Tlist:
     dx.append(dx[i] + iks(dx[i],dy[i]))
     dy.append(dy[i] + igrek(dx[i],dy[i]))
     i += 1
 return zip(dx,dy)
#
@interact(layout={'top':[['omega','x0','y0']],'bottom':[['T','h']]})
def _(title=['a','b'], h=selector(['0.005','0.01','0.05','0.1','0.5','1'], default='0.1', buttons=True),x0=input_box(2,label=r'$x_0$', width=10), y0=input_box(4,label=r'$y_0$', width=10), T=input_box(0, width=10), omega=input_box(1,label=r'$\omega$', width=10)):
 global oscylator_nietlumiony, background
 if T == 0:
     T = 2*pi/omega
 listT = srange(0,T,float(h), include_endpoint=True)
 background = desolve_odeint(vector([y,-omega^2*x]), [x0, y0], srange(0,T+0.1,0.1,include_endpoint=True), [x,y])
 oscylator_nietlumiony = schemat_eulera2D([y,-omega^2*x], [x0, y0], listT)
 print r'Parametry modelu'
 html(r'$\omega=%s, x_0=%s, y_0=%s$'%(omega,x0,y0))
 print r'Parametry symulacji'
 html(r'$h=%s, T=%s$'%(h,T))
 print '\nDla T=0 lista generowana jest automatycznie dla jednego okresu własnego oscylatora'
#
@interact
def _(krok=slider(1, len(oscylator_nietlumiony), 1, default=1, label=r'krok')):
...
 buf = zip(*oscylator_nietlumiony)
 minx, maxx, miny, maxy = min(buf[0]), max(buf[0]), min(buf[1]), max(buf[1])
 kroki = oscylator_nietlumiony[:krok]
 kroki_plot = list_plot(kroki, figsize=(4,4), axes_labels=[r'$x$',r'$y$'], size=30, xmin=minx, xmax=maxx, ymin=miny, ymax=maxy)
...
 txt_plot = text(r'$[x_0,y_0]$',kroki[0],vertical_alignment='bottom',horizontal_alignment='left')
 for i in range(1,len(kroki)):
     txt_plot += text(r'$[x_{%d},y_{%d}]$'%(i,i),kroki[i],vertical_alignment='bottom',horizontal_alignment='left')
...
 full_plot = list_plot(oscylator_nietlumiony, plotjoined=1, figsize=(4,4), axes_labels=[r'$x$',r'$y$'])
 full_plot += list_plot(background.tolist(), plotjoined=1, color='grey', alpha=0.5)
 html.table([["krzywe fazowe dla oscylatora bez tarcia",""],[full_plot+kroki_plot,kroki_plot+txt_plot]])
\end{verbatim}


W przypadku oscylatora nietłumionego, krzywe fazowe są zamknięte. Cząstka z biegiem czasu porusza się tak, że położenie $x$ i prędkość $v=y$  leżą na krzywej fazowej. Ponieważ jest to krzywa zamknięta, to po pewnym czasie cząstka znowu ``przebiega'' punkty, które się powtarzają. Powtarzanie się jest cechą charakterystyczną ruchu okresowego. Tak więc krzywa fazowa zamknięta przedstawia ruch okresowy (periodyczny). Okres takiego ruchu periodycznego to czas potrzebny na to, aby cząstka startując od punktu np. $\{x_0, y_0\}$ i  poruszając się po krzywej fazowej dotarła znowu do tego samego punktu  $\{x_0, y_0\}$.

W przypadku oscylatora tłumionego, krzywą fazową jest spirala zwijająca sie do punktu $\{0, 0\}$. Ruch po spirali oznacza, że zarówno $x$ jak i $v=y$ maleją w czasie i dla długich czasów położenie $x$ oraz prędkość $v$ dążą do zera, czyli cząstka zwalnia i na końcu zatrzymuje się. To jest ruch tłumiony: amplituda drgań maleje w czasie. To jest to, co obserwujemy w ruchu kulki zawieszonej na nitce: kulka wukonuje coraz to mniejsze drgania i po długim czasie wisi pionowo ( to jest coś co nazywa sie stanem równowagi lub położeniem stacjonarnym).

Gdy mamy bardziej skomplikowane krzywe fazowe, ich ``rozszyfrowanie'' może być trudniejsze. Ale ogólna zasada jest taka: gdy $x$ rośnie to oznacza wzrost położenia cząstki. Gdy $y$ maleje to oznacza, że maleje prędkość cząstki.  Gdy $x$ maleje to maleje współrzędna położenia cząstki. Gdy $y$ rośnie to rośnie prędkość cząstki.


\subparagraph{Tłumiony oscylator harmoniczny}
\label{ch1/chI021:id1}

\begin{verbatim}
var('x y')
def schemat_eulera2D(vec, ics, Tlist):
 i = 0
 dx = [ics[0]]
 dy = [ics[1]]
 h = Tlist[i+1] - Tlist[i]
 iks(x,y) = vec[0R]*h
 igrek(x,y) = vec[1R]*h
 for time in Tlist:
     dx.append(dx[i] + iks(dx[i],dy[i]))
     dy.append(dy[i] + igrek(dx[i],dy[i]))
     i += 1
 return zip(dx,dy)
#
@interact(layout={'top':[['omega','ggamma','x0','y0']],'bottom':[['T','h']]})
def _(title=['a','b'], h=selector(['0.05','0.01','0.1','0.5','1'], default='0.1', buttons=True),x0=input_box(2,label=r'$x_0$', width=10), y0=input_box(4,label=r'$y_0$', width=10), T=input_box(0, width=10), omega=input_box(1,label=r'$\omega$', width=10), ggamma=input_box(0.5,label=r'$\gamma$', width=10)):
 global oscylator_tlumiony, globggamma, globomega, background2
 globggamma = ggamma
 globomega = omega
 if T == 0:
     T = 2*pi/omega
 listT = srange(0,T,float(h),include_endpoint=True)
 background2 = desolve_odeint(vector([y,-omega^2*x-ggamma*y]), [x0, y0], srange(0,2*pi/omega+0.1,0.1,include_endpoint=True), [x,y])
 oscylator_tlumiony = schemat_eulera2D([y,-omega^2*x-ggamma*y], [x0, y0], listT)
 print r'Parametry modelu'
 html(r'$\gamma=%s, \omega=%s, x_0=%s, y_0=%s$'%(ggamma,omega,x0,y0))
 print r'Parametry symulacji'
 html(r'$h=%s, T=%s$'%(h,T))
 print '\nDla T=0 lista generowana jest automatycznie dla jednego okresu własnego oscylatora'
#
@interact
def _(krok=slider(1, len(oscylator_tlumiony), 1, default=1, label=r'krok')):
 buf = zip(*oscylator_tlumiony)
 minx, maxx, miny, maxy = min(buf[0]), max(buf[0]), min(buf[1]), max(buf[1])
 kroki = oscylator_tlumiony[:krok]
 kroki_plot = list_plot(kroki, figsize=(4,4), axes_labels=[r'$x$',r'$y$'], size=30, xmin=minx, xmax=maxx, ymin=miny, ymax=maxy)
 txt_plot = text(r'$[x_0,y_0]$',kroki[0],vertical_alignment='bottom',horizontal_alignment='left')
 for i in range(1,len(kroki)):
     txt_plot += text(r'$[x_{%d},y_{%d}]$'%(i,i),kroki[i],vertical_alignment='bottom',horizontal_alignment='left')
 full_plot = list_plot(oscylator_tlumiony, plotjoined=1, figsize=(4,4), axes_labels=[r'$x$',r'$y$'])
 full_plot += list_plot(background2.tolist(), plotjoined=1, color='grey', alpha=0.5)
 html.table([["krzywe fazowe dla oscylatora tłumionego",""],[full_plot+kroki_plot,kroki_plot+txt_plot]])
\end{verbatim}



\subsection{Pole wektorowe}
\label{ch1/chI021:pole-wektorowe}
Prawe strony układu równań różniczkowych
\phantomsection\label{ch1/chI021:equation-eqn61}\begin{gather}
\begin{split} \dot x = f(x, y), \quad \quad x(0) = x_0\end{split}\label{ch1/chI021-eqn61}\\\begin{split} \dot y= g(x, y),\quad \quad y(0) = y_0\end{split}\notag
\end{gather}
można traktować jak składowe pewnego pola wektorowego:
\phantomsection\label{ch1/chI021:equation-eqn63}\begin{gather}
\begin{split}\vec F = [F_x, F_y] = [f(x, y), g(x, y)]\end{split}\label{ch1/chI021-eqn63}
\end{gather}
W każdym punkcie płaszczyzny o współrzędnych $\{x, y\}$ rysujemy wektor  o składowych  $[f(x, y), g(x, y)]$. W ten sposób otrzymujemy pole wektorowe. No dobrze, ale jaką informację o układzie można ``wyciągnąć'' z tego pola wektorowego. Wykonajmy takie oto ćwiczenie numeryczne: Startujemy z warunku początkowego $\{x_0, y_0\}$ i rysujemy w tym punkcie wektor o składowych $[f(x_0, y_0), g(x_0, y_0)]$ czyli
\phantomsection\label{ch1/chI021:equation-eqn64}\begin{gather}
\begin{split}\mbox{w punkcie }  \quad \{x_0, y_0\}   \quad \mbox{rysujemy wektor o składowych} \quad [f(x_0, y_0), g(x_0, y_0)]\end{split}\label{ch1/chI021-eqn64}
\end{gather}
Jak poprzednio, rozwiązujemy numerycznie układ równań różniczkowych i obliczamy  $\{x_1, y_1\}$:
\phantomsection\label{ch1/chI021:equation-eqn65}\begin{gather}
\begin{split}\mbox{w punkcie }  \quad \{x_1, y_1\}   \quad \mbox{rysujemy wektor o składowych} \quad [f(x_1, y_1), g(x_1, y_1)]\end{split}\label{ch1/chI021-eqn65}
\end{gather}
Następnie obliczamy  $\{x_2, y_2\}$:
\phantomsection\label{ch1/chI021:equation-eqn66}\begin{gather}
\begin{split}\mbox{w punkcie }  \quad \{x_2, y_2\}   \quad \mbox{rysujemy wektor o składowych} \quad [f(x_2, y_2), g(x_2, y_2)]\end{split}\label{ch1/chI021-eqn66}
\end{gather}
W n-tym kroku iteracji obliczamy  $\{x_n, y_n\}$:
\phantomsection\label{ch1/chI021:equation-eqn67}\begin{gather}
\begin{split}\mbox{w punkcie }  \quad \{x_n, y_n\}   \quad \mbox{rysujemy wektor o składowych} \quad [f(x_n, y_n), g(x_n, y_n)]\end{split}\label{ch1/chI021-eqn67}
\end{gather}
Ponieważ wszystkie powyższe punkty  $\{x_i, y_i\}$ leżą na krzywej fazowej, to wektory $[f(x_i, y_i), g(x_i, y_i)]$  są przyczepione do tych krzywych fazowych. Zauważamy, że wektory te są styczne do krzywej fazowej. Jeżeli $\{x_i, y_i\}$ miały by interpretacje położenia cząstki na płaszczyźnie, to wektory  $[f(x_i, y_i), g(x_i, y_i)]$ miały by interpretację prędkości ponieważ $\dot x = f(x, y)$ oraz $\dot y = g(x, y)$. Wiemy, że $\dot x = v_x$ jest x-ową składową prędkości, z kolei $\dot y = v_y$ jest y-ową składową prędkości. Innymi słowy, otrzymane pole wektorowe to pole prędkości fikcyjnej cząstki.


\begin{verbatim}
var('x y')
@interact(layout={'top':[['omega','ggamma','x0','y0']],'bottom':[['T','h']]})
def _(title=['a','b'], h=selector(['0.05','0.01','0.1','0.5','1'], default='0.1', buttons=True),x0=input_box(2,label=r'$x_0$', width=10), y0=input_box(4,label=r'$y_0$', width=10), T=input_box(20, width=10), omega=input_box(1,label=r'$\omega$', width=10), ggamma=input_box(0.5,label=r'$\gamma$', width=10)):
 global oscylator_tlumiony, globggamma, globomega
 globggamma = ggamma
 globomega = omega
 listT = srange(0,T,float(h))
 oscylator_tlumiony = desolve_odeint(vector([y,-omega^2*x-ggamma*y]), [x0, y0], listT, [x,y])
 print r'Parametry modelu'
 html(r'$\gamma=%s, \omega=%s, x_0=%s, y_0=%s$'%(ggamma,omega,x0,y0))
 print r'Parametry symulacji'
 html(r'$h=%s, T=%s$'%(h,T))
vf = lambda u,a,b: (u[0]+u[1],u[1]-a*u[0]-b*u[1])
#
@interact
def _(krok=slider(1, len(oscylator_tlumiony), 1, default=1, label=r'krok')):
 kroki = oscylator_tlumiony[:krok]
 kroki_plot = list_plot(kroki.tolist(), figsize=(4,4), axes_labels=[r'$x$',r'$y$'], size=30, xmin=-4.5, xmax=4.5, ymin=-4.5, ymax=4.5)
 pole_wektorowe = arrow(kroki[0],vf(kroki[0],globomega^2,globggamma),color='red',xmax=vf(kroki[0],globomega^2,globggamma)[0])
 for krok in kroki[1:]:
     pole_wektorowe += arrow(krok,vf(krok,globomega^2,globggamma),color='red', width=.4)
 txt_plot = text(r'$[x_0,y_0]$',kroki[0],vertical_alignment='bottom',horizontal_alignment='left')
 for i in range(1,len(kroki)):
     txt_plot += text(r'$[x_{%d},y_{%d}]$'%(i,i),kroki[i],vertical_alignment='bottom',horizontal_alignment='left')
 shadowplot = list_plot(oscylator_tlumiony.tolist(), plotjoined=1, figsize=(4,4), axes_labels=[r'$x$',r'$y$'], alpha=0.2)
 full_plot = list_plot(oscylator_tlumiony.tolist(), plotjoined=1, figsize=(4,4), axes_labels=[r'$x$',r'$y$']) + plot_vector_field([y,-globomega^2*x-globggamma*y], (x, -4.5, 4.5), (y, -4.5, 4.5), plot_points=20, color='lime')
 html.table([["krzywe fazowe dla oscylatora tłumionego",""],[full_plot+kroki_plot,shadowplot+kroki_plot+txt_plot+pole_wektorowe]])
\end{verbatim}


Poniżej znajdziecie komórkę, w której zachęcamy wszystkich do poeksperymentowania z różnymi modelami. Miłej zabawy...

\begin{notice}{tip}{Wskazówka:}
Aby całość zadziałała poprawnie musicie zadeklarować model podając \code{dx} i \code{dy},
podać wartości wszystkich parametrów (teraz jest tylko \code{alpha})
oraz warunki początkowe \code{x0} i \code{y0}.
Na koniec zdecudujcie jaki okres dynamiki punktu chcecie symulować
przypisując do zmiennej \code{T} odpowiednią wartość.
\end{notice}


\begin{verbatim}
#########
# Model #
#########
# zmienne
var('x y')
#
# parametry
# UWAGA: jeżeli Twój model będzie zależny od innych parametrów
#        tu właśnie musisz je wszystkie wyspecyfikować
alpha = 1
#
# warunki początkowe
x0 = 1
y0 = 1
#
# model
dx = y
dy = -alpha*x - y
#
# czas (T) symulacji
T = 12
#
###################################################
# Symulacje + wizualizacja                        #
###################################################
listT = srange(0,T,0.1,include_endpoint=True)
numeryka = desolve_odeint(vector([dx, dy]), [x0, y0], listT, [x,y])
przestrzen_fazowa = list_plot(numeryka.tolist(), plotjoined=1, figsize=(4,4), axes_labels=[r'$x$',r'$y$'])
pole_wektorowe = plot_vector_field([dx,dy], (x, numeryka[:,0].min(), numeryka[:,0].max()), (y, numeryka[:,1].min(), numeryka[:,1].max()), plot_points=10, color='lime')
show(przestrzen_fazowa+pole_wektorowe)
\end{verbatim}



\section{Dynamiczne układy zachowawcze i dysypatywne}
\label{ch1/chI022::doc}\label{ch1/chI022:dynamiczne-uklady-zachowawcze-i-dysypatywne}

\subsection{Układy zachowawcze}
\label{ch1/chI022:uklady-zachowawcze}
Niektóre układy równań różniczkowych mają specyficzną strukturę i ukryte własności. Przykłady z fizyki mają taką specyficzną strukturę. Rozpatrzmy równanie Newtona dla cząstki o jednym stopniu swobody w postaci:
\phantomsection\label{ch1/chI022:equation-eqn1}\begin{gather}
\begin{split}m\ddot x = F(x) = -\frac{dV(x)}{dx} = - V'(x)\end{split}\label{ch1/chI022-eqn1}
\end{gather}
gdzie siła $F(x)$ jest potencjalna, tzn.  istnieje taka funkcja $V(x)$, że $F(x) = -V'(x)$.  Oznaczenie $V'(x)$ jest pochodną funkcji $V(x)$ względem zmiennej $x$. Funkcja  $V(x)$ nazywa się energią potencjalną, ale my będziemy pisali krótko - potencjał. Jeżeli znamy siłę $F(x)$ to potencjał można znależć obliczając całkę:
\phantomsection\label{ch1/chI022:equation-eqn2}\begin{gather}
\begin{split}V(x) = - \int_a^{\,x}  F(y) dy\end{split}\label{ch1/chI022-eqn2}
\end{gather}
gdzie $a$ jest dowolną liczbą wybieraną tak jak nam wygodnie. Np. możemy wybrać tak, aby w pewnym punkcie potencjał był zerowy lub nieskończony.

Równanie Newtona jest równaniem 2-go rzędu, autonomicznym.   Zapiszemy je w postaci
\phantomsection\label{ch1/chI022:equation-eqn3}\begin{gather}
\begin{split}\dot x = v, \quad \quad x(0)=x_0,\end{split}\label{ch1/chI022-eqn3}\\\begin{split}m\dot v = F(x) = -V'(x),  \qquad v(0)=v_0.\end{split}\notag
\end{gather}
Z tego wynika, że przestrzeń fazowa układu jest 2-wymiarowa $\{x, v\}$. W tej przestrzeni fazowej może analizować krzywe fazowe. Można zauważyć, że położenie $x(t)$ oraz prędkość $v(t)$ icząstki zmieniają się w czasie zgodnie z równaniem Newtona, to istnieje pewna funkcja (kombinacja) tych 2 funkcji $x(t)$ oraz $v(t)$, która nie zmienia się w czasie:
\phantomsection\label{ch1/chI022:equation-eqn4}\begin{gather}
\begin{split}E[x(t), v(t)] = \frac{1}{2} m v^2(t) + V(x(t)) = \frac{1}{2} m v^2(0) + V(x(0)) = E[x(0), v(0)]\end{split}\label{ch1/chI022-eqn4}
\end{gather}
Wielkość $E$ nazywa się  w fizyce całkowitą energią układu i składa się z 2 części: energii kinetycznej cząstki $E_k=mv^2/2$ oraz energii potencjalnej  cząstki $E_p = V(x)$. Jeżeli $E$ nie zmienia się w czasie, to znaczy że jest to funkcja stała ze względu na czas i pochodna wzgledem czasu powinna być zero. Sprawdźmy to:
\phantomsection\label{ch1/chI022:equation-eqn5}\begin{gather}
\begin{split}\frac{dE}{dt} = \frac{d}{dt}  E[x(t), v(t)] = \frac{\partial E}{\partial x}  \frac{dx}{dt} + \frac{\partial E}{\partial v}  \frac{dv}{dt} =  V'(x)  \dot x +  mv \dot v = -F(x) v + v F(x)\end{split}\label{ch1/chI022-eqn5}
\end{gather}
gdzie skorzystaliśmy ze związku pomiędzy siłą i energią potencjalną oraz z równania Newtona.

Ponieważ $E$ nie zmienia się w czasie, to mówimy że jest to stała ruchu lub całka ruchu, lub całka pierwsza układu (ostatnie nazwy wydają  się być dziwaczne, bo w wyrażeniu dla $E$ nie widać żadnej całki).  Istnienie stałych czy też całek ruchu ułatwia analizę układów. Pokażemy to na przykładzie oscylatora harmonicznego  dla którego postać siły jest dobrze znana:
\phantomsection\label{ch1/chI022:equation-eqn6}\begin{gather}
\begin{split}F(x) = - k x = - m\omega^2 x, \qquad V(x) = \frac{1}{2} k  x^2  = \frac{1}{2} m\omega^2 x^2, \quad \quad \omega^2 = \frac{k}{m}\end{split}\label{ch1/chI022-eqn6}
\end{gather}
Prawo zachowania energii mówi, że
\phantomsection\label{ch1/chI022:equation-eqn7}\begin{gather}
\begin{split}E = \frac{1}{2} m v^2(t) + \frac{1}{2} k x^2(t) = const. = \frac{1}{2} m v^2(0) + \frac{1}{2} k x^2(0)\end{split}\label{ch1/chI022-eqn7}
\end{gather}
Ponieważ ta wielkość jest niezmienna w czasie, to określa równanie krzywej fazowej na płaszczyźnie $XY$. Łatwo zauważyć, że powyższe równanie w zmiennych $\{x, y=v\}$ ma postać
\phantomsection\label{ch1/chI022:equation-eqn8}\begin{gather}
\begin{split} m y^2 +  k x^2 =  2E\end{split}\label{ch1/chI022-eqn8}
\end{gather}
Jest to równanie elipsy:
\phantomsection\label{ch1/chI022:equation-eqn9}\begin{gather}
\begin{split}\frac{x^2}{(2E/k)} + \frac{y^2}{(2E/m)} = 1\end{split}\label{ch1/chI022-eqn9}
\end{gather}
o osiach $a=2E/k$ oraz $b=2E/m$. Narysujmy sobie takę elipsę dla, powiedzmy, $E = 2, k = 0.2$ oraz $m=1$. Wiadomo, że każdy wie jak taka elipsa będzie wyglądać, ale zrobimy to bardziej po to, żeby wyrobić sobie naturalną umiejętność używania programów typu Sage do wizualizacji i interpretacji wyników.


\begin{verbatim}
@interact(layout={'top':[['E','k','m']],})
def _(title=['elipsa'], E=input_box(2,label=r'$E$', width=10), k=input_box(0.2,label=r'$k$', width=10), m=input_box(1,label=r'$m$', width=10)):
 a = 2*E/k
 b = 2*E/m
 ellipse((0,0),a,b,0,fill=True,alpha=0.3).show()
\end{verbatim}


Elipsa jest krzywą zamkniętą, więc ruch jest periodyczny. Można sobie wyobrażać, że ruch cząstki w potencjale $V(x)$ jest  podobny do  ruchu cząstki we wnętrzu połówki sfery (w czasze). Nie jest to prawdą, ale takie wyobrażenie wyrabia w nas intuicję o własnościach ruchu. Poniżej przedstawiamy krok po kroku co zrobić, aby narysować krzywe fazowe układu.

Rysujemy wykres przedstawiający potencjal $V(x)$. Poniżej tego wykresu, z osią pionową ustawioną jak w wykresie dla potencjału, rysujemy 2 symetryczne krzywe zadane przez prawo zachowania energii $\frac{1}{2}m v^2 + V(x) = E$ czyli stąd wynika że $v = \pm \sqrt{\frac{2}{m}[(E-V(x)]}$. Te dwie krzywe $v=v(x, E)$ są krzywymi fazowymi.
Cząstka porusza się w prawo gdy prędkość jest dodatnia $v>0$ (zielona krzywa) i w lewo gdy prędkość jest ujemna $v<0$ (czerwona krzywa). Prędkość jest zero wówczas, gdy $V(x) = E$. Wynika to z prawa zachowania energii (podstaw tam $v=0$). Równanie $V(x) = E$ wyznacza punkty zwrotu $x_i$ : cząstka w tych punktach ma zerową prędkość i zmienia kierunek ruch (zawraca).

Spróbujemy, krok po kroku zanalizować równanie Newtona aby uzyskać krzywe fazowe.
\phantomsection\label{ch1/chI022:equation-eqn10}\begin{gather}
\begin{split}m \ddot{x} = F\end{split}\label{ch1/chI022-eqn10}
\end{gather}
Jeżeli siła bedzie liniowa $F=-kx$ to dostaniemy wyżej opisane zagadnienie oscylatora harmonicznego. Na początku musimy zadeklarować nazwy zmiennych oraz parametrów użytych w modelu. Pamiętaj - każdorazowo, jeżeli chcesz obliczac coś symbolicznie, trzeba taką linijkę napisać i ją wykonać. W kolejnych linijkach ustalimy parametry układu, zdefiniujemy siły z jakimi mamy do czynienia i obliczymy potencjał (całka z siły brana ze znakiem minus). W następnym kroku, z prawa zachowania energii, obliczymy teraz jak prędkość zależy od położenia (owe krzywe fazowe).


\begin{verbatim}
#0 (kilka zmiennych)
var('x v')
#parametry dla wizualizacji
x0 = 1.3
v0 = 0.3
k = 0.2
m = 1
F = -k*x
#1
V = -integral(F,x)
p1 = plot(V, xmin=-x0, xmax=x0)
p1.show(figsize=4, axes_labels=[r'$x$',r'$V(x)=%s$'%V])
#
#prawo zachowania energii
E = m*v0^2 + V(x=x0)
PZE = m*v^2 + V == E
#i jego rozwiązanie
rozw = solve(PZE, v); show(rozw)
v1=rozw[0].rhs()
v2=rozw[1].rhs()
#
#ekstremalne wychylenie
#prawo zachowania energii dla v=0
rozw = solve(PZE(v=0), x); show(rozw)
xmin = rozw[0].rhs()
xmax = rozw[1].rhs()
#punkt początkowy (tak jak powyżej)
ball = (x0,V(x=x0))
p0  = point(ball,size=30)
p0 += text(r"  punkt startowy",ball,vertical_alignment='bottom',horizontal_alignment='left',fontsize=8)
#
#ekstrema
ball = (xmax,V(x=xmax))
p0 += point(ball,size=30,color='red')
p0 += text("ekstremum_",ball,vertical_alignment='bottom',horizontal_alignment='right',color='red',fontsize=8)
p12a = line((ball,(xmax,0)),linestyle='dotted',color='grey')
ball = (xmin,V(x=xmin))
p0 += point(ball,size=30,color='red')
p0 += text("_ekstremum",ball,vertical_alignment='bottom',horizontal_alignment='left',color='red',fontsize=8)
p12a += line((ball,(xmin,0)),linestyle='dotted',color='grey')
#
#potencjał
p1 = plot(V, xmin=xmin, xmax=xmax)
#
#krzywe fazowe
p12b = line(((xmin,0),(xmin,v2(x=0))),linestyle='dotted',color='grey')
p12b += line(((xmax,0),(xmax,v2(x=0))),linestyle='dotted',color='grey')
p2 =  plot(v1, (x,xmin,xmax), color='red')
p2 += plot(v2, (x,xmin,xmax), color='green')
#
(p0+p1+p12a).show(figsize=4, axes_labels=['$x$','$V(x)$'])
(p12b+p2).show(figsize=4,xmax=xmax)
sage:
var('x y z t')
xy_wsp = [('x','x'),('y','y')]+[('z','z')]
N = len(xy_wsp)
J  = matrix(SR,N)
\end{verbatim}



\subsection{Układy potencjalne}
\label{ch1/chI022:uklady-potencjalne}
Układ o 1 stopniu swobody jest  potencjalny (tzn. istnieje potencjał $V(x)$  pod warunkiem, że siła zależy tylko od położenia cząstki, tzn. $F=F(x)$. Jeżeli siła zależy także od prędkości cząstki, tzn gdy $F=F(x, v)$, nie istnieje potencjał $V$ taki aby $F = -V' = - dV/dx$. Dla układów o wielu stopniach swobody, opisywanych układem równań Newtona
\phantomsection\label{ch1/chI022:equation-eqn11}\begin{gather}
\begin{split}m_i \frac{d^2\vec r_i}{dt^2} = \vec F_i(\vec r_1,  \vec r_2, \vec r_3, ..., \vec r_N)\end{split}\label{ch1/chI022-eqn11}
\end{gather}
dla $N$ cząstek, układ jest potencjalny, gdy istnieje taka funkcja skalarna $V(\vec r_1,  \vec r_2, \vec r_3, ..., \vec r_N)$, że siła działająca na $i$-tą cząstkę jest gradientem potencjału ze znakiem minus. Prościej jest to wyjaśnić na przykładzie 1 cząstki poruszającej się w przestrzeni 3-wymiarowej:
\phantomsection\label{ch1/chI022:equation-eqn12}\begin{gather}
\begin{split}m\frac{d^2x}{dt^2} = F_1(x, y, z) = - \frac{\partial}{\partial x} V(x, y, z),\end{split}\label{ch1/chI022-eqn12}\\\begin{split}m\frac{d^2y}{dt^2}   = F_2(x, y, z) = - \frac{\partial}{\partial y} V(x, y, z),\end{split}\notag\\\begin{split}m\frac{d^2z}{dt^2} = F_3(x, y, z) = - \frac{\partial}{\partial x} V(x, y, z).\end{split}\notag
\end{gather}
W ogólnym przypadku, gdy mamy zadane 3 składowe siły $F_1,  F_2$ oraz $F_3$, nie musi istniej tylko jedna funkcja $V$ taka aby powyższe równania były spełnione. Nasuwa się pytanie, czy istnieje proste kryterium mówiące, że układ jest potencjalny. Jeżeli
\phantomsection\label{ch1/chI022:equation-eqn13}\begin{gather}
\begin{split}\vec F = - grad \; V \qquad \mbox{to} \qquad rot\; \vec F = - rot \;grad \;V  =  - \vec \nabla \times \vec \nabla V \equiv 0\end{split}\label{ch1/chI022-eqn13}
\end{gather}
gdzie operator $\vec\nabla$ jest operatorem różniczkowania
\phantomsection\label{ch1/chI022:equation-eqn14}\begin{gather}
\begin{split}\vec\nabla = \hat e_x \frac{\partial}{\partial x} + \hat e_y \frac{\partial}{\partial y} + \hat e_y \frac{\partial}{\partial y}\end{split}\label{ch1/chI022-eqn14}
\end{gather}
Wystarczy zatem sprawdzić, czy rotacja pola sił  $\vec F$  jest 0.
\setbox0\vbox{
\begin{minipage}{0.95\linewidth}
\textbf{Zadanie}

\medskip


Sprawdzić, czy  siły $\vec F(x, y, z)$ o składowych
\phantomsection\label{ch1/chI022:equation-eqn15}\begin{gather}
\begin{split} 1.  \quad \quad F_1(x, y,z) = \frac{y}{x^2 + y^2 + z^2},  \quad F_2(x, y,z) = - \frac{x}{x^2 + y^2 + z^2},  \quad F_3(x, y,z) = \frac{z}{x^2 + y^2 + z^2}\end{split}\label{ch1/chI022-eqn15}
\end{gather}\phantomsection\label{ch1/chI022:equation-eqn16}\begin{gather}
\begin{split} 2.  \quad \quad F_1(x, y,z) = \frac{x-z}{x^2 + y^2 },  \quad F_2(x, y,z) = x e^{-y^2},  \quad F_3(x, y,z) = z+5\end{split}\label{ch1/chI022-eqn16}
\end{gather}\phantomsection\label{ch1/chI022:equation-eqn17}\begin{gather}
\begin{split}3. \quad \quad F_1(x, y,z) = 25 x^4 y - 3y^2,  \quad F_2(x, y,z) = 5x^5 -6xy -5,  \quad F_3(x, y,z) =0\end{split}\label{ch1/chI022-eqn17}
\end{gather}
są potencjalne
\end{minipage}}
\begin{center}\setlength{\fboxsep}{5pt}\shadowbox{\box0}\end{center}

Jeżeli układ jest potencjalny to łatwo sprawdzić, podobnie jak wyżej w przypadku układu o 1-stopniu swobody,  że istnieje stała ruchu - całkowita energia układu:
\phantomsection\label{ch1/chI022:equation-eqn18}\begin{gather}
\begin{split}E = \sum_i \frac{m\vec v^2}{2} + V(\vec r_1,  \vec r_2, \vec r_3, ..., \vec r_N)  = constant, \qquad \frac{dE}{dt} = 0\end{split}\label{ch1/chI022-eqn18}
\end{gather}
Dlatego  takie pole sił nazywa się zachowawczym polem sił.  Wszystkie siły związane z potencjalnym polem sił są siłami zachowawczymi. Istnieją jednak siły, które nie są siłami potencjalnymi, mimo to pozostają siłami zachowawczymi. Przykładem może być siła Lorentza działająca na naładowaną cząstkę poruszającą się w polu magnetycznym. Nie należy tego mylić z zachowawczymi układami dynamicznymi. Tę kwestię postaramy się teraz wyjaśnić.


\subsection{Dynamiczne układy zachowawcze i dysypatywne}
\label{ch1/chI022:id1}
W teorii układów dynamicznych ważną rolę pełnią dwa pojęcia: zachowawcze układy dynamiczne i dysypatywne układy dynamiczne.  Znowu dla jasności wywodu rozpatrzymy przykład układu o 3-wymiarowej przestrzeni fazowej:
\phantomsection\label{ch1/chI022:equation-eqn19}\begin{gather}
\begin{split}\dot x = F_1(x, y, z), \quad x(0) = x_0,\end{split}\label{ch1/chI022-eqn19}\\\begin{split}\dot y = F_2(x, y, z),  \quad x(0) = x_0,\end{split}\notag\\\begin{split}\dot z = F_3(x, y, z),  \quad x(0) = x_0.\end{split}\notag
\end{gather}
Wybieramy w przestrzeni fazowej obszar $D(0)$  o objętości $M(0)$. Zawiera on wszystkie możliwe warunki początkowe
\phantomsection\label{ch1/chI022:equation-eqn20}\begin{gather}
\begin{split}\{x_0, y_0, z_0\} \in D(0)\end{split}\label{ch1/chI022-eqn20}
\end{gather}
Pod wpływem ewolucji każdy punkt $(x_0, y_0, z_0)$  z tego obszaru przejdzie po czasie $t$ do punktu $(x(t),  y(t), z(t))$.  Zbiór tych punktów w chwili $t$ tworzy obszar  $D(t)$  o objętości $M(t)$. Zachodzi pytanie:
\phantomsection\label{ch1/chI022:equation-eqn21}\begin{gather}
\begin{split} \mbox{w jakich przypadkach} \quad M(t) = M(0)\end{split}\label{ch1/chI022-eqn21}
\end{gather}
Innymi słowy, kiedy układ dynamiczny zachowuje objętość fazową. Zbadamy ten problem. Wprowadzimy nowe oznaczenia, aby ułatwić notację:
\phantomsection\label{ch1/chI022:equation-eqn22}\begin{gather}
\begin{split}x_t = x(t), \quad \quad y_t = y(t), \quad \quad z(t) = z_t\end{split}\label{ch1/chI022-eqn22}
\end{gather}
Objętość fazowa warunków początkowych w chwili $t=0$ wynosi
\phantomsection\label{ch1/chI022:equation-eqn23}\begin{gather}
\begin{split}M(0) = \int \int \int_{D(0)}  dx_0 dy_0 dz_0\end{split}\label{ch1/chI022-eqn23}
\end{gather}
Objętość fazowa w chwili $t$ wynosi
\phantomsection\label{ch1/chI022:equation-eqn24}\begin{gather}
\begin{split}M(t) = \int \int \int_{D(t)}  dx_t dy_t dz_t\end{split}\label{ch1/chI022-eqn24}
\end{gather}
Ewolucja układu to nic innego jak zamiana zmiennych $(x_0, y_0, z_0) \to (x_t, y_t, z_t)$. Dokonajmy tej zamiany zmiennych w drugiej całce:
\phantomsection\label{ch1/chI022:equation-eqn25}\begin{gather}
\begin{split}M(t) = \int \int \int_{D(t)}  dx_t dy_t dz_t   =  \int \int \int_{D(0)}  \frac{\partial (x_t, y_t, z_t)}{\partial (x_0, y_0, z_0)} \; dx_0 dy_0 dz_0  = \int \int \int_{D(0)}   J(t)  dx_0 dy_0 dz_0 \qquad\end{split}\label{ch1/chI022-eqn25}
\end{gather}
gdzie $J$ jest jakobianem transformacji  $(x_t, y_t, z_t) \to (x_0, y_0, z_0)$. Jeżeli objętość fazowa nie zmienia się w czasie (jest funkcją stałą), to jej pochodna
\phantomsection\label{ch1/chI022:equation-eqn26}\begin{gather}
\begin{split}\frac{dM(t)}{dt} = \int \int \int_{D(0)}  \frac{ dJ(t)}{dt}  dx_0 dy_0 dz_0  \qquad\end{split}\label{ch1/chI022-eqn26}
\end{gather}
wynosi zero. Jeżeli
\phantomsection\label{ch1/chI022:equation-eqn27}\begin{gather}
\begin{split} \frac{ dJ(t)}{dt} = 0  \qquad \mbox{to} \qquad \frac{dM(t)}{dt} = 0 \qquad \mbox{czyli } \qquad M(t)=M(0)\end{split}\label{ch1/chI022-eqn27}
\end{gather}
Więc rozpoczynamy obliczenia
\phantomsection\label{ch1/chI022:equation-eqn28}\begin{gather}
\begin{split} \frac{ dJ(t)}{dt} = \frac{d}{dt} \; \frac{\partial (x_t, y_t, z_t)}{\partial (x_0, y_0, z_0)} = \frac{d}{dt}  \begin{bmatrix}\frac{ \partial x_t}{\partial x_0}& \frac{\partial x_t}{\partial y_0}&\frac{ \partial x_t}{\partial z_0}\\ \frac{ \partial y_t}{\partial x_0}&  \frac{ \partial y_t}{\partial y_0} &\frac{ \partial y_t}{\partial z_0} \\ \frac{ \partial z_t}{\partial x_0}& \frac{ \partial z_t}{\partial y_0}&\frac{ \partial z_t}{\partial z_0} \end{bmatrix}\end{split}\label{ch1/chI022-eqn28}
\end{gather}
Należy powyższy wyznacznik rozwinąć i pamiętać, że rozwiązania równań różniczkowych
\phantomsection\label{ch1/chI022:equation-eqn29}\begin{gather}
\begin{split}x_t = x_t(x_0, y_0, z_0), \qquad y_t = y_t(x_0, y_0, z_0), \qquad z_t = z_t(x_0, y_0, z_0)\end{split}\label{ch1/chI022-eqn29}
\end{gather}
zależą od warunków początkowych $\{x_0, y_0, z_0\}$.  Po rozwinięciu wyznacznika pojawiają się wyrażenia typu
\phantomsection\label{ch1/chI022:equation-eqn30}\begin{gather}
\begin{split}\frac{d}{dt}  \frac{ \partial x_t}{\partial z_0} = \frac{ \partial }{\partial z_0} \frac{dx_t}{dt} = \frac{ \partial }{\partial z_0} \dot x_t = \frac{ \partial }{\partial z_0} F_1(x_t, y_t, z_t) = \frac{ \partial F_1}{\partial x_t}  \frac{ \partial x_t}{\partial z_0} +  \frac{ \partial F_1}{\partial y_t}  \frac{ \partial y_t}{\partial z_0}  +\frac{ \partial F_1}{\partial y_t}  \frac{ \partial y_t}{\partial z_0}\end{split}\label{ch1/chI022-eqn30}
\end{gather}
Jak widać, w tym prostym przypadku musimy przeprowadzić uciążliwe rachunki. Znacznie lepiej jest posłużyć się rachunkiem symbolicznym z wykorzystaniem SAGE.

Aby przeprowadzić dowód, najlepiej jest obejść  ograniczenia operacji na wyrazeniach z pochodnymi w Sage.Pochodna wyznacznika jest zrobiona automatycznie, potem jest recznie wykonane podstawienie:
\phantomsection\label{ch1/chI022:equation-eqn31}\begin{gather}
\begin{split} \frac{ \partial }{\partial z_0} \dot x_t = \frac{ \partial F_1}{\partial x_t}  \frac{ \partial x_t}{\partial z_0} +  \frac{ \partial F_1}{\partial y_t}  \frac{ \partial y_t}{\partial z_0}  +\frac{ \partial F_1}{\partial y_t}  \frac{ \partial y_t}{\partial z_0}\end{split}\label{ch1/chI022-eqn31}
\end{gather}

\begin{verbatim}
for i,(v,lv) in enumerate(xy_wsp):
for j,(u,lu) in enumerate(xy_wsp):
     J[i,j] = var("d%sd%s"%(v,u),latex_name=r'\displaystyle\frac{\partial %s_t}{\partial %s_0}'%(lv,lu))
     var("dF%sd%s"%(v,u),latex_name=r'\displaystyle\frac{\partial F_%s}{\partial %s_t}'%(lv,lu))
#
to_fun = dict()
for v in J.list():
 vars()[str(v).capitalize()] = function(str(v).capitalize(),t)
 var("%sd"%str(v))
 to_fun[v]=vars()[str(v).capitalize()]
 to_fun[vars()[str(v)+"d"]]=vars()[str(v).capitalize()].diff()
to_var = dict((v,k) for k,v in to_fun.items())
#
to_rhs = dict()
for i,(v,lv) in enumerate(xy_wsp):
 for j,(u,lu) in enumerate(xy_wsp):
     to_rhs[vars()["d%sd%sd"%(v,u)]] = sum([vars()["dF%sd%s"%(v,w)]*vars()["d%sd%s"%(w,u)] for w,wl in xy_wsp])
print "Zaczynamy od macierzy Jacobiego:"
show(J)
print "Wszystkie pochodne cząstkowe są reprezentowane przez niezależne zmienne, aby policzyc pochodną wyznacznika, zamieniamy je podstawiając słownik zamieniający zmienne na funkcje:"
show(J.subs(to_fun))
print "Liczymy wyznaczniki pochodną, oraz wracamy do zmiennych symbolicznych:"
J.subs(to_fun).det().diff(t).subs(to_var).show()
#
print "Uzywając słownika to_rhs, podstawiamy prawe strony ODE:"
J.subs(to_fun).det().diff(t).subs(to_var).subs(to_rhs).show()
#
print "Ostatecznie dzielimy otrzymany wzór przez Jacobian:"
#
final = J.subs(to_fun).det().diff(t).subs(to_var).subs(to_rhs)/J.det()
final.simplify_full().show()
\end{verbatim}


Ostatecznie otrzymamy wyrażenie
\phantomsection\label{ch1/chI022:equation-eqn32}\begin{gather}
\begin{split}\frac{dJ(t)}{dt} = J(t) \left[\frac{\partial F_1}{\partial  x_t} + \frac{\partial  F_2}{\partial  y_t} + \frac{\partial F_3}{\partial z_t} \right]  = J(t)\; \mbox{ div} \vec F\end{split}\label{ch1/chI022-eqn32}
\end{gather}
To, co jest w nawiasie kwadratowym  nazywa się dywergencją pola wektorowego $\vec F$. Wstawiamy to wyrażemie do równania \eqref{ch1/chI022-eqn26} i otrzymamy
\phantomsection\label{ch1/chI022:equation-eqn33}\begin{gather}
\begin{split}\frac{dM(t)}{dt} = \int \int \int_{D(0)}  \frac{ dJ(t)}{dt}  dx_0 dy_0 dz_0   = \int \int \int_{D(0)} J(t)\; \mbox{ div} \vec F  dx_0 dy_0 dz_0  =  \int \int \int_{D(t)} \; \mbox{ div} \vec F  dx_t dy_t dz_t \qquad\end{split}\label{ch1/chI022-eqn33}
\end{gather}
gdzie dokonaliśmy odwrotnego przejścia  ( z prawej strony na lewą stronę) jak  w równaniu \eqref{ch1/chI022-eqn25}.

Można teraz uogólnić ten wynik na dowolną ilość wymiarów przestrzeni fazowej  dla układu równań
\phantomsection\label{ch1/chI022:equation-eqn34}\begin{gather}
\begin{split}\frac{d\vec x}{dt} = \vec F (\vec x), \quad \quad \vec x = [x_1, x_2, x_3, ...., x_n], \quad \quad \vec F = [F_1, F_2, F_3, ..., F_n]\end{split}\label{ch1/chI022-eqn34}
\end{gather}
i otrzymamy
\begin{description}
\item[{Twierdzenie}] \leavevmode
Jeżeli dywergencja pola wektorowego $\vec F$  danego równania różniczkowego jest zero,

\end{description}
\phantomsection\label{ch1/chI022:equation-eqn35}\begin{gather}
\begin{split}\mbox{ div} \vec F = \sum_i \frac{\partial F_i}{\partial x_i} = 0\end{split}\label{ch1/chI022-eqn35}
\end{gather}
wówczas objętość fazowa jest zachowana, $M(t) = M(0)$. Takie układy dynamiczne nazywamy zachowawczymi. Jeżeli objętość fazowa maleje w czasie, to układ nazywamy dysypatywnym. Innymi słowy, układ jest dysypatywny gdy objętość $M(t) < M(0)$ dla $t>0$. Oznacza to, że dla układów dysypatywnych
\phantomsection\label{ch1/chI022:equation-eqn36}\begin{gather}
\begin{split}\frac{dM(t)}{dt} < 0\end{split}\label{ch1/chI022-eqn36}
\end{gather}
Gdyby
\phantomsection\label{ch1/chI022:equation-eqn37}\begin{gather}
\begin{split}\mbox{ div} \vec F = C_0 = const.\end{split}\label{ch1/chI022-eqn37}
\end{gather}
wówczas z równania \eqref{ch1/chI022-eqn33} otrzymujemy prostą relację
\phantomsection\label{ch1/chI022:equation-eqn38}\begin{gather}
\begin{split}\frac{dM(t)}{dt} = C_0 M(t)\end{split}\label{ch1/chI022-eqn38}
\end{gather}
która pozwala rozstrzygnąć czy układ jest dysypatywny.


\subsubsection{Przykład 1: Oscylator harmoniczny tłumiony}
\label{ch1/chI022:przyklad-1-oscylator-harmoniczny-tlumiony}\phantomsection\label{ch1/chI022:equation-eqn39}\begin{gather}
\begin{split}\dot x = y = F_1(x, y), \quad \quad x(0) = x_0,\end{split}\label{ch1/chI022-eqn39}\\\begin{split}\dot y = -\gamma y -\omega^2 x = F_2(x, y), \quad \quad y(0) = y_0.\end{split}\notag
\end{gather}
Łatwo obliczyć dywergencję pola
\phantomsection\label{ch1/chI022:equation-eqn40}\begin{gather}
\begin{split}\mbox{ div} \vec F =  \frac{\partial F_1}{\partial x} + \frac{\partial F_2}{\partial y} = -\gamma <0\end{split}\label{ch1/chI022-eqn40}
\end{gather}
Równanie \eqref{ch1/chI022-eqn37} przyjmuje postać
\phantomsection\label{ch1/chI022:equation-eqn41}\begin{gather}
\begin{split}\frac{dM(t)}{dt} = -\gamma  M(t),  \qquad \mbox{ jego rozwiązaniem jest funkcja malejąca } \qquad M(t) = M(0) e^{-\gamma t}\end{split}\label{ch1/chI022-eqn41}
\end{gather}
czyli  objętość fazowa (w tym przypadku powierzchnia fazowa) maleje  w czasie i dlatego jest to dysypatywny układ dynamiczny.


\subsubsection{Przykład 2: Model Lorenza}
\label{ch1/chI022:przyklad-2-model-lorenza}\phantomsection\label{ch1/chI022:equation-eqn42}\begin{gather}
\begin{split}\dot x = \sigma (y-x) = F_1(x, y,  z), \quad \quad x(0) = x_0,\end{split}\label{ch1/chI022-eqn42}\\\begin{split}\dot y = x(\rho - z) -y = F_2(x, y,  z),  \quad \quad y(0) = y_0,\end{split}\notag\\\begin{split}\dot z = x y - \beta z = F_3(x, y,  z), , \quad \quad z(0) = z_0.\end{split}\notag
\end{gather}
gdzie wszystkie parametry są dodatnie: $\sigma, \rho, \beta > 0$.

Obliczymy  dywergencję 3-wymiarowego pola $\vec F = [F_1, F_2, F_3]$. Proste rachunki pokazują, że
\phantomsection\label{ch1/chI022:equation-eqn43}\begin{gather}
\begin{split}\mbox{ div} \vec F =  \frac{\partial F_1}{\partial x} + \frac{\partial F_2}{\partial y}   + \frac{\partial F_3}{\partial z}  = -\sigma -1 - \beta <0\end{split}\label{ch1/chI022-eqn43}
\end{gather}
Objętość fazowa (w tym przypadku faktycznie objętość w 3 wymiarowej przestrzeni) maleje  eksponencjalnie w czasie, podobnie jak w poprzednim przykładzie.  Dlatego też jest to dysypatywny układ dynamiczny.


\section{Stany stacjonarne i ich stabilność}
\label{ch1/chI023:stany-stacjonarne-i-ich-stabilnosc}\label{ch1/chI023::doc}
Czasami rozwiązanie równań różniczkowych dąży do stałej wartości dla długich czasów, $t\to \infty$. Mówimy wówczas, że istnieje rozwiązanie stacjonarne (stan stacjonarny, punkt stały, punkt równowagi):
\phantomsection\label{ch1/chI023:equation-eqn1}\begin{gather}
\begin{split}\lim_{t \to \infty}  \vec x(t) = \vec x_s\end{split}\label{ch1/chI023-eqn1}
\end{gather}
Może istnieć kilka stanów stacjonarnych, a nawet nieskończenie wiele stanów stacjonarnych. Który z tych stanów się realizuje,  zależy to od warunku początkowego $\vec x(0) = \vec x_0$. Rozwiązanie stacjonarne $\vec x_s$ nie zależy od czasu. Nazwa ``rozwiązanie stacjonarne'' nie jest bezpodstawne. Faktycznie jest to rozwiązanie układu równań  różniczkowych
\phantomsection\label{ch1/chI023:equation-eqn2}\begin{gather}
\begin{split}\frac{d\vec x}{dt} = \vec F(\vec x), \quad \quad \vec x(0)  = \vec x_0\end{split}\label{ch1/chI023-eqn2}
\end{gather}
z warunkiem początkowym $\vec x(0) = \vec x_s$. Jeżeli  $\vec x(t) = \vec x_s$ jest rozwiązaniem stacjonarnym to musi spełniać układ \eqref{ch1/chI023-eqn2}., czyli
\phantomsection\label{ch1/chI023:equation-eqn3}\begin{gather}
\begin{split}\mbox{jeżeli}  \quad \quad \vec x(t) = \vec x_s \quad \quad\mbox{to} \quad \quad\frac{d\vec x(t)}{dt} = \frac{d\vec x_s}{dt}=0 = \vec F(\vec x(t)) = \vec F(\vec x_s)) = 0\end{split}\label{ch1/chI023-eqn3}
\end{gather}
Innymi słowy, stan stacjonarny wyznaczny jest z warunku:
\phantomsection\label{ch1/chI023:equation-eqn4}\begin{gather}
\begin{split}\vec F(\vec x_s)) = 0\end{split}\label{ch1/chI023-eqn4}
\end{gather}
Powyższy warunek to układ $n$-równań algebraicznych. Zwykle udaje  nam się go rozwiązać analitycznie w niewielu przypadkach,  w szczególności gdy wymiar przestrzeni fazowej  $n >1$.  Natomiast możemy taki układ rozwiązywać numerycznie. Jeżeli już wyznaczymy stan stacjonarny, to nasuwa się pytanie na ile jest on stabily, tzn. gdy nieco wytrącimy układ z tego stanu to czy powróci on do poprzedniego stanu stacjonarnego czy też oddali się od niego.  Na przykład stanem stacjonarnym kulki poruszającej się na nitce w polu ziemskim jest położenie pionowe. Gdy kulkę wychylimy z tego położenia, po dostatecznie długim czasie powróci ona do pozycji pionowej i tak tam pozostanie nieruchoma. Jest to stabilny stan równowagi. Rozważmy teraz kulkę mogącą poruszać się tylko po sferze. Gdy umieścimy kulkę na  biegunie północnym sfery w polu ziemskim to nieskończenie małe zaburzenie spowoduje, że kulka spadnie z tego położenie i nigdy do niego nie powróci. W obu tych przypadkach zakładamy  rzeczywiste warunki ruchu z tarciem. Pominięcie tarcia spowoduje radykalnie różne zachowanie. Te dwa przyklady pozwalają nabyć intuicję, co to znaczy że stan stacjonarny jest stabilny lub jest niestabilny.  Z grubsza można powiedzieć, że stan stacjonarny  $\vec x_s$ jest stabilny jeśli każda trajektoria startującego z punktu bliskiego $\vec x_s$ pozostaje blisko $\vec x_s$ wraz z upływem czasu. Natomiast  $\vec x_s$ jest niestabilny gdy  każda każda trajektoria startującego z punktu bliskiego $\vec x_s$ oddala się od tego punktu gdy $t\to \infty$.  Można podać bardziej precyzyjne definicje.
\begin{description}
\item[{Definicja}] \leavevmode
Mówimy, że stan stacjonarny  $\vec x_s$ jest stabilny jeżeli  dla dowolnego $\epsilon >0$ istnieje
takie $\delta(\epsilon) >0$, że dla każdego $\vec x_0$ takiego że $| \vec x_0 -\vec x_s| < \delta$
rozwiązanie $\vec x(t)$ spełnia nierówność: $|\vec x(t) - \vec x_s| < \epsilon$ dla dowolnych czasów $t>0$.
Jeżeli dodatkowo $\lim_{t\to \infty} \vec x(t)  = \vec x_s$ to stan stacjonarny $\vec x_s$ jest asymptotycznie stabilny.

\end{description}

Innymi słowy dla stabilnych stanów rozwiązanie $\vec x(t)$ jest cały czas blisko rozwiązania stacjonarnego $\vec x_s$, a dla asymptotycznie stabilnych stanów rozwiązanie $\vec x(t)$ dąży do $\vec x_s$  gdy czas $t\to \infty$.


\subsection{Przypadek A: Jedno równanie różniczkowe}
\label{ch1/chI023:przypadek-a-jedno-rownanie-rozniczkowe}

\subsubsection{Przykład: Równanie różniczkowe liniowe}
\label{ch1/chI023:przyklad-rownanie-rozniczkowe-liniowe}\phantomsection\label{ch1/chI023:equation-eqn5}\begin{gather}
\begin{split}\dot x = \lambda x = f(x), \qquad \lambda  \in R\end{split}\label{ch1/chI023-eqn5}
\end{gather}
Stan stacjonarny znajdujemy jako rozwiazanie równania
\phantomsection\label{ch1/chI023:equation-eqn6}\begin{gather}
\begin{split}f(x_s) = \lambda x_s =0 \qquad \mbox{stąd otrzymujemy stan stacjonarny} \qquad x_s = 0\end{split}\label{ch1/chI023-eqn6}
\end{gather}
Pytamy, czy ten stan jest stabilny. Musimy zaburzyć rozwiązanie stacjonarne $x(t) = x_s = 0$ i wystartować z dostatecznie bliskiego w stosunku do $x_s=0$ warunku początkowego $X_0$. Rozwiązaniem równania różniczkowego liniowego jest funkcja
\phantomsection\label{ch1/chI023:equation-eqn7}\begin{gather}
\begin{split}x(t) = X_0 e^{\lambda t}\end{split}\label{ch1/chI023-eqn7}
\end{gather}
Jeżeli
\phantomsection\label{ch1/chI023:equation-eqn8}\begin{gather}
\begin{split}\lambda  < 0  \quad \quad \mbox{ to} \quad \quad x(t) \to 0 \quad \quad \mbox{dla wszystkich}  \;   X_0  \;  \mbox{bliskich} \; 0\end{split}\label{ch1/chI023-eqn8}
\end{gather}
Wówczas stan stacjonarny $x_s=0$ jest stabilny i dodatkowo jest asymptotycznie stabilny ponieważ rozwiązanie to dąży do $x_s =0$  gdy  $t\to \infty$ .

Jeżeli
\phantomsection\label{ch1/chI023:equation-eqn9}\begin{gather}
\begin{split}\lambda >0  \quad \quad \mbox{ to} \quad \quad x(t) \to \infty  \quad \quad \mbox{dla wszystkich}  \;   X_0  \;  \mbox{bliskich} \; 0\end{split}\label{ch1/chI023-eqn9}
\end{gather}
Wówczas stan stacjonarny $x_s=0$ jest niestabilny.


\begin{verbatim}
var('x')
pa1=plot(0.005*exp(-x), (x,0,1), figsize=(6,3),color="red")
pa2=plot(0.005*exp(x), (x,0, 1), figsize=(6,3))
pa3=plot(-0.005*exp(-x), (x,0,1), figsize=(6,3),color="red")
pa4=plot(-0.005*exp(x), (x,0, 1), figsize=(6,3))
show(pa1+pa2+pa3+pa4)
\end{verbatim}


Na rysunku przedstawiono zagadnienie stabilności:  czerwone krzywe dążą do 0 gdy $t\to \infty$. Niebieskie  krzywe uciekają od   0 gdy $t\to \infty$.

Warunki początkowe $x(0)=\pm 0.05$ są blisko stanu stacjonarnego $x_s=0$. Przykład ten sugeruje nam metodę badania stabilności stanu stacjonarnego. Teraz podamy tę metodę.


\subsection{Metoda linearyzacji badania stabilności}
\label{ch1/chI023:metoda-linearyzacji-badania-stabilnosci}
Rozpatrujemy układ 1-wymiarowy:
\phantomsection\label{ch1/chI023:equation-eqn10}\begin{gather}
\begin{split}\frac{dx}{dt} = \dot x = f(x), \quad \quad f(x_s) = 0\end{split}\label{ch1/chI023-eqn10}
\end{gather}
Aby zbadać stabilność stanu  $x_s$, analizujemy zaburzenie
\phantomsection\label{ch1/chI023:equation-eqn11}\begin{gather}
\begin{split}h(t) = x(t) - x_s,  \quad \quad  |h(0)| = |x(0) - x_s| < \delta\end{split}\label{ch1/chI023-eqn11}
\end{gather}
Funkcja $h(t)$ powinna być mała, gdy stan $x_s$ jest stabilny. Jak daleko jest rozwiązanie $x(t)$ od rozwiązania $x_s$. Zobaczmy, jakie równanie różniczkowe spełnia $h(t)$:
\phantomsection\label{ch1/chI023:equation-eqn12}\begin{gather}
\begin{split}\frac{dh}{dt} = \frac{d}{dt} [x(t) - x_s] = \frac{dx}{dt} = f(x) = f( x_s +h)  = f(x_s) + f'(x_s) h + \frac{1}{2} f''(x_s) h^2 + \frac{1}{3!} f'''(x_s) h^3+ ....\end{split}\label{ch1/chI023-eqn12}
\end{gather}
Ponieważ $f(x_s)=0$, otrzymujemy równanie różniczkowe dla odchylenia $h(t)$ od stanu stacjonarnego
\phantomsection\label{ch1/chI023:equation-eqn13}\begin{gather}
\begin{split}\frac{dh}{dt} =  f'(x_s) h + \frac{1}{2} f''(x_s) h^2 + \frac{1}{3!} f'''(x_s) h^3+ ....\end{split}\label{ch1/chI023-eqn13}
\end{gather}
Jeżeli :math:{}` f'(x\_s) ne 0{}`, to pierwszy istotny wyraz w tym równaniu jest liniowy względem $h$.  Wyrazy $h^2,  h^3$ i wyższych potęg są pomijalnie małe. Jeżeli np. $h =10^{-2}$ to  $h^2 = 10^{-4},  h^3 = 10^{-6}$. Wówczas $h^2,  h^3$ i wyższe potęgi $h$  można pominąć jako bardzo małe.  Otrzymujemy równanie różniczkowe liniowe
\phantomsection\label{ch1/chI023:equation-eqn14}\begin{gather}
\begin{split}\frac{dh}{dt} =\lambda h, \quad \quad \lambda = f'(x_s)\end{split}\label{ch1/chI023-eqn14}
\end{gather}
z rozwiązaniem
\phantomsection\label{ch1/chI023:equation-eqn15}\begin{gather}
\begin{split}h(t) = h(0)  e^{\lambda t}\end{split}\label{ch1/chI023-eqn15}
\end{gather}
Wiemy już z powyższego przykładu, że gdy $\lambda < 0$ to $h(t) \to 0$ gdy $t\to \infty$. To oznacza, że  zaburzenie $x(t) \to x_s$ gdy  $t \to \infty$  i wówczas stan stacjonarny $x_s$ jest stabilny.  Jeżeli  $\lambda > 0$ to $|h(t)| \to \infty$ gdy $t\to \infty$. To oznacza, że  zaburzenie x(t) ucieka od $x_s$ gdy  $t \to \infty$  i wówczas stan stacjonarny $x_s$ jest niestabilny. Otrzymujemy następujące kryterium na stabilność stanu stacjonarnego:

Jeżeli $\lambda = f'(x_s) < 0$ to  stan stacjonarny jest stabilny; jeżeli $\lambda = f'(x_s) > 0$ to  stan stacjonarny jest niestabilny.

Jeżeli $\lambda = f'(x_s) = 0$ to  nie wiem nic na temat stabilności. Musimy badać następne niezerowe wyrazy.  Jeżeli $f''(x_s) \ne 0$ zatrzymujemy pierwszy nieznikający wyraz czyli  badamy równanie
\phantomsection\label{ch1/chI023:equation-eqn16}\begin{gather}
\begin{split}\frac{dh}{dt} =\gamma h^2, \quad \quad \gamma  =  \frac{1}{2}f''(x_s)\end{split}\label{ch1/chI023-eqn16}
\end{gather}
Jeżeli  $f'(x_s) =0$ oraz $f''(x_s) =0$ to  bierzemy następny nieznikający wyraz i badamy równanie
\phantomsection\label{ch1/chI023:equation-eqn17}\begin{gather}
\begin{split} \frac{dh}{dt} =\nu h^3, \quad \quad \nu  =    \frac{1}{3!} f'''(x_s)\end{split}\label{ch1/chI023-eqn17}
\end{gather}
Jeżeli w tych przypadkach $h(t) \to 0$ gdy $t\to \infty$, to  stan stacjonarny $x_s$ jest stabilny. W przeciwnym przypadku  - nie jest stabilny.


\subsection{Metoda potencjału badania stabilności}
\label{ch1/chI023:metoda-potencjalu-badania-stabilnosci}
W jednym wymiarze, równanie różniczkowe  zawsze można przedstawić w postaci
\phantomsection\label{ch1/chI023:equation-eqn18}\begin{gather}
\begin{split}\frac{dx}{dt} = \dot x = f(x) = -\frac{dV(x)}{dx} = -V'(x), \quad \quad f(x_s) = 0\end{split}\label{ch1/chI023-eqn18}
\end{gather}
gdzie funkcja
\phantomsection\label{ch1/chI023:equation-eqn19}\begin{gather}
\begin{split}V(x) = -\int f(x)  dx\end{split}\label{ch1/chI023-eqn19}
\end{gather}
nazywana jest ``potencjałem''.  W ogólności to nie jest potencjał fizyczny który pojawia się w równaniu Newtona  dla cząstki z tłumieniem:
\phantomsection\label{ch1/chI023:equation-eqn20}\begin{gather}
\begin{split}m \ddot x + \gamma \dot x = -V'(x)\end{split}\label{ch1/chI023-eqn20}
\end{gather}
Jeżeli ruch cząstki odbywa się w środowisku o dużym  tarciu, w tzw. reżimie przetłumionym, w którym przyśpieszenie cząstki jest znikomo małe (formalnie gdy $m=0$), wówczas równanie Newtona ma postać
\phantomsection\label{ch1/chI023:equation-eqn21}\begin{gather}
\begin{split}\gamma \dot x = -V'(x)\end{split}\label{ch1/chI023-eqn21}
\end{gather}
które po przeskalowaniu ma postać:
\phantomsection\label{ch1/chI023:equation-eqn22}\begin{gather}
\begin{split}\dot x = -\frac{1}{\gamma} V'(x) = - {\tilde V} '(x)\end{split}\label{ch1/chI023-eqn22}
\end{gather}
Stąd też ``historycznie'' wywodzi sią nazwa potencjał dla abstrakcyjnego układu dynamicznego:
\phantomsection\label{ch1/chI023:equation-eqn23}\begin{gather}
\begin{split} \dot x = f(x) =  = -V'(x)\end{split}\label{ch1/chI023-eqn23}
\end{gather}
Stan stacjonarny $x_s$ określony równaniem
\phantomsection\label{ch1/chI023:equation-eqn24}\begin{gather}
\begin{split}f(x_s) = -V'(x_s) = 0\end{split}\label{ch1/chI023-eqn24}
\end{gather}
to punkt ekstremalny potencjału (o ile  pochodna parzystego rzędu $V^{(2k)}(x_s) \ne 0$).  Punkt $x_s$ jest stabilny gdy
\phantomsection\label{ch1/chI023:equation-eqn25}\begin{gather}
\begin{split}\lambda = f'(x_s) = - V''(x_s) < 0\end{split}\label{ch1/chI023-eqn25}
\end{gather}
czyli $V''(x_s) > 0$. To odpowiada minimum potencjału. Natomiast punkt $x_s$  niestabily odpowiada maksimum potencjału. Mamy więc klarowy obraz: Rysujemy potencjał $V(x)$. Minima potencjału to stabilne stany stacjonarne; maksima potencjału to niestabilne stany stacjonarne.
\setbox0\vbox{
\begin{minipage}{0.95\linewidth}
\textbf{Zadania}

\medskip


Wyznacz stany stacjonarne i zbadaj ich (asymptotyczną) stabilność. Korzystaj z metody linearyzacji i metody potencjału.
\begin{enumerate}
\item {} 
$\dot x = 4 x - x^3$

\item {} 
$\dot x = 1+x^4$

\item {} 
$\dot x =3 \sin(x)$

\item {} 
$\dot x =2x \sin(x)$

\item {} 
$\dot x =-x^2 \sin(x)$

\end{enumerate}
\end{minipage}}
\begin{center}\setlength{\fboxsep}{5pt}\shadowbox{\box0}\end{center}

Poniżej pokazujemy wyniki dla zadania 1. Są 3 stany stacjonarne: $x_{s1} = 2,  x_{s2} = 0,  x_{s3} = -2$. Stany  $x_{s1} = 2$  oraz  $x_{s3} = -2$  są asymptotycznie stabilne (rozwiązania dążą do tych stanów). Stan $x_{s2} = 0$  jest niestabilny (rozwiązania uciekają od tego stanu).


\begin{verbatim}
var('x,y,z,u,Z,Y,t')
T0 = srange(0,1.5,0.01)
f11=4*x-x^3
f12=4*y-y^3
f13=4*z-z^3
f15=4*u-u^3
f16=0
sol5=desolve_odeint( vector([f11, f12, f13, 0, 0,f15]), [4,0.2,-0.2,2,-2,-4],T0,[x,y,z,Z,Y,u])
line( zip ( T0,sol5[:,0]) ,figsize=(7, 4)) +\
... line( zip ( T0,sol5[:,1]) ,color='red')+\
... line( zip ( T0,sol5[:,2]) ,color='green') +\
... line( zip ( T0,sol5[:,4]) ,color='gray') +\
... line( zip ( T0,sol5[:,5]) ,color='black')+\
... line( zip ( T0,sol5[:,3]) ,color='violet')
\end{verbatim}


Z powyższego przykładu zauważmy następujące cechy:
\begin{enumerate}
\item {} 
Zbiór warunków początkowych dzieli się na dwa podzbiory $A_1 = (-\infty, 0)$ oraz  $A_2=(0, \infty)$. Dla warunków  początkowych ze zbioru $A_1$ rozwiązania $x(t) \to -2$ dla $t\to\infty$, a  dla warunków  początkowych ze zbioru $A_2$ rozwiązania $x(t) \to 2$ dla $t\to\infty$.

\item {} 
Krzywe $x(t)$  są funkcjami monotonicznymi: albo cały czas rosną w czasie, albo cały czas maleją gdy czas rośnie. Dlaczego? Rozważmy  2 przykłady warunków początkowych widocznych na rysunku:

\end{enumerate}
\begin{enumerate}
\item {} 
gdy $x(0) = 4$ to $(dx/dt) (x=4) = 4*4 -4^3 = -48 < 0$.   Pochodna jest ujemna, a to oznacza że funkcja maleje. Podobnie jest dla wszystkich warunków początkowych $x(0) > 2$

\item {} 
gdy $x(0) = 0.2$ to $(dx/dt) (x=0.2) = 4*0.2 -(0.2)^3 > 0$.  Pochodna jest dodatnia, a to oznacza że funkcja rośnie. Podobnie jest dla wszystkich warunków początkowych $x(0)  \in (0, 2)$

\end{enumerate}
\begin{enumerate}
\setcounter{enumi}{2}
\item {} 
Różne krzywe $x(t)$ nigdy się nie przecinają. Wynika to z tego, że rozwiązania są jedyne i jednoznaczne. Jeżeli by się przecinały,  to z punktu przecięcia wychodziłoby kilka rozwiązań. A to jest niemożliwe na mocy twierdzenia o jednoznaczności rozwiązań.

\end{enumerate}


\subsection{Przypadek B: Układ  2 równań różniczkowych}
\label{ch1/chI023:przypadek-b-uklad-2-rownan-rozniczkowych}
Dla jasności prezentacji, rozpatrujemy układ 2 równań różniczkowych
\phantomsection\label{ch1/chI023:equation-eqn26}\begin{gather}
\begin{split}\dot x = f(x, y),\end{split}\label{ch1/chI023-eqn26}\\\begin{split}\dot y = g(x, y).\end{split}\notag
\end{gather}
Stany stacjonarne $(x_s, y_s)$  określone są jako rozwiązania układu równań algebraicznych
\phantomsection\label{ch1/chI023:equation-eqn27}\begin{gather}
\begin{split}f(x_s, y_s)=0,\end{split}\label{ch1/chI023-eqn27}\\\begin{split}g(x_s, y_s) =0.\end{split}\notag
\end{gather}
Gdyby istniał potencjał $V(x, y)$, powyżej przedstawione wnioski otrzymane z metody potencjału są  także prawdziwe dla układów wielowymiarowyxch. Ponieważ w ogólnym  przypadku nie musi istnieć ``potencjał'',  zbadamy  stabilność stanów   $(x_s, y_s)$  stosując metodę linearyzacji. Definiujemy  nowe funkcje
\phantomsection\label{ch1/chI023:equation-eqn28}\begin{gather}
\begin{split}X(t) = x(t) - x_s,\end{split}\label{ch1/chI023-eqn28}\\\begin{split}Y(t)=y(t)-y_s.\end{split}\notag
\end{gather}
Charakteryzują one odchylenie od stanu stacjonarnego, które są małe gdy stan stacjonarny jest stabilny. Zobaczymy, jakie równania różniczkowe spełniają te funkcje:
\phantomsection\label{ch1/chI023:equation-eqn29}\begin{gather}
\begin{split}\dot X(t) = \dot x(t) - \dot x_s  = \dot x(t) =  f(x(t), y(t)) = f(x_s +X(t), y_s + Y(t))  = f(x_s, y_s) + \frac{\partial f}{\partial x}  X + \frac{\partial f}{\partial y} Y+ ...\end{split}\label{ch1/chI023-eqn29}
\end{gather}\phantomsection\label{ch1/chI023:equation-eqn30}\begin{gather}
\begin{split} \dot Y(t)= \dot y(t) - \dot y_s  = \dot y(t) =  g(x(t), y(t)) =g(x_s +X(t), y_s + Y(t))  = gf(x_s, y_s) + \frac{\partial g}{\partial x}  X + \frac{\partial g}{\partial y} Y + ...\end{split}\label{ch1/chI023-eqn30}
\end{gather}
Wszystkie pochodne cząstkowe są obliczane w punkcie $(x_s, y_s)$. Ponieważ w punkcie stacjonarnym $f(x_s, y_s)=g(x_s, y_s)=0$ otrzymujemy zlinearyzowane równania różniczkowe w postaci
\phantomsection\label{ch1/chI023:equation-eqn31}\begin{gather}
\begin{split}\dot X =  a_{11} X +  a_{12} Y\end{split}\label{ch1/chI023-eqn31}
\end{gather}\phantomsection\label{ch1/chI023:equation-eqn32}\begin{gather}
\begin{split} \dot Y=  a_{21} X +  a_{22} Y\end{split}\label{ch1/chI023-eqn32}
\end{gather}
gdzie macierz współczynników
\phantomsection\label{ch1/chI023:equation-eqn33}\begin{gather}
\begin{split} \qquad \quad J = \begin{bmatrix}\frac{ \partial f}{\partial x}& \frac{\partial f}{\partial y}\\ \frac{\partial g}{\partial x}& \frac{\partial g}{\partial y}  \end{bmatrix}  =   \begin{bmatrix}a_{11} &  a_{12} \\ a_{21}& a_{22}  \end{bmatrix}   \quad \quad \mbox{obliczona w punkcie} \quad (x_s, y_s)\end{split}\label{ch1/chI023-eqn33}
\end{gather}
jest macierzą Jacobiego. Ponieważ otrzymaliśmy liniowy układ równań różniczkowych,  jego rozwiązanie poszukujemy w postaci funkcji
\phantomsection\label{ch1/chI023:equation-eqn34}\begin{gather}
\begin{split}X(t) = A e^{\lambda t}, \quad \quad  Y(t) = B e^{\lambda t}\end{split}\label{ch1/chI023-eqn34}
\end{gather}
gdzie stałe $A$ oraz $B$ są określone przez warunki początkowe $(X(0), Y(0))$.

Zauważamy, że
\phantomsection\label{ch1/chI023:equation-eqn35}\begin{gather}
\begin{split} \dot X = \lambda X, \quad \quad \dot Y = \lambda  Y\end{split}\label{ch1/chI023-eqn35}
\end{gather}
Wstawiamy te relacje do  układu równań zlinearyzowanych:
\phantomsection\label{ch1/chI023:equation-eqn36}\begin{gather}
\begin{split}\lambda A = a_{11} A +  a_{12} B\end{split}\label{ch1/chI023-eqn36}
\end{gather}\phantomsection\label{ch1/chI023:equation-eqn37}\begin{gather}
\begin{split} \lambda  Y=  a_{21} A +  a_{22} B\end{split}\label{ch1/chI023-eqn37}
\end{gather}
Jest to zagadnienie własne dla macierzy Jacobiego, gdzie $\lambda$ to są wartości własne macierzy Jacobiego. To   jest także  równoważne liniowemu układowi równań algebraicznych
\phantomsection\label{ch1/chI023:equation-eqn38}\begin{gather}
\begin{split}\ (a_{11} - \lambda) A +  a_{12} B  = 0\end{split}\label{ch1/chI023-eqn38}
\end{gather}\phantomsection\label{ch1/chI023:equation-eqn39}\begin{gather}
\begin{split}a_{21} A +  (a_{22} -\lambda) B  = 0\end{split}\label{ch1/chI023-eqn39}
\end{gather}
Taki układ ma niezerowe rozwiązanie dla $A$ oraz $B$ gdy wyznacznik
\phantomsection\label{ch1/chI023:equation-eqn40}\begin{gather}
\begin{split}\mbox{Det}  |J-\lambda I| = \mbox{Det} \begin{bmatrix}a_{11} -\lambda &  a_{12} \\ a_{21}& a_{22} - \lambda  \end{bmatrix} = (a_{11} -\lambda) ( a_{22} -\lambda) -a_{12} a_{21} = \lambda^2  - (a_{11} +a_{22} ) \lambda +a_{11} a_{22} -a_{12} a_{21} = 0\end{split}\label{ch1/chI023-eqn40}
\end{gather}
Macierz $I$ jest macierzą jednostkową, tzn. diagonalną $2\times 2$ i wszystkie elementy diagonalne to 1. Z powyższej relacji  otrzymujemy równanie kwadratowe dla nieznanych wartości  własnych $\lambda = \lambda_1$  oraz $\lambda = \lambda_2$.

Rozwiązanie zlinearyzowanego układu jest w postaci kombinacji liniowej :
\phantomsection\label{ch1/chI023:equation-eqn41}\begin{gather}
\begin{split}X(t) = A_1  e^{\lambda_1 t} + A_2 e^{\lambda_2 t}, \quad \quad  Y(t) = B_1 e^{\lambda_1 t} +  B_2 e^{\lambda_2 t}\end{split}\label{ch1/chI023-eqn41}
\end{gather}
Pytanie o stabilność stanu stacjonarnego $(x_s, y_s)$ jest równoważne pytaniu: kiedy
\phantomsection\label{ch1/chI023:equation-eqn42}\begin{gather}
\begin{split}\lim_{t\to \infty} X(t) = 0  \quad \quad \lim_{t\to \infty} Y(t) = 0\end{split}\label{ch1/chI023-eqn42}
\end{gather}
Funkcje exponencjalne dążą do zera jeżeli część rzeczywista  wartości własnych macierzy Jacobiego $\lambda_i$ są  ujemne:
\phantomsection\label{ch1/chI023:equation-eqn43}\begin{gather}
\begin{split} Re(\lambda_1) < 0, \quad \quad Re(\lambda_2) < 0\end{split}\label{ch1/chI023-eqn43}
\end{gather}
Wówczas stan stacjonarny jest asymptotycznie stabilny. Jeżeli
\phantomsection\label{ch1/chI023:equation-eqn44}\begin{gather}
\begin{split} Re(\lambda_1) > 0, \quad \quad \mbox{i/lub} \quad \quad Re(\lambda_2) >  0\end{split}\label{ch1/chI023-eqn44}
\end{gather}
to stan stacjonarny jest niestabilny. Jeżeli
\phantomsection\label{ch1/chI023:equation-eqn45}\begin{gather}
\begin{split} Re(\lambda_1) = Re(\lambda_2) =  0\end{split}\label{ch1/chI023-eqn45}
\end{gather}
to stan stacjonarny jest stabilny, ale nie  jest asymptotycznie stabilny. Jeżeli wartości własne macierzy Jacobiego wynoszą zero, metoda linearyzacji nie rozstrzyga o stabilności.

Zamiast wyznaczyć wartości własne $(\lambda_{1}, \lambda_{2})$ tej macierzy, wystarczy sprawdzić, kiedy część rzeczywista wartości własnych jest ujemna (lub dodatnia).  Ponieważ macierz Jacobiego jest macierzą $2 \times 2$, więc otrzymujemy równanie kwadratowe  dla $\lambda$. Aby wartości własne miały część rzeczywistą ujemną muszą zachodzić dwie relacje:
\phantomsection\label{ch1/chI023:equation-eqn46}\begin{gather}
\begin{split}\lambda_1 + \lambda_2  = a_{11}  + a_{22} <0  \quad \mbox{oraz} \quad \lambda_1 \; \lambda_2 = a_{11} a_{22}  -a_{12}a_{21} > 0,\end{split}\label{ch1/chI023-eqn46}
\end{gather}
to oznacza że
\phantomsection\label{ch1/chI023:equation-eqn46a}\begin{gather}
\begin{split}\mbox{Tr} \, J < 0, \quad \quad \mbox{det} \,J > 0\end{split}\label{ch1/chI023-eqn46a}
\end{gather}
gdzie Tr oznacza ślad macierzy, czyli sumę elementów diagonalnych macierzy oraz Det jest wyznacznikiem macierzy.


\subsubsection{Przykład 1}
\label{ch1/chI023:przyklad-1}\phantomsection\label{ch1/chI023:equation-eqn47}\begin{gather}
\begin{split}\dot x= 1-xy = f(x, y),\end{split}\label{ch1/chI023-eqn47}\\\begin{split}\dot y = x-y^2 = g(x, y).\end{split}\notag
\end{gather}
Łatwo znaleźć stany stacjonarne
\phantomsection\label{ch1/chI023:equation-eqn48}\begin{gather}
\begin{split} 1-xy = 0,\end{split}\label{ch1/chI023-eqn48}\\\begin{split} x-y^2 = 0.\end{split}\notag
\end{gather}
Z drugiego równania $x=y^2$ wstawiamy do pierwszego równania: $1-y^3=0$ czyli $y^3=1$. Stąd $y=1$ oraz $x=1$. Otrzymujemy stan stacjonarny
\phantomsection\label{ch1/chI023:equation-eqn49}\begin{gather}
\begin{split}(x_s, y_s) = (1, 1)\end{split}\label{ch1/chI023-eqn49}
\end{gather}
Obliczmy elementy macierzy Jacobiego
\phantomsection\label{ch1/chI023:equation-eqn50}\begin{gather}
\begin{split} \qquad \quad J = \begin{bmatrix}\frac{ \partial f}{\partial x}& \frac{\partial f}{\partial y}\\ \frac{\partial g}{\partial x}& \frac{\partial g}{\partial y}  \end{bmatrix}  =   \begin{bmatrix}-y & -x \\ 1& -2y  \end{bmatrix}    =  \begin{bmatrix}-1 & -1 \\ 1& -2  \end{bmatrix}\quad \quad \mbox{w punkcie} \quad (x=1, y=1)\end{split}\label{ch1/chI023-eqn50}
\end{gather}
Ślad macierz $J$, czyli suma elementów diagonalnych Tr:math:\emph{J = -3 \textless{} 0} oraz wyznacznik Det:math:\emph{J = 3 \textgreater{} 0}. Spełnione są relacje dla stany stacjonarnego który jest asymptotycznie stabilny. Wniosek: stan stacjonarny $(x_s, y_s) = (1, 1)$ jest asymptotycznie stabilny.

Można sprawdzić to, wyliczając explicite wartości własne macierzy Jacobiego:
\phantomsection\label{ch1/chI023:equation-eqn51}\begin{gather}
\begin{split}\lambda_1 = \frac{1}{2} (-3+i \sqrt 3), \quad \quad \lambda_1 = \frac{1}{2} (-3-i \sqrt 3)\end{split}\label{ch1/chI023-eqn51}
\end{gather}
Ich części rzeczywiste są ujemne:  $\Re(\lambda_1) = -3/2,  \Re(\lambda_2) = -3/2$.


\subsection{Przypadek C: Stabilność stanów stacjonarnych układów wielowymiarowych}
\label{ch1/chI023:przypadek-c-stabilnosc-stanow-stacjonarnych-ukladow-wielowymiarowych}
Dla układu równań różniczkowych o wymiarze większym niż 2 stosujemy te same metody co dla układu 2 równań różniczkowych.  Oczywiście istnieje więcej różnych przypadków i większe bogactwo własności  stanów stacjonarnych.  Nie będziemy przedstawiali tu przypadku o dowolnym wymiarze. Rozważymy przypadek 3 równań, aby pokazać podobieństwo do przypadku 2 równań. Analizujemy układ 3 równań różniczkowych:
\phantomsection\label{ch1/chI023:equation-eqn52}\begin{gather}
\begin{split}\dot x = F_1(x, y, z), \quad x(0) = x_0,\end{split}\label{ch1/chI023-eqn52}\\\begin{split}\dot y = F_2(x, y, z),  \quad y(0) = y_0,\end{split}\notag\\\begin{split}\dot z = F_3(x, y, z),  \quad z(0) = z_0.\end{split}\notag
\end{gather}
Stany stacjonarne są określone przez rozwiązania układu równań algebraicznych:
\phantomsection\label{ch1/chI023:equation-eqn53}\begin{gather}
\begin{split}F_1(x, y, z) = 0, \quad  F_2(x, y, z) = 0,  \quad  F_3(x, y, z)=0\end{split}\label{ch1/chI023-eqn53}
\end{gather}
Z równań tych otrzymujemy  stan stacjonarny   $(x, y, z,) = (x_{s}, y_{s}, z_s)$. Następnie obliczmy macierz Jacobiego
\phantomsection\label{ch1/chI023:equation-eqn54}\begin{gather}
\begin{split} J = \begin{bmatrix}\frac{ \partial F_1}{\partial x}& \frac{\partial F_1}{\partial y}&\frac{ \partial F_1}{\partial z}\\ \frac{ \partial F_2}{\partial x}&  \frac{ \partial F_2}{\partial y} &\frac{ \partial F_2}{\partial z} \\ \frac{ \partial F_3}{\partial x}& \frac{ \partial F_3}{\partial y}&\frac{ \partial F_3}{\partial z} \end{bmatrix}\end{split}\label{ch1/chI023-eqn54}
\end{gather}
Obliczmy  macierz Jacobiego w punkcie stacjonarnym $J=J(x_s, y_s, z_s)$.  W ten sposób otrzymujemy macierz liczbową. Kolejnym krokiem jest znalezienie wartości własnych $\lambda = \lambda_i   (i=1, 2, 3)$   tej macierzy, czyli rozwiązanie  wielomianu 3-go stopnia dla $\lambda$:
\phantomsection\label{ch1/chI023:equation-eqn55}\begin{gather}
\begin{split} \mbox{Det} (J -\lambda I)  = \mbox{Det}  \begin{bmatrix}\frac{ \partial F_1}{\partial x} -\lambda & \frac{\partial F_1}{\partial y}&\frac{ \partial F_1}{\partial z}\\ \frac{ \partial F_2}{\partial x}&  \frac{ \partial F_2}{\partial y} -\lambda &\frac{ \partial F_2}{\partial z} \\ \frac{ \partial F_3}{\partial x}& \frac{ \partial F_3}{\partial y}&\frac{ \partial F_3}{\partial z} -\lambda \end{bmatrix}  =  0\end{split}\label{ch1/chI023-eqn55}
\end{gather}
Macierz $I$ jest macierzą jednostkową, tzn. diagonalną $3\times 3$ i wszystkie elementy diagonalne to 1.

Jeżeli części  rzeczywiste  wszystkich wartości własnych macierzy Jacobiego $\lambda_i$ są  ujemne:
\phantomsection\label{ch1/chI023:equation-eqn56}\begin{gather}
\begin{split} Re(\lambda_1) < 0, \quad \quad Re(\lambda_2) < 0, \quad \quad Re(\lambda_3) < 0\end{split}\label{ch1/chI023-eqn56}
\end{gather}
to  stan stacjonarny jest asymptotycznie stabilny. Jeżeli  chociaż jedna z wartości własnych $\lambda_i$ ma część rzeczywistą dodatnią
\phantomsection\label{ch1/chI023:equation-eqn57}\begin{gather}
\begin{split} Re(\lambda_1) > 0, \quad \quad \mbox{i/lub} \quad \quad Re(\lambda_2) >  0, \quad \quad \mbox{i/lub} \quad \quad Re(\lambda_3) >  0\end{split}\label{ch1/chI023-eqn57}
\end{gather}
to stan stacjonarny jest niestabilny.

W przypadkach wielowymiarowych wygodnie jest stosować metody numeryczne. Obliczanie wartości własnych macierzy jest numerycznie zadaniem łatwym.


\subsubsection{Przykład 2: Model Lorenza}
\label{ch1/chI023:przyklad-2-model-lorenza}\phantomsection\label{ch1/chI023:equation-eqn58}\begin{gather}
\begin{split}\dot x = \sigma (y-x) = F_1(x, y,  z), \quad \quad x(0) = x_0,\end{split}\label{ch1/chI023-eqn58}\\\begin{split}\dot y = x(\rho - z) -y = F_2(x, y,  z),  \quad \quad y(0) = y_0,\end{split}\notag\\\begin{split}\dot z = x y - \beta z = F_3(x, y,  z), , \quad \quad z(0) = z_0.\end{split}\notag
\end{gather}
gdzie wszystkie parametry są dodatnie: $\sigma, \rho, \beta > 0$.

KROK 1: Znajdujemy stany stacjonarne czyli rozwiązujemy układ równań algebraicznych
\phantomsection\label{ch1/chI023:equation-eqn59}\begin{gather}
\begin{split}\sigma (y-x) =0,\end{split}\label{ch1/chI023-eqn59}\\\begin{split}x(\rho - z) -y = 0,\end{split}\notag\\\begin{split}x y - \beta z = 0.\end{split}\notag
\end{gather}
Możemy posłużyć się programem w Sage, ale układ ten jest na tyle prosty, że możemy go roziązać  sami. Z pierwszego równania wynika, że
\phantomsection\label{ch1/chI023:equation-eqn60}\begin{gather}
\begin{split}y=x\end{split}\label{ch1/chI023-eqn60}
\end{gather}
Z trzeciego równania otrzymujemy
\phantomsection\label{ch1/chI023:equation-eqn61}\begin{gather}
\begin{split}z= \frac{1}{\beta} x^2\end{split}\label{ch1/chI023-eqn61}
\end{gather}
Wstawiamy $x$ oraz $z$ do drugiego równania otrzymując relację:
\phantomsection\label{ch1/chI023:equation-eqn62}\begin{gather}
\begin{split}x \rho - x   {\frac{1}{\beta} }x^2 -x=0\end{split}\label{ch1/chI023-eqn62}
\end{gather}
czyli
\phantomsection\label{ch1/chI023:equation-eqn63}\begin{gather}
\begin{split}x [ \rho - 1  - {\frac{1}{\beta} }x^2]=0\end{split}\label{ch1/chI023-eqn63}
\end{gather}
Ta relacja jest prosta i wynika z niej że
\phantomsection\label{ch1/chI023:equation-eqn64}\begin{gather}
\begin{split}x= x_1 = 0 \quad \mbox{lub} \quad x=  x_2 = \sqrt{\beta ( \rho -1)} \quad \mbox{lub} \quad x= x_3 =  - \sqrt{\beta ( \rho -1)}\end{split}\label{ch1/chI023-eqn64}
\end{gather}
Otrzymujemy 3 stany stacjonarne:
\phantomsection\label{ch1/chI023:equation-eqn65}\begin{gather}
\begin{split}P_1 = (x_1, y_1, z_1) =  (0, 0, 0),\end{split}\label{ch1/chI023-eqn65}\\\begin{split}P_2 = (x_2, y_2, z_2) =  ( \sqrt{\beta ( \rho -1)}, \sqrt{\beta ( \rho -1)}, \rho - 1),\end{split}\notag\\\begin{split}P_3 = (x_3, y_3, z_3) =  ( - \sqrt{\beta ( \rho -1)}, -  \sqrt{\beta ( \rho -1)}, \rho - 1).\end{split}\notag
\end{gather}
Dla każdego stanu stacjonarnego musimy zbadać jego stabilność analizując zagadnienie własne dla macierzy Jacobiego. No to do dzieła...


\begin{verbatim}
#kilka zmiennych
reset()
var('x y z sigma rho beta alpha')
#i kilka zalozen
assume(sigma>0)
assume(rho>0)
assume(beta>0)
#definiujemy rownania rozniczkowe
F1 = sigma*(y-x)
F2 = x*(rho-z) - y
F3 = x*y - beta*z
html.table([['$F_1$','$F_2$','$F_3$'],[F1,F1,F3]])
print "stany stacjonarne"
rozw = solve([F1==0,F2==0,F3==0],[x,y,z])
P1 = rozw[2]
P2 = rozw[0]
P3 = rozw[1]
html.table([['$P_1$','$P_2$','$P_3$'],[P1,P2,P3]])
print 'Macierz Jakobiego'
J = matrix([
[diff(F1,x),diff(F1,y),diff(F1,z)],
[diff(F2,x),diff(F2,y),diff(F2,z)],
[diff(F3,x),diff(F3,y),diff(F3,z)]
])
show(J)
#
@interact
def _(punkt=['P1','P2','P3']):
 P = dict(zip(['P1','P2','P3'],[P1,P2,P3]))[punkt]
 J = matrix([
 [
 diff(F1,x)(x=P[0].rhs(),y=P[1].rhs(),z=P[2].rhs()),
 diff(F1,y)(x=P[0].rhs(),y=P[1].rhs(),z=P[2].rhs()),
 diff(F1,z)(x=P[0].rhs(),y=P[1].rhs(),z=P[2].rhs())
 ],
 [
 diff(F2,x)(x=P[0].rhs(),y=P[1].rhs(),z=P[2].rhs()),
 diff(F2,y)(x=P[0].rhs(),y=P[1].rhs(),z=P[2].rhs()),
 diff(F2,z)(x=P[0].rhs(),y=P[1].rhs(),z=P[2].rhs())
 ],
 [
 diff(F3,x)(x=P[0].rhs(),y=P[1].rhs(),z=P[2].rhs()),
 diff(F3,y)(x=P[0].rhs(),y=P[1].rhs(),z=P[2].rhs()),
 diff(F3,z)(x=P[0].rhs(),y=P[1].rhs(),z=P[2].rhs())
 ]
 ])
 print "macierz Jakobiego w punkcie %s" % P
 show(J)
 #zagadnienie własne macierzy
 dJ = det(J - alpha*matrix(3,3,1)) == 0
 rozwdJ1 = solve(dJ,alpha)
 show(rozwdJ1)
 b = 1
 s = 2
 r = 3
 i = 0
 for a in rozwdJ1:
     print "sprawdzamy"
     print "wartość własna alpha_%d"%(i+1)
     buf = real(a.rhs()(rho=r,beta=b,sigma=s))
     show(n(buf))
     if buf < 0:
         print "jest < 0"
     else:
         print "jest > 0"
         i += 1
 if i == 0:
     print "\n"
     print "Wynika z tego, że dla"
     print "beta=%s, sigma=%s, rho=%s"%(b,s,r)
     print "%s jest punktem stabilnym"%P
 else:
     print "\n"
     print "Wynika z tego, że dla"
     print "beta=%s, sigma=%s, rho=%s"%(b,s,r)
     print "%s nie jest punktem stabilnym"%P
\end{verbatim}



\section{Atraktory}
\label{ch1/chI024:atraktory}\label{ch1/chI024::doc}
Atraktor $A$ to taki zbiór w przestrzeni fazowej układu dynamicznego, że wiele trajektorii startujących nawet bardzo daleko od tego zbioru  dąży w miarę upływu czasu do tego zbioru $A$.  Najprościej jest to zrozumieć na przykładzie oscylatora tłumionego. Realizacją takiego oscylatora może być wahadło matematyczne czyli kulka  na nieważkim pręcie zawieszona w jakimś środowisku (byleby nie w próżni). Na kulkę działa siła przyciągania ziemskiego i siła tarcia. Kulkę wychylamy o dowolny kąt od pozycji pionowej  i obserwujemy trajektorię  kulki. Kulka wykonuje coraz to mniejsze wahania i po dostatecznie długim czasie zatrzymuje się w pozycji pionowej. Kulkę możemy wychylać o dowolny kąt, nadawać jej dowolną prędkość, a i tak po pewnym czasie kulka zatrzyma  się w pozycji pionowej, która odpowiada zerowemu wychyleniu kulki. Ten zerowy kąt wychylenia jest atraktorem. W tym przypadku atraktorem jest  punkt w przestrzeni fazowej. Ponieważ przestrzeń fazowa oscylatora harmonicznego  tłumionego jest 2-wymiarowa położenie-prędkość, atraktorem jest punkt $A = (położenie = 0, prędkość = 0)$.

Podamy teraz bardziej formalną definicję.
\begin{description}
\item[{Atraktor}] \leavevmode
Atraktorem nazywamy ograniczony zbiór w przestrzeni fazowej, do którego dążą asymptotycznie w czasie (tzn. gdy $t \to \infty$) obszary warunków początkowych o niezerowej objętości  w przestrzeni fazowej. Atraktor to inaczej zbiór przyciągający: przyciąga on trajektorie o różnych warunkach początkowych.  Ale nie musi on przyciągać wszystkich trajektorii. Dla danego układu dynamicznego może istnieć wiele atraktorów, nawet nieskończenie wiele. Atroktory mogą mieć  nieskomplikowaną strukturę: to może być  punkt, kilka piunktów, krzywa taka jak okrąg czy zdeformowana elipsa, część płaszczyzny, torus (podobny do dętki),  część przestrzeni. Atraktory mogą też  mieć skomplikowaną strukturę: może to być zbiór fraktalny, tzn. zbiór o niecałkowitym wymiarze, np. 0.63, 2.06. Taki atraktor nazywa się \emph{dziwnym atraktorem}.

\end{description}

Z atraktorami związane są obszary przyciągania lub baseny przyciągania $B(A)$. Basenem przyciągania atraktora $A$ nazywamy zbiór warunków początkowych $x_0$ , dla których trajektorie są przyciągane do $A$, czyli
\phantomsection\label{ch1/chI024:equation-eqn1}\begin{gather}
\begin{split}B(A) = \{ x_0: lim_{t \to \infty} x(t; x_0) \in A\}\end{split}\label{ch1/chI024-eqn1}
\end{gather}
gdzie $x(t; x_0)$ jest trajektorią startującą z warunku początkowego $x_0$, np. rozwiązaniem układu równań różniczkowych  z odpowiednimi warunkami początkowymi $\vec x_0$.


\subsection{Przykład 1: oscylator harmoniczny tłumiony}
\label{ch1/chI024:przyklad-1-oscylator-harmoniczny-tlumiony}
Jest tylko jeden atraktor: to punkt (0, 0). Basenem przyciągania jest cała płaszczyzna fazowa.


\begin{verbatim}
var('x,y')
T = srange(0,50,0.01)
sol1=desolve_odeint(vector([y,-x -0.2*y]), [0,1], T, [x,y])##warunek początkowy x=2, y=4
sol2=desolve_odeint(vector([y,-x -0.2*y]), [0,0.85], T, [x,y])##warunek początkowy x=-1, y=0.5
sol3=desolve_odeint(vector([y,-x -0.2*y]), [0,0.7], T, [x,y])##warunek początkowy x=0, y=0.9
p1=plot(x^2, -2, 2,figsize=(6,3), )
g1=list_plot(sol1.tolist(), plotjoined=1, figsize=(6,3),axes_labels=[r'$x$',r'$y$'])
g1 +=list_plot(sol2.tolist(), plotjoined=1, figsize=(6,3),color="red", axes_labels=[r'$x$',r'$y$'])
g1 +=list_plot(sol3.tolist(), plotjoined=1, figsize=(6,3),color="green", axes_labels=[r'$x$',r'$y$'])
html.table([["potencjał kwadratowy","oscylator tłumiony"],[p1,g1]])
html("<p align='center'>wszystkie rozwiązania dążą do punktu (0,0) </p>")
\end{verbatim}



\subsection{Przykład 2: oscylator nieliniowy (bistabilny)  tłumiony}
\label{ch1/chI024:przyklad-2-oscylator-nieliniowy-bistabilny-tlumiony}
Są dwa  atraktory:  punkt $(-x_s, 0)$ oraz symetryczny punkt $(x_s, 0)$, gdzie $x_s$ jest stanem stacjonarnym. Płaszczyzna fazowa dzieli się na 2 baseny przyciągania, które są ``pasiastymi'' naprzemiennymi zbiorami.Przejrzysta wizualizacja jest opracowana na naszej stronie internetowej:

\href{http://visual.icse.us.edu.pl/wizualizacje/mechanika-teoretyczna/zobacz/BasenyPrzyciagania/}{http://visual.icse.us.edu.pl/wizualizacje/mechanika-teoretyczna/zobacz/BasenyPrzyciagania/}


\begin{verbatim}
var('x,y')
T1 = srange(0,30,0.01)
so1=desolve_odeint(vector([y,2*x-1.2*x^3 -0.2*y]), [0,1], T1, [x,y])##warunek początkowy x=2, y=4
so2=desolve_odeint(vector([y,2*x-1.2*x^3 -0.2*y]), [0,2], T1, [x,y])##warunek początkowy x=-1, y=0.5
so3=desolve_odeint(vector([y,2*x-1.2*x^3-0.2*y]), [0,3], T1, [x,y])##warunek początkowy x=0, y=0.9
so4=desolve_odeint(vector([y,2*x-1.2*x^3-0.2*y]), [0,4], T1, [x,y])##warunek początkowy x=0, y=0.9
p11=plot(0.3*x^4 - x^2, -2, 2,figsize=(6,3), )
g11=list_plot(so1.tolist(), plotjoined=1, figsize=(6,3),axes_labels=[r'$x$',r'$y$'])
g11 +=list_plot(so2.tolist(), plotjoined=1, figsize=(6,3),color="red", axes_labels=[r'$x$',r'$y$'])
g11 +=list_plot(so3.tolist(), plotjoined=1, figsize=(6,3),color="green", axes_labels=[r'$x$',r'$y$'])
g11 +=list_plot(so4.tolist(), plotjoined=1, figsize=(6,3),color="black", axes_labels=[r'$x$',r'$y$'])
html.table([["potencjał bistabilny","oscylator nieliniowy tłumiony"],[p11,g11]])
html("<p align='center'> rozwiązania dążą albo do punktu $(-x_s,0)$ albo to punktu $(x_s,0)$ </p>")
\end{verbatim}



\subsection{Przykład 3: Cykl graniczny}
\label{ch1/chI024:przyklad-3-cykl-graniczny}
Atraktorem jest krzywa zamknięta (okrąg, elipsa, inne dowolne krzywe zamknięte).  Poniżej przedstawiamy dwa przykłady zaczerpnięte z modeli biologicznych.


\begin{verbatim}
var('x,y')
T3 = srange(0,50,0.01)
de1=y+x*(0.2-(x^2+y^2))
de2=-x+y*(0.2-(x^2+y^2))
s1=desolve_odeint(vector([de1, de2]), [0.5,0.5], T3, [x,y])##warunek początkowy x=2, y=4
s2=desolve_odeint(vector([de1, de2]), [0.01, 0.01], T3, [x,y])##warunek początkowy x=2, y=4
h1=list_plot(s1.tolist(), plotjoined=1, figsize=(6,3),color="red",axes_labels=[r'$x$',r'$y$'])
h2=list_plot(s2.tolist(), plotjoined=1, figsize=(6,3),axes_labels=[r'$x$',r'$y$'])
show(h1+h2)
\end{verbatim}



\begin{verbatim}
var('x,y')
a, b, d = 1.3, 0.33, 0.1
F(x,y)=x*(1-x) - a*x*y/(x+d)
G(x,y)= b*y*(1-y/x)
T = srange(0,80,0.01)
sl1=desolve_odeint(vector([F,G]), [0.2,0.3], T, [x,y])
sl2=desolve_odeint(vector([F,G]), [0.2,0.2], T, [x,y])
j1=list_plot(sl1.tolist(), plotjoined=1, color="red", figsize=(6, 3))
j2=list_plot(sl2.tolist(), plotjoined=1,  figsize=(6, 3))
show(j1+j2)
\end{verbatim}



\begin{verbatim}
var('x,y')
a, b, d = 1.3, 0.33, 0.1
F(x,y)=x*(1-x) - a*x*y/(x+d)
G(x,y)= b*y*(1-y/x)
T = srange(0,200,0.01)
sl1=desolve_odeint(vector([F,G]), [0.2,0.3], T, [x,y])
sl2=desolve_odeint(vector([F,G]), [0.2,0.2], T, [x,y])
j1=list_plot(sl1.tolist(), plotjoined=1, color="red", figsize=(6, 3))
j2=list_plot(sl2.tolist(), plotjoined=1,  figsize=(6, 3))
show(j1+j2)
\end{verbatim}



\subsection{Przykład 4: Atraktor Lorenza}
\label{ch1/chI024:przyklad-4-atraktor-lorenza}
Jest to przykład tak zwanego dziwnego atraktora. Najprostsza jego definicja to taka, że posiada on strukturę fraktala. O układzie Lorenza generującym ten fraktal można poczytać w poprzednim rozdziale tego skryptu, traktującym o stanach stacjonarnych.


\begin{verbatim}
var('x y z')
rho=28
sigma=10
beta=8/3
F1 = sigma*(y-x)
F2 = x*(rho-z) - y
F3 = x*y - beta*z
T = srange(0,100,0.01)
atraktor_lorenza = desolve_odeint(vector([F1,F2,F3]), [0,0.5,1], T, [x,y,z])
p2d = list_plot(zip(atraktor_lorenza[:,0],atraktor_lorenza[:,1]), plotjoined=1, figsize=4)
p3d = list_plot(atraktor_lorenza.tolist(), plotjoined=1, viewer='tachyon', figsize=4)
print "2D rysunek atraktora Lorenza"
p2d.show()
print "3D rysunek atraktora Lorenza"
p3d.show()
\end{verbatim}



\chapter{Chaos w układach dynamicznych}
\label{index:chaos-w-ukladach-dynamicznych}

\section{Determinizm a przewidywalność}
\label{ch2/chII011:determinizm-a-przewidywalnosc}\label{ch2/chII011::doc}
Determinizm jest terminem wieloznacznym.  W odniesieniu do nauk przyrodniczych możemy mówić o tym że :

Każde zjawisko jest wyznaczone przez swoje przyczyny i całokształt warunków.
Każde zjawisko podlega prawidłowościom przyrody, które wyrażane są w postaci odpowiednich praw nauki.
Jeżeli dysponujemy odpowiednią wiedzą (o przyczynach,  prawidłowościach), to można (przynajmniej w zasadzie) przewidywać przyszły bieg zdarzeń.  Poznawcze znaczenie zasady przyczynowości  sprowadza się do możliwości przewidywania. Determinizm i indeterminizm  występują już w starożytnej filozofii przyrody. Demokryt twierdził, że nie ma w świecie zdarzeń przypadkowych, lecz „wszystko dzieje się wskutek konieczności''. Determinizm  łączono z poglądem, że prawa przyrody mają charakter jednoznaczny. Okazało się jednak że wiele zjawisk podlega jedynie prawom statystycznym. Pogląd, że pewne zjawiska przyrody nie podlegają prawom jednoznacznym, ale jedynie statystycznym to indeterminizm .  Doniosła rola praw statystycznych w fizyce współczesnej wymaga rozszerzenia formuły determinizmu: każde zdarzenie podlega prawom przyrody jednoznacznym (determinizm jednoznaczny, mocny) bądź statystycznym (determinizm probabilistyczny, umiarkowany). Wówczas indeterminizm (umiarkowany) mieści się w ramach determinizmu  w szerszym ujęciu. Wydawało się, że w mechanice, stworzonej przez Galileusza i Newtona, znając położenie i prędkość (lub położenie i pęd) ciała materialnego można przewidzieć cały przyszły ruch tego ciała (a więc podać położenie i pęd w każdej chwili późniejszej), a także ustalić, jaki był ruch tego ciała w przeszłości. Wszystko to jest słuszne  przy założeniu, że znane są  wszystkie siły działające na ciało w przeszłości i w przyszłości. Wynikało to z twierdzeń matematycznych o jednoznaczności rozwiązań równań różniczkowych, a takimi są równania Newtona.

Jednakże historia pokazała, że  w fizyce klasycznej są znane teorie statystyczne, niedeterministyczne.  Rozważmy dla przykładu  gaz zamknięty w jakimś naczyniu. Wiemy, że w warunkach równowagi termodynamicznej dwie równe co do objętości części tego naczynia będą zawierać jednakową liczbę cząsteczek gazu. Nie wiemy jednak, które cząsteczki znajdą się w której z dwu połówek naczynia. Sytuacja pozornie przypomina prawo rozpadu: można podać taki czas, w którym rozpadnie się połowa atomów, i w ten sposób podzielić wszystkie atomy na dwie równe części - te, które się w tym czasie rozpadną, i te, które się nie rozpadną, podobnie jak podzieliliśmy cząsteczki gazu według kryterium, w której połowie naczynia się znajdują. W klasycznej fizyce statystycznej znamy prawa rządzące zachowaniem pojedynczych cząsteczek (są nimi z założenia prawa mechaniki newtonowskiej), a nasza niewiedza co do tego zachowania jest spowodowana po pierwsze niemożliwością śledzenia ruchu wielu miliardów obiektów, a po drugie, brakiem potrzeby, aby to czynić: wystarczy nam znać właśnie tylko pewne wielkości średnie, które ujawniają się fenomenologicznie, na przykład jako temperatura gazu, czy też jego ciśnienie. Tak więc rzeczywisty kompletny opis stanu gazu musiałby zawierać informację dotyczącą N wektorów położenia i N wektorów pędu (N - liczba cząsteczek gazu), co jest liczbą ogromną, podczas gdy opis statystyczny ogranicza się do kilku potrzebnych liczb. Opis statystyczny odnosi się do ogromnej liczby cząstek i jest to opis oparty o teorię prawdopodobieństwa i teorię procesów stochastycznych. Z definicji jest to opis niedeterministyczny. Ale jak powiedzieliśmy, opis jednej cząstki jest w pełni deterministyczny. Twierdzenia o jednoznaczności rozwiązań równań różniczkowych dawały nadzieję na totalny determinizm i przewidywalność ruchu pojedyńczych cząstek. Nadzieja ta z praktycznego punktu widzenia okazała się mrzonką.  W latach 50-tych XX wieku pokazano, że z praktycznego punktu widzenia determinizm mechaniki Newtona jest złudny i ugruntowana wiara w przewidywalność zachowania się prostych układów  mechanicznych  załamała się. Pojawiły się liczne przykłady, a później teoria matematyczna, pokazujące  niemożliwość przewidywania czasowej ewolucji prostych układów mechanicznych. Podkreślamy, że chodzi tu o praktyczne aspekty przewidywalności. Z matematycznego punktu widzenia, przewidywalność jest ciągle słuszna.  Dobitnym przykładem nieprzewidywaloności w praktyce jest prognoza pogody, co udowadnia codzienne życie. Poniżej przedstawimy zagadnienia, które ukażą nam, co oznacza nieprzewidywalność w teorii deterministycznej. Pokażemy, dlaczego ewolucja określona przez determinizm równań Newtona jest nieprzewidywalna. Ta deterministyczna nieprzewidywalność ma swoją nazwę: deterministyczny chaos.


\section{Model chaosu. Układ bistabilny (oscylator Duffinga)}
\label{ch2/chII011:model-chaosu-uklad-bistabilny-oscylator-duffinga}
Aby dobrze zrozumieć istotę chaotycznego zachowania się układów dynamicznych, posłużymy się prostym przykładem z mechaniki klasycznej Newtona. Rozpatrujemy jednowymiarowy ruch cząstki wzdłuż osi OX opisany równaniem Newtona:
\phantomsection\label{ch2/chII011:equation-eqn1}\begin{gather}
\begin{split}m \ddot x = F(x, \dot x, t) =ax - bx^3 - \gamma \dot x + A \cos(\Omega t), \qquad x(0) = x_0, \quad  \dot x(0) = v(0) = v_0\end{split}\label{ch2/chII011-eqn1}
\end{gather}
Model ten wydaje się być banalnie prosty.

jest to ruch cząstki w polu siły $F(x) = ax-bx^3  (a, b > 0)$
jest to ruch tłumiony tarciem $F(v) = - \gamma v = -\gamma \dot x$ oraz
na cząstkę działa siła periodyczna w czasie $F(t) = A\cos(\Omega t)$.
Zauważmy, że średnia wartość siły $F(t)$ na jednym okresie wynosi  zero, czyli średnio działa zerowa siła! Siła $F(x) = x-x^3$ to jest siła potencjalna. Dlatego warto wykreślić potencjał $V(x)$  odpowiadający tej sile:
\phantomsection\label{ch2/chII011:equation-eqn2}\begin{gather}
\begin{split}V(x) = - \int F(x) dx = \frac{1}{4}  b x^4 - \frac{1}{2}  a x^2\end{split}\label{ch2/chII011-eqn2}
\end{gather}
Potencjał  ten nazywa się potencjałem bistabilnym i jest jednym z najważniejszych modelowych potencjałów w fizyce, począwszy od teorii Netwona, poprzez teorię przejść fazowych, teorię aktywacji  reakcji chemicznych, a na teorii cząstek elementarnych (mechanizm Higgsa) skończywszy. Poniżej pokazujemy wykres tego potencjału (mówiąc precyzyjnie: energii potencjalnej).

Powyższe równanie Newtona  ma kilka realizacji.
\begin{description}
\item[{1 Pierwszy przykład realizacji: Metalowa kulka zawieszona na nieważkim}] \leavevmode
pręcie w polu dwóch magnesów (które modelują bistabilny potencjał).
Na kulkę działa w kierunku poziomym periodyczne pole elektryczne
$A\cos(\omega t)$. Układ ten jest przedstawiony na rysunku.
\begin{figure}[htbp]
\centering
\capstart

\includegraphics{pendulum.png}
\caption{Przykład realizacji układu bistabilnego.}\end{figure}

\end{description}

2 Drugi przykład realizacji: Obwód elektroniczny, który jest szczegółowo opisany w pracy:
\begin{quote}

B. K. Jones and G. Trefan, Am. J. Phys. 69 (2001) str. 464.
``The Duffing oscillator: A precise electronic analog chaos demonstrator
for the undergraduate laboratory ``
\end{quote}


\begin{verbatim}
# Przeskalowany potencjał bistabilny: a=b=1
p = plot(0.25*x^4 - 0.5*x^2, (x,-1.6,1.6), figsize=(6,4), axes_labels=[r'$x$',r'$V(x)$'], color="blue")
p += text("$-x_s$",(-1,0.025),fontsize=16, color='black')
p += text("$x_s$",(1,0.025),fontsize=16, color='black')
p.show()
\end{verbatim}



\subsection{Skalowanie}
\label{ch2/chII011:skalowanie}
Układ opisany powyżej zawiera 6 parametrów. Część parametrów można wyeliminować poprzez przeskalowanie równania do postaci bezwymiarowej. Istnieje kilka  możliwości. Zwykle zaczynamy od skalowania czasu i położenia. Nowy bezwymiarowy czas $\tau$ ma postać:
\phantomsection\label{ch2/chII011:equation-eqn3}\begin{gather}
\begin{split}s = \frac{t}{\tau_0}, \qquad \tau_0^2 = \frac{m}{a}\end{split}\label{ch2/chII011-eqn3}
\end{gather}
Nowe bezwymiarowe położenie definiujemy jako
\phantomsection\label{ch2/chII011:equation-eqn4}\begin{gather}
\begin{split}X = \frac{x}{L}, \qquad L^2 = \frac{a}{b}\end{split}\label{ch2/chII011-eqn4}
\end{gather}
Wówczas bezwymiarowa postać równania ruchu jest następująca:
\phantomsection\label{ch2/chII011:equation-eqn5}\begin{gather}
\begin{split}\ddot X = X - X^3 - \gamma_0 \dot X + A_0 \cos(\omega_0 s), \qquad X(0) = X_0, \quad  \dot X(0) = \dot X_0\end{split}\label{ch2/chII011-eqn5}
\end{gather}
Obecnie występują 3 przeskalowane parametry:
\phantomsection\label{ch2/chII011:equation-eqn6}\begin{gather}
\begin{split} \gamma_0  = \frac{\tau_0^2}{m L} \gamma, \qquad A_0 = \frac{\tau_0^2}{m L} A, \qquad \omega_0 = \tau_0 \Omega\end{split}\label{ch2/chII011-eqn6}
\end{gather}
W dalszej części będziemy posługiwali się tylko i wyłącznie przeskalowanym równaniem. Dlatego wygodnie będzie używać ``starych'' oznaczeń: Bedziemy analizowali równanie w postaci
\phantomsection\label{ch2/chII011:equation-eqn7}\begin{gather}
\begin{split}\ddot x = x - x^3 - \gamma \dot x + A \cos(\omega_0 t ), \qquad x(0) = x_0, \quad  \dot x(0) = \dot y_0 = v_0\end{split}\label{ch2/chII011-eqn7}
\end{gather}
gdzie przeskalowany potencjał
\phantomsection\label{ch2/chII011:equation-eqn7a}\begin{gather}
\begin{split}V(x) = - \int F(x) dx = \frac{1}{4} x^4 - \frac{1}{2} x^2\end{split}\label{ch2/chII011-eqn7a}
\end{gather}
Przeskalowane równanie jest w takiej postaci, że przyjmujemy wartości parametrów $m=1,  a=1,  b=1$.


\subsection{Krok 1. Układ zachowawczy}
\label{ch2/chII011:krok-1-uklad-zachowawczy}
W pierwszym  kroku rozpatrujemy najprostszy przypadek (pamiętajmy o przeskalowanej postaci, w której masa cząstki $m=1$)):
\phantomsection\label{ch2/chII011:equation-eqn8}\begin{gather}
\begin{split}\ddot x = x - x^3 = - V'(x), \qquad x(0) = x_0, \quad  \dot x(0) = v(0) =  v_0\end{split}\label{ch2/chII011-eqn8}
\end{gather}
Jest on równoważny układowi 2 równań różniczkowych, autonomicznych, pierwszego rzędu:
\phantomsection\label{ch2/chII011:equation-eqn9}\begin{gather}
\begin{split}\dot x = v, \qquad x(0) = x_0,\end{split}\label{ch2/chII011-eqn9}\\\begin{split}\dot v = x - x^3, \qquad v(0) = v_0.\end{split}\notag
\end{gather}
Oznacza to, że przestrzeń fazowa jest 2-wymiarowa.

Taki przypadek był już rozpatrywany: jest to układ zachowawczy o jednym stopniu swobody. Istnieje jedna stała ruchu (jedna całka ruchu), a mianowicie całkowita energia układu:
\phantomsection\label{ch2/chII011:equation-eqn10}\begin{gather}
\begin{split}\frac{1}{2} \dot x^2(t) + V(x(t)) = const. = E  = E_k + E_p = \frac{1}{2} \dot x^2(0) + V(x(0)) = \frac{1}{2}  v_0^2 + V(x_0)\end{split}\label{ch2/chII011-eqn10}
\end{gather}
na którą składa się energia kinetyczna $E_k$ oraz energia potencjalna $E_p$.  Stała $E$ jest określona przez warunki początkowe $x(0) = x_0$ oraz $v(0) = v_0$.  Ponieważ jest zachowana całkowita energia układu, ruch jest periodyczny. Nie istnieją atraktory i nie istnieją  asymptotycznie stabilne stany stacjonarne. Krzywe fazowe są zamknięte co oznacza że  cząstka porusza się periodycznie w czasie. W zależności od warunków początkowych, amplituda dragań jest większa lub mniejsza, ponieważ warunki początkowe wyznaczają wartość stałej ruchu $E$. Jeżeli dwa warunki początkowe $(x_{01}, v_{01})$  oraz  $(x_{02}, v_{02})$ nieznacznie się różnią, np. w sensie odległości:
\phantomsection\label{ch2/chII011:equation-eqn11}\begin{gather}
\begin{split}| [x_{01}^2 +  v_{01}^2] - [x_{02}^2 +  v_{02}^2] | << 1\end{split}\label{ch2/chII011-eqn11}
\end{gather}
to krzywe fazowe nieznacznie się różnią i ruch cząstki dla tych dwóch warunków początkowych nieznacznie się różni. Mówimy wówczas, że układ jest nieczuły na zmianę warunków początkowych.  Jak widać z powyższego wzoru, dwa różne warunki początkowe oznaczają, że układ ma dwie różne energie $E$. To z kolei oznacza, że częstości ruchu periodycznego także będą różne.  Różnica częstości powoduje, że cząstki  będą się powoli oddalać od siebie, ale tempo oddalania będzie liniowe w czasie.  Gdyby tempo oddalania było znacznie szybsze, a mianowicie rosło eksponencjalnie w czasie, zachowanie takie nazwalibyśmy chaotycznym.  Do tego problemu powrócimy poniżej, ponieważ jest on kluczowym dla zrozumienia chaotycznego zachowania się układu.

Poniżej przedstawiamy potencjał i  krzywe fazowe dla tego przypadku.


\begin{verbatim}
#parametry dla wizualizacji
var('x v')
x0 = 1.5
v0 = 0.2
E = 0.25*x0^4 - 0.5*x0^2 + v0^2
#
#prawo zachowania energii
V=0.25*x^4 - 0.5*x^2
PZE = v^2 + V == E
#
#wychylenia ekstremalne
print "ekstremalne wychylenia dla (x0,v0) = (%.2f,%.2f)"%(x0,v0)
rozw = solve(PZE(v=0), x); show(rozw)
xmin = min([i.rhs() for i in rozw if imag(i.rhs()) == 0])
xmax = max([i.rhs() for i in rozw if imag(i.rhs()) == 0])
#
#i jego rozwiązanie
print "ekstremalne prędkości dla (x0,v0) = (%.2f,%.2f)"%(x0,v0)
rozw = solve(PZE, v); show(rozw)
v1=rozw[0].rhs()
v2=rozw[1].rhs()
vmax = abs(v1(x=0))
#
#krzywe fazowe
start_point = (x0,V(x=x0))
p0 = point(start_point,size=30) + text(r"$  x_0$",start_point,vertical_alignment='bottom',horizontal_alignment='left')
p1 = plot(V,(x,xmin,xmax))
p21 = plot(v1, (x,xmin,xmax), color='red')
p22 = plot(v2, (x,xmin,xmax), color='green')
(p0+p1).show(figsize=4)
(p21+p22).show(figsize=4)
\end{verbatim}



\subsection{Krok 2. Układ dysypatywny czyli wpływ tarcia.}
\label{ch2/chII011:krok-2-uklad-dysypatywny-czyli-wplyw-tarcia}
W drugim  kroku dodajemy tarcie i rozpatrujemy równanie ruchu w postaci:
\phantomsection\label{ch2/chII011:equation-eqn12}\begin{gather}
\begin{split}\ddot x =  x - x^3 -\gamma \dot x , \qquad x(0) = x_0, \quad  \dot x(0) = v(0) =  v_0\end{split}\label{ch2/chII011-eqn12}
\end{gather}
Jest on równoważny układowi 2 równań różniczkowych, autonomicznych, pierwszego rzędu:
\phantomsection\label{ch2/chII011:equation-eqn13}\begin{gather}
\begin{split}\dot x = v, \qquad x(0) = x_0,\end{split}\label{ch2/chII011-eqn13}\\\begin{split}\dot v = x - x^3 -\gamma v , \qquad v(0) = v_0.\end{split}\notag
\end{gather}
Oznacza to, że przestrzeń fazowa jest 2-wymiarowa.

Taki przypadek był także rozpatrywany: jest to układ dysypatywny o jednym stopniu swobody. Nie istnieje już stała ruchu $E$.  Całkowita energia układu maleje w czasie.  W tym układzie  istnieją 3 stany stacjonarne. Stany te określone są przez równanie:
\phantomsection\label{ch2/chII011:equation-eqn14}\begin{gather}
\begin{split}x-x^3=0, \qquad \mbox{stąd} \qquad x_{s0}=0, \quad x_{s1} = 1, \quad x_{s2} = -1\end{split}\label{ch2/chII011-eqn14}
\end{gather}
Stany stacjonarne $x_{s1} = 1$ oraz $x_{s2} = -1$  są  stabilne. Stan $x_{s0}=0$ jest niestabilny. Istnieją 2 atraktory  $A_1= x_{s1} = 1$ oraz $A_2= x_{s2} = -1$ i  2 obszary przyciągania $B(A_1)$ oraz $B(A_2)$, których suma mnogościowa $B(A_1) \cup  B(A_2) = R^2$ jest całą płaszczyzną.  Krzywe fazowe  zawsze dążą do jednego z atraktorów lub do niestabilnego stanu stacjonarnego. Jeżeli dwa warunki początkowe $(x_{01}, v_{01})$  oraz  $(x_{02}, v_{02})$ nieznacznie się różnią np. w sensie odległości:
\phantomsection\label{ch2/chII011:equation-eqn15}\begin{gather}
\begin{split}| [x_{01}^2 +  v_{01}^2] - [x_{02}^2 +  v_{02}^2] | << 1\end{split}\label{ch2/chII011-eqn15}
\end{gather}
i są w tym samym obszarze przyciągania, to krzywe fazowe nieznacznie się różnią i ruch cząstki dla tych dwóch warunków początkowych nieznacznie się różni. Mówimy wówczas, że układ jest nieczuły na zmianę warunków początkowych. Natomiast jeżeli dwa warunki początkowe $(x_{01}, v_{01}) \in B(A_1)$  oraz  $(x_{02}, v_{02}) \in B(A_2)$ nieznacznie się różnią, ale są w dwóch obszarach przyciągania $B(A_1)$ oraz $B(A_2)$, to trajektorie zaczną po pewnym czasie różnić się znacznie, będą przyciągane do dwóch różnych atraktorów  i będą dążyć  do dwóch różnych stanów stacjonarnych :math:{}` x\_\{s1\} = 1{}` oraz :math:{}` x\_\{s2\} = -1{}`. Tym niemniej, w takiej sytuacji mówimy, że układ jest nieczuły na zmianę warunków początkowych w sensie o którym mowa powyżej.
\begin{figure}[htbp]
\centering
\capstart

\includegraphics{baseny_tarcie.jpg}
\caption{Diagram basenów przyciągania dla potencjału bistabilnego}\end{figure}

Kolor niebieski to obszar warunków początkowych które są ``przyciągane''  do atraktora $(1, 0)$, do prawego
minimum potencjału. Kolor czerwony to obszar warunków początkowych które są ``przyciągane''  do atraktora
$(-1, 0)$, do lewego minimum potencjału. W zależności od wartości stałej tłumienia $\gamma$, diagram
ten przybiera nieco inne kształty, ale struktura dwu-kolorowych pasów pozostaje. Brzeg obszarów przyciągania jest
gładką krzywą, której wymiar wynosi 1. Jeżeli warunki początkowe są położone dokładnie na tym brzegu, to cząstka
porusza się do niestabilnego stanu stacjonarnego $(x=0, v=0)$ (maksimum potencjału).


\begin{verbatim}
# wykresy dla przypadku z tłumieniem
var('x v')
x01, v01 = 1.50, 0
x02, v02 = 1.52, 0
#
# siła
F = x-x^3
V = -integrate(F,x)
#
# tarcie: parametr gamma
g = 0.1
#
# numeryczne rozwiazanie równań ruchu
T = srange(0,20*pi,0.01)
num1 = desolve_odeint(vector([v,F-g*v]), [x01,v01], T, [x,v])
num2 = desolve_odeint(vector([v,F-g*v]), [x02,v02], T, [x,v])
#
#krzywe fazowe
lt  = plot(V, (x, -max([abs(x01),abs(x02)]),max([abs(x01),abs(x02)])), color='black', figsize=4)
lt += point((x01,V(x=x01)), color='green', size=50, axes_labels=['$x$','$V(x)$'])
lt += point((x02,V(x=x02)), color='red', size=50)
lb  = list_plot(num1.tolist(), plotjoined=1, color='green', axes_labels=['$x(t)$','$v(t)$'])
lb += list_plot(num2.tolist(), plotjoined=1, color='red', figsize=4)
rt  = list_plot(zip(T,num1[:,0].tolist()), plotjoined=1, color='green', axes_labels=['$t$','$x(t)$'])
rt += list_plot(zip(T,num2[:,0].tolist()), plotjoined=1, color='red', figsize=4)
rb  = list_plot(zip(T,num1[:,1].tolist()), plotjoined=1, color='green', axes_labels=['$t$','$v(t)$'])
rb += list_plot(zip(T,num2[:,1].tolist()), plotjoined=1, color='red', figsize=4)
#
html("""
<p align='center'>rozwiązania z warunkami początkowymi
<span style="color:green">($x_{01},v_{01}$)=(%.2f,%.2f)</span>
<span style="color:red">($x_{02},v_{02}$)=(%.2f,%.2f)</span>
dążą do tego samego atraktora:
(x,v)=(-1,0)
</p>
"""%(x01,v01,x02,v02))
html.table([[lt,rt],[lb,rb]])
\end{verbatim}


Na powyższym zestawie rysunków,  2 warunki początkowe leżą w tym samym obszarze  przyciągania  atraktora $(-1, 0)$. Oznacza to, że 2 warunki początkowe są umiejscowione w czerwonym obszarze na diagramie basenów przyciągania pokazanym powyżej. Układ nie jest czuły na zmianę warunków początkowych, gdy leżą one w tym samym basenie przyciągania.


\begin{verbatim}
# wykresy dla przypadku z tłumieniem
var('x v')
x01, v01 = 1.58, 0
x02, v02 = 1.57, 0
#
# siła
F = x-x^3
V = -integrate(F,x)
#
# tarcie: parametr gamma
g = 0.1
#
# numeryczne rozwiazanie równań ruchu
T = srange(0,20*pi,0.01)
num1 = desolve_odeint(vector([v,F-g*v]), [x01,v01], T, [x,v])
num2 = desolve_odeint(vector([v,F-g*v]), [x02,v02], T, [x,v])
#
# wykresy funkcji
lt  = plot(V, (x, -max([abs(x01),abs(x02)]),max([abs(x01),abs(x02)])),color='black',  figsize=4)
lt += point((x01,V(x=x01)), color='blue', size=50, axes_labels=['$x$','$V(x)$'])
lt += point((x02,V(x=x02)), color='red', size=50)
lb  = list_plot(num1.tolist(), plotjoined=1, color='blue', axes_labels=['$x(t)$','$v(t)$'])
lb += list_plot(num2.tolist(), plotjoined=1, color='red', figsize=4)
rt  = list_plot(zip(T,num1[:,0].tolist()), plotjoined=1, color='blue', axes_labels=['$t$','$x(t)$'])
rt += list_plot(zip(T,num2[:,0].tolist()), plotjoined=1, color='red', figsize=4)
rb  = list_plot(zip(T,num1[:,1].tolist()), plotjoined=1, color='blue', axes_labels=['$t$','$v(t)$'])
rb += list_plot(zip(T,num2[:,1].tolist()), plotjoined=1, color='red', figsize=4)
#
html("""
<p align='center'>rozwiązania z warunkami początkowymi
<span style="color:blue">($x_{01},v_{01}$)=(%.2f,%.2f)</span>
<span style="color:red">($x_{02},v_{02}$)=(%.2f,%.2f)</span>
dążą do różnych atraktorów:
<span style="color:blue">(x,v)=(1,0)</span>
<span style="color:red">(x,v)=(-1,0)</span>
</p>
"""%(x01,v01,x02,v02))
html.table([[lt,rt],[lb,rb]])
\end{verbatim}


Na powyższym zestawie rysunków,  2 warunki początkowe leżą w dwóch różnych obszarach  przyciągania.  Oznacza to, że 1 warunek  początkowy leży w  niebieskim obszarze na diagramie basenów przyciągania, natomiast  2 warunek  początkowy leży w  czerwonym obszarze na diagramie basenów przyciągania. Te dwa warunki początkowe leżą blisko brzegu 2 basenów przyciągania. Dlatego układ jest czuły na zmianę warunków początkowych, pod warunkiem że leżą one w dwóch różnych basenach przyciągania. Ale to nie jest jeszcze kryterium własności chaotyczych układu.


\subsection{Krok 3. Układ z tarciem i periodyczną siłą.}
\label{ch2/chII011:krok-3-uklad-z-tarciem-i-periodyczna-sila}
W trzecim kroku dodajemy siłę periodyczną w czasie  i rozpatrujemy równanie ruchu w wyjściowej pełnej postaci:
\phantomsection\label{ch2/chII011:equation-eqn16}\begin{gather}
\begin{split}\ddot x =  x - x^3 -\gamma \dot x  +  A \cos (\omega_0 t) , \qquad x(0) = x_0, \quad  \dot x(0) = v(0) =  v_0\end{split}\label{ch2/chII011-eqn16}
\end{gather}
Jest on równoważny układowi 3 równań różniczkowych, autonomicznych, pierwszego rzędu:
\phantomsection\label{ch2/chII011:equation-eqn17}\begin{gather}
\begin{split}\dot x = v, \qquad x(0) = x_0,\end{split}\label{ch2/chII011-eqn17}\\\begin{split}\dot v = x - x^3 -\gamma v + A \cos z , \qquad v(0) = v_0,\end{split}\notag\\\begin{split}z = \omega_0, \qquad z(0) = 0.\end{split}\notag
\end{gather}
Oznacza to, że przestrzeń fazowa jest 3-wymiarowa.

Matematycy wolą przepisać powyższy układ równań dla ``tradycyjnych''  3 zmiennych $(x, y, z)$ w postaci:
\phantomsection\label{ch2/chII011:equation-eqn18}\begin{gather}
\begin{split}\dot x = y, \qquad x(0) = x_0,\end{split}\label{ch2/chII011-eqn18}\\\begin{split}\dot y = x - x^3 -\gamma y + A \cos z , \qquad y(0) = y_0,\end{split}\notag\\\begin{split}z = \omega_0, \qquad z(0) = 0.\end{split}\notag
\end{gather}
czyli prędkość cząstki $v$ jest teraz oznaczona jako $v=y$.

Okazuje się, że pełny układ wykazuje radykalnie inne własności od poprzednich 2 przypadków. Z punktu widzenia fizyki mamy taki oto proces:  Cząstka porusza się w bistabilnym potencjale. Ponieważ potencjał dąży do nieskończoności gdy położenie dąży do nieskończoności, ruch cząstki jest ograniczony; cząstka jest uwięziona w potencjale i nie może uciec do nieskończoności. Siła tarcia pcha cząstkę do jednego ze (``starych'') stanów stacjonarnych  $x_{s1}$  lub $x_{s2}$. Z kolei zewnętrzna siła periodyczna w czasie pompuje energię do układu i przeciwdziała sile tarcia. Cząstka już nie dąży do stanu stacjonarnego, nie zatrzyma się dla długich czasów ale będzie  ciągle poruszać się i nigdy już nie spocznie. Istotne stają się efekty inercjalne związane z masą czastki, które są odzwierciedlone w wyrazie $\dot y$, czyli przyśpieszeniu cząstki. Istotne jest to, że nie jest to ruch przetłumiony. W konsekwencji układ nie posiada stanu stacjonarnego w postaci punktu w przestrzeni fazowej jak to było w przypadku 2. Wszystkie te powyższe czynniki stają się istotne dla zrozumienia  skomplikowanych i złożonych własności ewolucji cząstki.


\begin{verbatim}
# przykładowa trajektoria  (górny wykres)
# wraz z krzywą fazową (dolny wykres)
var('x y z')
T = srange(0,150*pi,0.01)
sol=desolve_odeint( vector([y,x-x^3-0.26*y+0.3*cos(z), 1]), [0.1,0.1,0],T,[x,y,z])
t = line(zip(T,sol[:,0]), figsize=(12,4), axes_labels=["$t$","$x(t)$"], frame=1, axes=0)
b = line(zip(sol[:,0],sol[:,1]), figsize=(12,4), axes_labels=["$x(t)$","$v(t)$"], frame=1, axes=0)
html.table([[t],[b]])
\end{verbatim}



\subsection{Ruch periodyczny o okresie 1}
\label{ch2/chII011:ruch-periodyczny-o-okresie-1}
W modelu występują 3 bezwymiarowe parametry: współczynnik tarcia $\gamma$, amplituda zewnętrznej siły $A$ oraz częstość drgań $\omega_0$ siły periodycznej w czasie. Poniżej pokażemy kilka charakterystycznych trajektorii układu. Zaczniemy od prostej periodycznej ewolucji, ruchu okresowego o tzw. okresie 1.

Załóżmy następujące wartości parametrów:
\phantomsection\label{ch2/chII011:equation-eqn19}\begin{gather}
\begin{split}\gamma = 0.15, \qquad A = 0.3, \qquad \omega_0 = 1\end{split}\label{ch2/chII011-eqn19}
\end{gather}
W tym przypadku obserwujemy regularny ruch. Jeżeli nieco zaburzymy warunki początkowe, to nowy ruch jest także regularny (trzeba być ostrożnym, gdy mówimy ``nieco zaburzymy'').


\begin{verbatim}
# wykresy dla przypadku z tłumieniem
var('x y z')
x0, y0, z0 = 0.1,0.1,0
kolor = 'green'
#
# siła
F = x-x^3
V = -integrate(F,x)
#
# tarcie: parametr gamma
g = 0.1
A = 0.3
w = 1
#
# układ różniczkowych równań ruchu
dx = y
dy = F - g*y + A*cos(z)
dz = w
#
# numeryczne rozwiazanie równań ruchu
T = srange(0,30*pi,0.01)
num = desolve_odeint(vector([dx,dy,dz]), [x0,y0,z0], T, [x,y,z])
#
# wykresy funkcji
xmin = 1.5
lt  = plot(V, (x,-xmin,xmin), figsize=4)
lt += point((x0,V(x=x0)), color=kolor, size=50, axes_labels=['$x$','$V(x)$'])
lb  = list_plot(zip(num[:,0],num[:,1]), plotjoined=1, color=kolor, axes_labels=['$x(t)$','$v(t)$'], figsize=4)
rt  = list_plot(zip(T,num[:,0].tolist()), plotjoined=1, color=kolor, axes_labels=['$t$','$x(t)$'], figsize=4)
rb  = list_plot(zip(T,num[:,1].tolist()), plotjoined=1, color=kolor, axes_labels=['$t$','$v(t)$'], figsize=4)
#
html("""Układ równań różniczkowych
$\dot{x} = %s$
$\dot{y} = %s$
$\dot{z} = %s$
z warunkami początkowymi
$(x_0,y_0,z_0) = (%.2f,%.2f,%.2f)$
"""%(dx,dy,dz,x0,y0,z0))
html.table([[lt,rt],[lb,rb]])
\end{verbatim}


Przyjrzyjmy sie teraz dwóm trajektoriom startującym z bliskich warunków początkowych. Rozpatrzmy ich początkową i asymptotyczną (dla długich czasów) ewolucję.


\begin{verbatim}
# wykresy dla przypadku z tłumieniem
var('x y z')
x01, y01, z01 = 0.1,0.1,0
x02, y02, z02 = 0.11,0.1,0
#
# siła
F = x-x^3
V = -integrate(F,x)
#
# tarcie: parametr gamma
g = 0.1
A = 0.3
w = 1
#
# układ różniczkowych równań ruchu
dx = y
dy = F - g*y + A*cos(z)
dz = w
#
# numeryczne rozwiazanie równań ruchu
T = srange(0,200*pi,0.01)
num1 = desolve_odeint(vector([dx,dy,dz]), [x01,y01,z01], T, [x,y,z])
num2 = desolve_odeint(vector([dx,dy,dz]), [x02,y02,z02], T, [x,y,z])
#
lnum = int(len(num1[:,0])/10)
trans1 = num1[:lnum]
asymp1 = num1[-lnum:]
trans2 = num2[:lnum]
asymp2 = num2[-lnum:]
#
# wykresy funkcji
lt = list_plot(zip(trans1[:,0],trans1[:,1]), plotjoined=1, color='green', axes_labels=['$x(t)$','$v(t)$'], figsize=4)
lt += list_plot(zip(trans2[:,0],trans2[:,1]), plotjoined=1, color='red')
rt = list_plot(zip(T[:lnum],trans1[:,0].tolist()), plotjoined=1, color='green', axes_labels=['$t$','$x(t)$'], figsize=4)
rt += list_plot(zip(T[:lnum],trans2[:,0].tolist()), plotjoined=1, color='red')
lb = list_plot(zip(asymp1[:,0],asymp1[:,1]), plotjoined=1, color='green', axes_labels=['$x(t)$','$v(t)$'], figsize=4)
lb += list_plot(zip(asymp2[:,0],asymp2[:,1]), plotjoined=0, color='red')
rb = list_plot(zip(T[-lnum:],asymp1[:,0].tolist()), plotjoined=1, color='green', axes_labels=['$t$','$x(t)$'], figsize=4)
rb += list_plot(zip(T[-lnum:],asymp2[:,0].tolist()), plotjoined=1, color='red')
#
html("""Układ równań różniczkowych
$\dot{x} = %s$
$\dot{y} = %s$
$\dot{z} = %s$
z różnymi warunkami początkowymi
<span style="color:green;">$(x_{01},y_{01},z_{01}) = (%.2f,%.2f,%.2f)$</span>
<span style="color:red;">$(x_{02},y_{02},z_{02}) = (%.2f,%.2f,%.2f)$</span>
"""%(dx,dy,dz,x01,y01,z01,x02,y02,z02))
html.table([[lt,rt],[lb,rb]])
\end{verbatim}


Na dwóch górnych diagramach przedstawioną reżim krótkich czasów. Ponieważ 2 warunki początkowe nieco się różnią, więc początkowa ewolucja nieco się różni. Kolor czerwony i zielony jest rozróżnialny na prawym górnym rysunku pokazującym ewolucję $x(t)$ dla krótkich czasów.  Jeżeli przyjrzymy się reżimowy długich czasów (dwa dolne diagramy) to zauważymy duże podobieństwo w ewolucji: krzywe fazowe są zamknięte więc jest to prosty ruch periodyczny, przypominający nieco zdeformowaną funkcję typu $\sin(\alpha t)$ czy też $\cos(\alpha t)$. Jest to funkcja okresowa z charakterystycznym jednym jedynym  okresem $T$. Dlatego mówimy, że jest ruch periodyczny o okresie 1. Dwie krzywe $x(t)$ na dolnym prawym rysunku nie są rozróżnialne.

Można zrobić doświadczenie numeryczne i wybierać różne warunki początkowe. Zobaczymy, że trajektorie dążą do tego samego okresowego rozwiązania, są przyciagane do tego okresowego rozwiązania. Innymi słowy, ta krzywa fazowa o okresie 1  jest ATRAKTOREM.  Atraktor ten nazywa się periodycznym atraktorem o okresie 1 lub 1-okresowym  atraktorem. Można by postawić pytanie: jak wygląda basen przyciągania dla tego atraktora. Aby dać odpowiedź na to pytanie należy zbadać numerycznie np. kwadrat warunków początkowych  $(x_0, y_0)$ i wybrać te warunki początkowe które dążą do powyższej krzywej fazowej o okresie 1. Okazuje się, że basen przyciągania jest ``porządnym'' zbiorem, którego brzeg jest gładką krzywą, podobnie jak w przypadku zilustrowanym powyżej dla układu tylko z tarciem, bez siły okresowej.


\subsection{Ruch periodyczny o okresie 3}
\label{ch2/chII011:ruch-periodyczny-o-okresie-3}
Załóżmy następujące wartości parametrów:
\phantomsection\label{ch2/chII011:equation-eqn20}\begin{gather}
\begin{split}\gamma = 0.22, \qquad A = 0.3, \qquad \omega_0 = 1\end{split}\label{ch2/chII011-eqn20}
\end{gather}
W tym przypadku obserwujemy także periodyczny ruch, ale nieco bardziej skomplikowany. Nie jest to prosty periodyczny ruch, ale tzw. ruch o okresie 3, tzn. teraz okres jest 3 razy dłuższy niż w poprzednim przypadku.


\begin{verbatim}
# wykresy dla przypadku z tłumieniem
var('x y z')
x0, y0, z0 = 0.1,0.1,0
kolor = 'red'
#
# siła
F = x-x^3
V = -integrate(F,x)
#
# tarcie: parametr gamma
g = 0.22
A = 0.3
w = 1
#
# układ różniczkowych równań ruchu
dx = y
dy = F - g*y + A*cos(z)
dz = w
#
# numeryczne rozwiazanie równań ruchu
T = srange(0,20*pi,0.01)
num = desolve_odeint(vector([dx,dy,dz]), [x0,y0,z0], T, [x,y,z])
#
# wykresy funkcji
xmin = 1.5
lt  = plot(V, (x,-xmin,xmin), figsize=4)
lt += point((x0,V(x=x0)), color=kolor, size=50, axes_labels=['$x$','$V(x)$'])
lb  = list_plot(zip(num[:,0],num[:,1]), plotjoined=1, color=kolor, axes_labels=['$x(t)$','$v(t)$'], figsize=4)
rt  = list_plot(zip(T,num[:,0].tolist()), plotjoined=1, color=kolor, axes_labels=['$t$','$x(t)$'], figsize=4)
rb  = list_plot(zip(T,num[:,1].tolist()), plotjoined=1, color=kolor, axes_labels=['$t$','$v(t)$'], figsize=4)
#
html("""Układ równań różniczkowych
$\dot{x} = %s$
$\dot{y} = %s$
$\dot{z} = %s$
z warunkami początkowymi
$(x_0,y_0,z_0) = (%.2f,%.2f,%.2f)$
"""%(dx,dy,dz,x0,y0,z0))
html.table([[lt,rt],[lb,rb]])
\end{verbatim}


I znów zobaczymy, jak początkowa ewolucja różni się od tej po długim czasie.


\begin{verbatim}
# wykresy dla przypadku z tłumieniem
var('x y z')
x01, y01, z01 = 0.10,0.1,0
x02, y02, z02 = 0.11,0.1,0
#
# siła
F = x-x^3
V = -integrate(F,x)
#
# tarcie: parametr gamma
g = 0.22
A = 0.3
w = 1
#
# układ różniczkowych równań ruchu
dx = y
dy = F - g*y + A*cos(z)
dz = w
#
# numeryczne rozwiazanie równań ruchu
T = srange(0,200*pi,0.01)
num1 = desolve_odeint(vector([dx,dy,dz]), [x01,y01,z01], T, [x,y,z])
num2 = desolve_odeint(vector([dx,dy,dz]), [x02,y02,z02], T, [x,y,z])
#
lnum = int(len(num1[:,0])/10)
trans1 = num1[:lnum]
asymp1 = num1[-lnum:]
trans2 = num2[:lnum]
asymp2 = num2[-lnum:]
#
# wykresy funkcji
lt = list_plot(zip(trans1[:,0],trans1[:,1]), plotjoined=1, color='green', axes_labels=['$x(t)$','$v(t)$'], figsize=4)
lt += list_plot(zip(trans2[:,0],trans2[:,1]), plotjoined=1, color='red')
rt = list_plot(zip(T[:lnum],trans1[:,0].tolist()), plotjoined=1, color='green', axes_labels=['$t$','$x(t)$'], figsize=4)
rt += list_plot(zip(T[:lnum],trans2[:,0].tolist()), plotjoined=1, color='red')
lb = list_plot(zip(asymp1[:,0],asymp1[:,1]), plotjoined=1, color='green', axes_labels=['$x(t)$','$v(t)$'], figsize=4)
lb += list_plot(zip(asymp2[:,0],asymp2[:,1]), plotjoined=0, color='red')
rb = list_plot(zip(T[-lnum:],asymp1[:,0].tolist()), plotjoined=1, color='green', axes_labels=['$t$','$x(t)$'], figsize=4)
rb += list_plot(zip(T[-lnum:],asymp2[:,0].tolist()), plotjoined=1, color='red')
#
html("""Układ równań różniczkowych
$\dot{x} = %s$
$\dot{y} = %s$
$\dot{z} = %s$
z różnymi warunkami początkowymi
<span style="color:green;">$(x_{01},y_{01},z_{01}) = (%.2f,%.2f,%.2f)$</span>
<span style="color:red;">$(x_{02},y_{02},z_{02}) = (%.2f,%.2f,%.2f)$</span>
"""%(dx,dy,dz,x01,y01,z01,x02,y02,z02))
html.table([[lt,rt],[lb,rb]])
\end{verbatim}


Dla długich czasów, krzywe fazowe są zamknięte, ale nie są  to krzywe typu zdeformowana elipsa.  To są krzywe z 2 pętelkami. Tym niemniej, ruch jest periodyczny.

Podobnie jak poprzednim przypadku, można zrobić doświadczenie numeryczne i wybierać różne warunki początkowe. Zobaczymy, że wiele trajektorii dąży do tej samej  okresowej orbity, są one  przyciagane do tej  zamkniętek krzywej fazowej. Innymi słowy, ta krzywa fazowa o okresie 3  jest ATRAKTOREM.  Atraktor ten nazywa się periodycznym atraktorem o okresie 3 lub 3-okresowym  atraktorem.  Basen przyciągania dla tego atraktora  na płaszczyźnie warunków początkowych $(x_0, y_0)$  jest ``porządnym'' zbiorem o wymiarze 2 (czyli kawałek płaszczyzny), którego brzeg jest gładką krzywą.


\subsection{Ruch chaotyczny}
\label{ch2/chII011:ruch-chaotyczny}
Załóżmy następujące wartości parametrów:
\phantomsection\label{ch2/chII011:equation-eqn21}\begin{gather}
\begin{split}\gamma = 0.25, \qquad A = 0.3, \qquad \omega_0 = 1\end{split}\label{ch2/chII011-eqn21}
\end{gather}
W tym przypadku obserwujemy ruch, który wydaje się być wyjątkowo nieregularny, chaotyczny.


\begin{verbatim}
# wykresy dla przypadku chaotycznego
var('x y z')
x0, y0, z0 = 0.1,0.1,0
kolor = 'firebrick'
#
# siła
F = x-x^3
V = -integrate(F,x)
#
# tarcie: parametr gamma
g = 0.25
A = 0.3
w = 1
#
# układ różniczkowych równań ruchu
dx = y
dy = F - g*y + A*cos(z)
dz = w
#
# numeryczne rozwiazanie równań ruchu
T = srange(0,50*pi,0.01)
num = desolve_odeint(vector([dx,dy,dz]), [x0,y0,z0], T, [x,y,z])
#
# wykresy funkcji
xmin = 1.5
lt  = plot(V, (x,-xmin,xmin), figsize=4)
lt += point((x0,V(x=x0)), color=kolor, size=50, axes_labels=['$x$','$V(x)$'])
lb  = list_plot(zip(num[:,0],num[:,1]), plotjoined=1, color=kolor, axes_labels=['$x(t)$','$v(t)$'], figsize=4)
rt  = list_plot(zip(T,num[:,0].tolist()), plotjoined=1, color=kolor, axes_labels=['$t$','$x(t)$'], figsize=4)
rb  = list_plot(zip(T,num[:,1].tolist()), plotjoined=1, color=kolor, axes_labels=['$t$','$v(t)$'], figsize=4)
#
html("""Układ równań różniczkowych
$\dot{x} = %s$
$\dot{y} = %s$
$\dot{z} = %s$
z warunkami początkowymi
$(x_0,y_0,z_0) = (%.2f,%.2f,%.2f)$
"""%(dx,dy,dz,x0,y0,z0))
html.table([[lt,rt],[lb,rb]])
\end{verbatim}


Zobaczmy, jak tym razem ewoluują rozwiązania o 2 bliskich warunkach początkowych.


\begin{verbatim}
var('x y z')
x01, y01, z01 = 0.1,0.1,0
x02, y02, z02 = 0.11,0.1,0
#
# siła
F = x-x^3
V = -integrate(F,x)
#
# tarcie: parametr gamma
g = 0.25
A = 0.3
w = 1
#
# układ różniczkowych równań ruchu
dx = y
dy = F - g*y + A*cos(z)
dz = w
#
# numeryczne rozwiazanie równań ruchu
T = srange(0,200*pi,0.01)
num1 = desolve_odeint(vector([dx,dy,dz]), [x01,y01,z01], T, [x,y,z])
num2 = desolve_odeint(vector([dx,dy,dz]), [x02,y02,z02], T, [x,y,z])
#
lnum = int(len(num1[:,0])/10)
trans1 = num1[:lnum]
asymp1 = num1[-lnum:]
trans2 = num2[:lnum]
asymp2 = num2[-lnum:]
#
# wykresy funkcji
lt = list_plot(zip(trans1[:,0],trans1[:,1]), plotjoined=1, color='green', axes_labels=['$x(t)$','$v(t)$'], figsize=4)
lt += list_plot(zip(trans2[:,0],trans2[:,1]), plotjoined=1, color='red')
rt = list_plot(zip(T[:lnum],trans1[:,0].tolist()), plotjoined=1, color='green', axes_labels=['$t$','$x(t)$'], figsize=4)
rt += list_plot(zip(T[:lnum],trans2[:,0].tolist()), plotjoined=1, color='red')
lb = list_plot(zip(asymp1[:,0],asymp1[:,1]), plotjoined=1, color='green', axes_labels=['$x(t)$','$v(t)$'], figsize=4)
lb += list_plot(zip(asymp2[:,0],asymp2[:,1]), plotjoined=1, color='red')
rb = list_plot(zip(T[-lnum:],asymp1[:,0].tolist()), plotjoined=1, color='green', axes_labels=['$t$','$x(t)$'], figsize=4)
rb += list_plot(zip(T[-lnum:],asymp2[:,0].tolist()), plotjoined=1, color='red')
#
html("""Układ równań różniczkowych
$\dot{x} = %s$
$\dot{y} = %s$
$\dot{z} = %s$
z różnymi warunkami początkowymi
<span style="color:green;">$(x_{01},y_{01},z_{01}) = (%.2f,%.2f,%.2f)$</span>
<span style="color:red;">$(x_{02},y_{02},z_{02}) = (%.2f,%.2f,%.2f)$</span>
"""%(dx,dy,dz,x01,y01,z01,x02,y02,z02))
html.table([[lt,rt],[lb,rb]])
\end{verbatim}


Początkowa ewolucja dwóch rozwiązań jest nierozróżnialna (ponieważ 2 warunki początkowe są bardzo blisko siebie). Po pewnym charakterystycznym czasie, zwanym czasem Lapunowa, trajektorie zaczynają różnić się coraz bardziej, zaczynają rozbiegać się: patrz trajektoria czerwona i zielona na dolnym prawym rysunku.
\begin{figure}[htbp]
\centering
\capstart

\includegraphics{chaos_traj.png}
\caption{Schematyczne trajektorie w reżimie chaotycznym.}\end{figure}

W reżimie chaotycznym, te dwie trajektorie oddalają się od siebie w eksponencjalnie szybkim tempie określonym przez zależność:
\phantomsection\label{ch2/chII011:equation-eqn22}\begin{gather}
\begin{split}|x_1(t) - x_2(t)| = |x_1(0) - x_2(0)|\mbox{e}^{\lambda t}, \qquad \lambda > 0\end{split}\label{ch2/chII011-eqn22}
\end{gather}
lub
\phantomsection\label{ch2/chII011:equation-eqn23}\begin{gather}
\begin{split}|\Delta x(t)| = |\Delta x_0|\mbox{e}^{\lambda t}, \qquad \lambda > 0\end{split}\label{ch2/chII011-eqn23}
\end{gather}
gdzie $\lambda$ nazywa sie wykładnikiem Lapunowa.

Różnice w ewolucji stają się zbyt duże i pojawia się dylemat: która trajektoria jest właściwa, skoro nasza aparatura nie rozróżnia bliskich warunków początkowych. Determinizm staje się złudnym. Nie możemy przewidywać właściwej ewolucji układu.

Przedstawiony powyżej reżim chaotyczny nie jest jedyny. W układzie istnieje wiele takich wartości parametrów $(\gamma, A, \omega)$, dla których pojawia się ruch chaotyczny. Należy nadmienić, że dla długich czasów  wiele trajektorii generowanych przez różne warunki początkowe zachowuje się bardzo podobnie, wiele trajektorii jest przyciąganych. Tu także istnieje atraktor i jego basen przyciągania. Jednakże ten atraktor jest dziwny: jego wymiar nie jest liczbą całkowitą i atraktor  jest fraktalem. Dlatego nazywa się dziwnym atraktorem.  Brzeg basenu przyciągania tego atraktora też ma dziwną strukturę  i jego wymiar jest fraktalny.
\setbox0\vbox{
\begin{minipage}{0.95\linewidth}
\textbf{Zadania}

\medskip

\begin{enumerate}
\item {} 
Niech $\gamma = 0.1, \quad \omega_0 =1.4 , \quad (x_0, y_0, z_0) = (-0.5, -0.2, 0)$.
Zmieniaj parametr $A=0.1,  0.32,  0.338,  0.35$.

Obserwuj scenariusz  podwojenia okresu:
\begin{enumerate}
\item {} 
pojawia się atraktor periodyczny o okresie 1.

\item {} 
pojawia się atraktor periodyczny o okresie 2.

\item {} 
pojawia się atraktor periodyczny o okresie 4.

\item {} 
pojawia się atraktor periodyczny o okresie 8 (trudno  trafić).

\item {} 
pojawia się ruch nieregularny, chaotyczny.

\end{enumerate}

\item {} 
Zbadaj zachowanie się układu dla następujących wartości parametrów:
$\gamma = 1.35  -  1.38, \quad A=1, \quad \omega_0 =1, \quad (x_0, y_0, z_0) = (0.0, 0.5, 0)$.

\item {} 
To samo dla wartości
$\gamma = 0.5, \quad A=0.34875, \quad \omega_0 =1, \quad (x_0, y_0, z_0) = (0,  0, 0)$

\end{enumerate}
\end{minipage}}
\begin{center}\setlength{\fboxsep}{5pt}\shadowbox{\box0}\end{center}



\renewcommand{\indexname}{Indeks}
\printindex
\end{document}
